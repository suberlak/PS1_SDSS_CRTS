%%
%% Beginning of file 'sample62.tex'
%%
%% Modified 2018 January
%%
%% This is a sample manuscript marked up using the
%% AASTeX v6.2 LaTeX 2e macros.
%%
%% AASTeX is now based on Alexey Vikhlinin's emulateapj.cls 
%% (Copyright 2000-2015).  See the classfile for details.

%% AASTeX requires revtex4-1.cls (http://publish.aps.org/revtex4/) and
%% other external packages (latexsym, graphicx, amssymb, longtable, and epsf).
%% All of these external packages should already be present in the modern TeX 
%% distributions.  If not they can also be obtained at www.ctan.org.

%% The first piece of markup in an AASTeX v6.x document is the \documentclass
%% command. LaTeX will ignore any data that comes before this command. The 
%% documentclass can take an optional argument to modify the output style.
%% The command below calls the preprint style  which will produce a tightly 
%% typeset, one-column, single-spaced document.  It is the default and thus
%% does not need to be explicitly stated.
%%
%%
%% using aastex version 6.2
\documentclass[twocolumn]{aastex62}
\usepackage{natbib}
%% The default is a single spaced, 10 point font, single spaced article.
%% There are 5 other style options available via an optional argument. They
%% can be envoked like this:
%%
%% \documentclass[argument]{aastex62}
%% 
%% where the layout options are:
%%
%%  twocolumn   : two text columns, 10 point font, single spaced article.
%%                This is the most compact and represent the final published
%%                derived PDF copy of the accepted manuscript from the publisher
%%  manuscript  : one text column, 12 point font, double spaced article.
%%  preprint    : one text column, 12 point font, single spaced article.  
%%  preprint2   : two text columns, 12 point font, single spaced article.
%%  modern      : a stylish, single text column, 12 point font, article with
%% 		  wider left and right margins. This uses the Daniel
%% 		  Foreman-Mackey and David Hogg design.
%%  RNAAS       : Preferred style for Research Notes which are by design 
%%                lacking an abstract and brief. DO NOT use \begin{abstract}
%%                and \end{abstract} with this style.
%%
%% Note that you can submit to the AAS Journals in any of these 6 styles.
%%
%% There are other optional arguments one can envoke to allow other stylistic
%% actions. The available options are:
%%
%%  astrosymb    : Loads Astrosymb font and define \astrocommands. 
%%  tighten      : Makes baselineskip slightly smaller, only works with 
%%                 the twocolumn substyle.
%%  times        : uses times font instead of the default
%%  linenumbers  : turn on lineno package.
%%  trackchanges : required to see the revision mark up and print its output
%%  longauthor   : Do not use the more compressed footnote style (default) for 
%%                 the author/collaboration/affiliations. Instead print all
%%                 affiliation information after each name. Creates a much
%%                 long author list but may be desirable for short author papers
%%
%% these can be used in any combination, e.g.
%%
%% \documentclass[twocolumn,linenumbers,trackchanges]{aastex62}
%%
%% AASTeX v6.* now includes \hyperref support. While we have built in specific
%% defaults into the classfile you can manually override them with the
%% \hypersetup command. For example,
%%
%%\hypersetup{linkcolor=red,citecolor=green,filecolor=cyan,urlcolor=magenta}
%%
%% will change the color of the internal links to red, the links to the
%% bibliography to green, the file links to cyan, and the external links to
%% magenta. Additional information on \hyperref options can be found here:
%% https://www.tug.org/applications/hyperref/manual.html#x1-40003
%%
%% If you want to create your own macros, you can do so
%% using \newcommand. Your macros should appear before
%% the \begin{document} command.
%%
\newcommand{\vdag}{(v)^\dagger}
\newcommand\aastex{AAS\TeX}
\newcommand\latex{La\TeX}

%% Reintroduced the \received and \accepted commands from AASTeX v5.2
\received{January 1, 2019}
\revised{January 17, 2019}
\accepted{February 1, 2019}%\today}
%% Command to document which AAS Journal the manuscript was submitted to.
%% Adds "Submitted to " the arguement.
\submitjournal{ApJ}

%% Mark up commands to limit the number of authors on the front page.
%% Note that in AASTeX v6.2 a \collaboration call (see below) counts as
%% an author in this case.
%
%\AuthorCollaborationLimit=3
%
%% Will only show Schwarz, Muench and "the AAS Journals Data Scientist 
%% collaboration" on the front page of this example manuscript.
%%
%% Note that all of the author will be shown in the published article.
%% This feature is meant to be used prior to acceptance to make the
%% front end of a long author article more manageable. Please do not use
%% this functionality for manuscripts with less than 20 authors. Conversely,
%% please do use this when the number of authors exceeds 40.
%%
%% Use \allauthors at the manuscript end to show the full author list.
%% This command should only be used with \AuthorCollaborationLimit is used.

%% The following command can be used to set the latex table counters.  It
%% is needed in this document because it uses a mix of latex tabular and
%% AASTeX deluxetables.  In general it should not be needed.
%\setcounter{table}{1}

%%%%%%%%%%%%%%%%%%%%%%%%%%%%%%%%%%%%%%%%%%%%%%%%%%%%%%%%%%%%%%%%%%%%%%%%%%%%%%%%
%%
%% The following section outlines numerous optional output that
%% can be displayed in the front matter or as running meta-data.
%%
%% If you wish, you may supply running head information, although
%% this information may be modified by the editorial offices.
\shorttitle{Improved DRW for S82 QSO}
\shortauthors{Suberlak et al.}
%%
%% You can add a light gray and diagonal water-mark to the first page 
%% with this command:
% \watermark{text}
%% where "text", e.g. DRAFT, is the text to appear.  If the text is 
%% long you can control the water-mark size with:
%  \setwatermarkfontsize{dimension}
%% where dimension is any recognized LaTeX dimension, e.g. pt, in, etc.
%%
%%%%%%%%%%%%%%%%%%%%%%%%%%%%%%%%%%%%%%%%%%%%%%%%%%%%%%%%%%%%%%%%%%%%%%%%%%%%%%%%

%% This is the end of the preamble.  Indicate the beginning of the
%% manuscript itself with \begin{document}.

\begin{document}

\title{Improving Damped Random Walk parameters for SDSS Stripe82 Quasars with baseline extension with PanStarrs1 data. }


%% LaTeX will automatically break titles if they run longer than
%% one line. However, you may use \\ to force a line break if
%% you desire. In v6.2 you can include a footnote in the title.

%% A significant change from earlier AASTEX versions is in the structure for 
%% calling author and affilations. The change was necessary to implement 
%% autoindexing of affilations which prior was a manual process that could 
%% easily be tedious in large author manuscripts.
%%
%% The \author command is the same as before except it now takes an optional
%% arguement which is the 16 digit ORCID. The syntax is:
%% \author[xxxx-xxxx-xxxx-xxxx]{Author Name}
%%
%% This will hyperlink the author name to the author's ORCID page. Note that
%% during compilation, LaTeX will do some limited checking of the format of
%% the ID to make sure it is valid.
%%
%% Use \affiliation for affiliation information. The old \affil is now aliased
%% to \affiliation. AASTeX v6.2 will automatically index these in the header.
%% When a duplicate is found its index will be the same as its previous entry.
%%
%% Note that \altaffilmark and \altaffiltext have been removed and thus 
%% can not be used to document secondary affiliations. If they are used latex
%% will issue a specific error message and quit. Please use multiple 
%% \affiliation calls for to document more than one affiliation.
%%
%% The new \altaffiliation can be used to indicate some secondary information
%% such as fellowships. This command produces a non-numeric footnote that is
%% set away from the numeric \affiliation footnotes.  NOTE that if an
%% \altaffiliation command is used it must come BEFORE the \affiliation call,
%% right after the \author command, in order to place the footnotes in
%% the proper location.
%%
%% Use \email to set provide email addresses. Each \email will appear on its
%% own line so you can put multiple email address in one \email call. A new
%% \correspondingauthor command is available in V6.2 to identify the
%% corresponding author of the manuscript. It is the author's responsibility
%% to make sure this name is also in the author list.
%%
%% While authors can be grouped inside the same \author and \affiliation
%% commands it is better to have a single author for each. This allows for
%% one to exploit all the new benefits and should make book-keeping easier.
%%
%% If done correctly the peer review system will be able to
%% automatically put the author and affiliation information from the manuscript
%% and save the corresponding author the trouble of entering it by hand.



\correspondingauthor{Krzysztof Suberlak}
\email{suberlak@uw.edu}

\author[0000-0002-9589-1306]{Krzysztof L. Suberlak}
\affiliation{Department of Astronomy \\ University of Washington \\ Seattle, WA 98195, USA}


\author{\v{Z}eljko Ivezi\'c}
\affiliation{Department of Astronomy \\ University of Washington \\ Seattle, WA 98195, USA}



\author{Chelsea MacLeod}
\affiliation{Harvard Smithsonian Center for Astrophysics \\ 60 Garden St, Cambridge, MA 02138, USA}




% Abstract of the paper
\begin{abstract}

 %Aim: Improve on DRW parameters reported in \cite{macleod2011} by an increase of the QSO light curve baseline.  We compare the tools used to fitting for $\tau$ and $SF_{\infty}$ to those of \cite{kozlowski2017a}.
\end{abstract}


%%%%%%%%%%%%%%%%% NEW NEW NEW BODY %%%%%%%%%%%%%%%%%%%%%%%%%%%%%%%%%

\section{Introduction}

Quasars are variable.  Their light curves have been successfully described using the Damped Random Walk (DRW) model \citep{kelly2009, macleod2010, kozlowski2010, zu2011, kasliwal2015a}. The origin of variability is debated, with thermal origin being the favorite explanation \citep{kelly2013}, connected to the inhomogeneity of the accretion disk \citep{dexter2011}, or even magnetically elevated disks \citep{dexter2019}. 

The DRW parameters have been linked to the physical quasar properties: \citet{macleod2010} found correlations of the characteristic timescale and variability amplitude  to the black hole mass, and quasar luminosity.

Inherent variability, and modelling it as  a DRW, is also a reliable way to distinguish quasars from stars based on  optical photometry \citep{macleod2011}.  In that case the fit biases are less important than the fact that DRW timescale and amplitude for QSO are order of  magnitude different from stars \citep{macleod2011}.  It is especially useful for selecting quasars in the intermediate redshift range that could not be easily identified by color-color diagrams \citep{sesar2007,yang2017}.  

Accurate QSO population studies are important for measurement of Quasar Luminosity Function, and variability has been used before to increase the completeness of quasar selection  \citep{ ross2013, palanque2013, alsayyad2016, mcgreer2013, mcgreer2018}. 


Because DRW is a stochastic process, two light curves with identical DRW parameters will not look identical. However, once can still fit the available data with the DRW model and recover fit parameters. It has been found (eg. \citet{macleod2011, kozlowski2010, kozlowski2017a}) that regardless of method,  we can most reliably recover input parameters if we use the longest light curve baseline possible. A rule of thumb is that the light curve has to be at least  ten  times longer than the recovered timescale.  We confirm this observation with simulations of DRW light curves spanning a variety of ratios of input timescale to light curve lenght.  

The light curve baseline is the key in an unbiased recovery of light curve parameters. As it has been  8 years since \citet{macleod2010} have published their research,   we can now benefit from additional data from other surveys that have observed the same quasars since. We show how combining the SDSS data with CRTS, PTF , PS1,  and simulated LSST data, decreases the bias in recovered parameters.  Thus with added data, extending the baseline on average by 50\%,  we revisit correlations studied by \citet{macleod2010}.  We confirm the general trends, and provide forecast for improvement with the advent of ZTF, LSST. Extended baseline is the advantage that is not afforded by studies only using single survey data (eg. \citet{hernitschek2016}



\section{Methods}
\subsection{DRW as a Gaussian Process}
DRW (Ornstein–Uhlenbeck process) can be understood as a member of a class of Gaussian Processes (GP). Each GP is described by a kernel - a covariance function that contains a measure of correlation between two points $x_{n}$, $x_{m}$, separted by $\Delta t_{nm}$. For the  DRW process, the kernel is 

\begin{eqnarray}
k(\Delta t_{nm}) &=& a \exp{(-\Delta t_{tm} / \tau)} \\
                 &=& \sigma^{2}\exp{(-\Delta t_{tm} / \tau)}  \\
                 &=& \sigma^{2} ACF(\Delta t_{tm})
\end{eqnarray} 

Here $a$ or $\sigma^{2}$ is an amplitude of correlation decay as a function of $t_{tm}$,  while $\tau$ is the characteristic timescale over which correlation drops by $1/e$. For a DRW,  the correlation function $k(\Delta t_{nm})$ is also related to the autocorrelation function $ACF$. 

Related to $k(\Delta t_{nm})$ is the   structure function of the DRW process (see \citet{macleod2012, bauer2009, graham2015a} for an overview) , which expresses the rms of  magnitude differences $\Delta m$ as a function of temporal separation $\Delta t$, is : 

\begin{equation}
SF(\Delta t) = SF_{\infty} (1-\exp{(-|\Delta t|/\tau)})^{1/2}
\end{equation}

where $SF_{\infty}$ is the asymptotic value of $SF$ for large time lags. It is known that for QSOs SF follows approximately power law, $SF \propto \Delta t^{\beta}$,  and it levels out for large $\Delta t$) (see \citet{macleod2012}).  Note that $SF_\infty = \sqrt{2} \sigma$  in the above. 


To estimate the  DRW  parameters and fit simulated or real data we employ Celerite - a new fast GP modelling tool\citep{foreman2017}. Combined with the DRW kernel this is similar to the method used by \citet{rybicki1992, kozlowski2010, macleod2010} - like in previous work, we use a  prior on $\sigma$ and $\tau$ uniform in log space.  The main difference is that rather than adopting the Maximum A-Posteriori (MAP) as the 'best-fit' value for sought parameters,  we find the expectation value of the marginalized log posterior. If the posterior space was a 2D Gaussian in $\sigma$, $\tau$ space, the expectation value would coincide with the maximum of the log posterior. However, due to non-Gaussian shape of the log posterior, we find that the expectation value is a better   estimate of $\sigma$ and $\tau$  rather than MAP. 

\subsection{The impact of light curve baseline}\label{sec:baseline}

\cite{kozlowski2017a}  reports that we cannot trust any results of DRW fitting unless the light curve length is at least ten times longer than the characteristic timescale. We confirm these generic trends by repeating \cite{kozlowski2017a} simulation setup. We  model 10 000  DRW light curves with fixed length (baseline) $t_{exp}=8$ years, $SF_{\infty} = 0.2$ mag,  but with different input timescales. We parametrize the ratio of timescale to baseline by $\rho = \tau / t_{exp}$. Given that the baselines for all light curves are fixed,  by selecting different input $\tau$ we probe a logarithmic grid in $\rho \in   \{ 0.01 : 15\}$. For each of  100 distinct values of $\rho$ we perform 100 light curve realizations.  

To simulate observational conditions  we add to the true underlying signal  $s(t)$ a noise offset, $n(t)$.  Like \cite{kozlowski2017a},  we assume $n(t)$ to be drawn from a Gaussian distribution $\mathcal{N}(0,\sigma(t))$ with a width $\sigma(t)$ , corresponding to the  photometric uncertainty at the given epoch,  $e(t)$) : 

\begin{equation}
y(t) = s(t) + n(t) 
\end{equation}

The $s(t)$ is found by iterating over the array of time steps $t$.  At each step, we draw a point from a Gaussian distribution, for which the mean and standard deviation are re-calculated at each timestep. Starting at $t_{0}$, the signal is equal to the mean magnitude, $s_{0} = m$. After a timestep $\Delta t_{i} = t_{i+1} - t_{i}$, the signal $s_{i+1}$ is drawn from  $\mathcal{N}(loc, stdev)$, with : 

\begin{equation}
loc = s_{i} e ^ { - r  }  + m \left( 1 - e ^{ - r }\right)
\end{equation}

and 

\begin{equation}
stdev^{2} =  0.5  \, \mathrm{SF}_{\infty}^{2} \left( 1 - e ^{  - 2 r  }  \right)
\end{equation}

where  $r = \Delta t_{i} / \tau$, $\tau$ is the damping timescale,   $SF_{\infty}$ is the  variability amplitude , and $m$ the mean magnitude.  This follows the formalizm in \cite{kelly2009} (eqs. A4 and A5) as well as in \cite{macleod2010} (Sec. 2.2 ), and is equivalent to the setup of  \cite{kozlowski2017a}. 


We adopt SDSS S82-like cadence with N=60 epochs, or OGLE-III like cadence with N=445 epochs.  The errors were set by the adopted mean magnitudes, $r=17$ and $I=18$ , as in \cite{kozlowski2017a} :

\begin{eqnarray}
\sigma_{SDSS}^{2} &=& 0.013^{2} + \exp{(2 (r-23.36))} \\
\sigma_{OGLE}^{2} &=& 0.004^{2} + \exp{(1.63 (I - 22.55))}
\end{eqnarray}


\begin{figure*}
\plottwo{figs/SDSS_Jeff1_expectation.png}{figs/OGLE_Jeff1_expectation.png}
\caption{Probing the parameter space of $\rho = \tau / t_{exp}$, with a simulation of  10 000 light curves : 100 light curves per each of 100 $\rho$ values spaced uniformly in logarithmic space between $\rho \in   \{ 0.01 : 15\}$ . Thus with the baseline $t_{exp}$ set to 8 years,  we sample a range of 100 input timescales, as in \citet{kozlowski2017a}. Left panel shows the SDSS-like cadence with N=60 points, and the right panel OGLE-like cadence with N=445 points. The dotted horizontal and solid vertical lines represent $\rho = 0.1$, i.e. the baseline is ten times longer than considered timescale. The diagonal line is $y=x$, i.e. the line that would be followed if the recovered  $\rho$ ($\tau$) was exactly the same as  
the input $\rho$ ($\tau$).} 
\label{fig:Fig2kozlowski}
\end{figure*}

Fig~\ref{fig:Fig2kozlowski} shows that as stipulated in \cite{kozlowski2017a},  the recovered $\rho$ becomes meaningless ('unconstrained') if the available baseline is not at least ten times longer than the underlying timescale. It also means that by extending the baseline we can move from the biased region to the unbiased regime. 

Encouraged by this result, we extend the baselines of  quasar light curves, and revisit relations studied by  \cite{macleod2011} and \cite{hernitschek2016}. 

\section{Data}
\subsection{Surveys}
We focus on data pertaining to a 290 deg$^{2}$ region of southern sky, repeatedly observed by the SDSS between 1998 and 2008. Originally aimed at supernova discovery, objects in this area, known as Stripe82 (S82), were  re-observed on average 60 times (see \citealt{macleod2012} Sec. 2.2 for overview, and \citealt{annis2014} for details ) . Availability of well-calibrated \citep{ivezic2007}, long-baseline light curves spurred variability research (see \citealt{sesar2007}). The catalog prepared by \citep{schneider2008} as part of DR9  contains 9258 spectroscopically confirmed quasars.  

\begin{figure}%[ht!]
\plotone{figs/illustrate_epochal_coverage.png}
\caption{Raw photometric measurements for quasars in Stripe 82 from SDSS(r),  PS1(gri),  PTF(gR), CRTS(V).}
\label{fig:rawBaselines}
\end{figure} 

\begin{figure*}
\plotone{figs/lightcurves_extent.png}
\caption{The contribution to quasars light curve baseline from surveys, including the planned LSST coverage. Vertical offset is arbitrary. Note how PS1 and PTF extend the baseline of SDSS by approximately $50\%$, and how inclusion of LSST triples the SDSS baseline. }
\label{fig:lcExtent}
\end{figure*} 


We extend the SDSS  light curves with PanSTARRS (PS1) \citep{chambers2011,flewelling2018}, CRTS \citep{drake2009}, and PTF \citep{rau2009}. We find 9248 PS1 matches, 6455 PTF matches, and 7737 CRTS matches to SDSS S82 quasars. There are  6444 quasars with SDSS-PS1-PTF-CRTS data.  Fig.~\ref{fig:rawBaselines}  shows the distribution of raw epochs, and Fig~\ref{fig:lcExtent} the  baseline coverage of various surveys.    Each survey uses a unique set of bandpasses and cadences : SDSS light curves contain near-simultaneous $\{u,g,r,i,z\}_{SDSS}$ , and the other are  non-simultaneous : $\{g,r,i,z,y\}_{PS1}$ ,  $\{g,R\}_{PTF}$, $V_{CRTS}$.  

\subsection{Photometric offsets}

To utilize all data we define a common 'target' bandpass. SDSS r band is closest to PS1 r, PTFr, CRTSV. For this reason we translate photometry from nearby filters ($\{g,R\}_{PTF}$, $\{g,r,i\}_{PS1}$, $V_{CRTS}$) to the  'master' $r_{SDSS}$  band.

With two photometric systems, eg. SDSS(ugriz), and PS1(grizy),  we can find offsets (or color terms) from one to another. Consider SDSS as target system,  PS1 as the auxiliary system, so that we find offsets from PS1 to SDSS. This amounts to creating 'synthetic' SDSS bands from PS1, using the SDSS color to spread the stellar locus. More generally, we would always use the color of the target system : 

\begin{equation}
r_{PS1} -  r_{SDSS} = f ( SDSS (g-i ))
\end{equation}

the function is a polynomial fitted to the stellar locus on the plot of $SDSS (g-i )$ vs $r_{PS1} -  r_{SDSS} $. We use $SDSS(g-i)$ because is provides a larger wavelength baseline than $(g-r)$. 

Note that there are other possible choices for the target band and the intermediate color to spread the stellar locus. For instance, rewriting the above as  $m - s = f(x)$, \cite{tonry2012} derived offsets from SDSS to PS1  using $x$ = SDSS($g-r$), $m = PS1 \{g,r,i,z,y\}$, and $s$ = SDSS($r$).

Since quasars occupy a blue region in the color-color diagram (Fig.~\ref{fig:quasarColors}), we calculate photometric offsets specifically for this region of the spectrum.  We only show the color-magnitude diagrams  for PS1 offsets  (Fig. ~\ref{fig:offsetsPS1}) since we choose not to include CRTS and PTF data in the final sample. 

% QSO COLORS 
\begin{figure*}
\plotone{figs/SDSS_S82_CMD_qso_stars.png}
\caption{Regions occupied in color-color space by S82 quasars (colors) and standard stars (contours) 	\citep{schneider2010}. We show only 10 000  randomly chosen stars from the  full 1 mln + standard stars catalog \citealt{ivezic2007}. }
\label{fig:quasarColors}
\end{figure*} 


% PS1 
\begin{figure*}
\plotone{figs/Offsets_PS1-SDSSr_SDSSgi_ext-NO.png}
\caption{The SDSS-PS1 offsets. We plot only bright stars that have SDSS(r) < 19, and that fulfill  mErr * sqrt(Nobs) < 0.03 . Each panel plots about 6000 stars of the 47000 CRTS S82 stars . Vertical dashed lines mark the region in SDSS color space occupied by quasars (see Fig.~\ref{fig:quasarColors}), used to fit the stellar locus with a polynomial. }
\label{fig:offsetsPS1}
\end{figure*} 



%NOTE ABOUT EXTINCTION:  due to dust present between us and the standard stars (or background quasars), the observed light will appear slightly redder because dust preferentially scatters blue light away. This depends on the location of the source on the  sky and is related to the dust inhomogeneities in the Milky Way. 

%However, in deriving bandpass to bandpass transformation all that matters is the flux received at the Top of the Atmosphere (TOA), for which all photometry is corrected at the pipeline processing level (both for SDSS,  PTF, PS1, and CRTS). In other words, all that matters is that any SED at TOA will have slightly different magnitude (hence colors) depending on whether we observe it with SDSS(r),  PS1(r), or PTF(R).  Therefore, to derive photometric offsets (which are time-independent), we use Sloan colors not corrected for extincton (and likewise, PS1, PTF, etc.) .  

%Correction for insterstellar extinction is only needed for color selection, since otherwise objects would appear to be of the later stellar type (redder) than they really are. To correct for extinction we use the most up-to-date maps of stellar extinction Bayestar17 (Green+2018). These extinction maps are 3D probabilistic, and we assume a uniform distance of 4 kpc for the dust column for all stars.  


\section{Results}

Having established in Sec.~\ref{sec:baseline} that extending the light curve baseline improves the recovery of input DRW parameters,  we combine SDSS light curves with PS1 data. We first consider the theoretical improvement in the fit, simulating DRW light curves for which we select SDSS-PS1 , or SDSS-only sections.  Then we fit the real data with DRW model, and divide by (1+z) to study the timescales in rest frame. 


\subsection{Simulated DRW}

We simulated the DRW using the real cadences and errors corresponding to sections of combined light curves (SDSS,PS1,CRTS,PTF), including the portion corresponding to the forecasted LSST contribution.  For LSST segment we assumed a cadence of 50 epochs per year,  for 10 years (between 2023-2033), and  magnitude-dependent photometric uncertainty from Sec. 3.5 in \citet{lsstscibook}:

\begin{eqnarray}
\sigma_{LSST}(m)^{2} = \sigma_{sys}^{2} + \sigma_{rand}^{2} \\
\sigma_{rand}^{2} = (0.04-\gamma)x + \gamma x^{2} \\
x = 10^{0.4(m-m_{5})}
\end{eqnarray}

with  $\sigma_{sys} = 0.005$, $\gamma=0.039$, $m_{5} = 24.7$ (see Table 3.2  therein).

For all light curves we assumed  $\tau = 575 $ days, $SF_{\infty} = 0.2$ mag (the median of S82 distribution in ~\citet{macleod2010}). Fig.~\ref{fig:simLC} shows an example simulated DRW for SDSS-PS1-LSST.

\begin{figure*}%[ht!]
\plotone{figs/1072282_w_LSST1.png}
\caption{Simulated DRW process sampled at real cadence of SDSS, PS1, and simulated cadence of LSST. To each observed  point we add Gaussian noise corresponding to the reported  heteroscedastic (different for all points) errors for SDSS-PS1, and simulated magnitude-dependent errors for LSST. }
\label{fig:simLC}
\end{figure*} 


In accordance with Fig.~\ref{fig:Fig2kozlowski}, we find that extending the baseline decreases the bias in retrieved DRW parameters  $\tau$ and $\sigma$ - see Fig.~\ref{fig:simLCresults1} and Fig.~\ref{fig:simLCresults2}


\begin{figure*}
\plotone{figs/Simulated_DRW_w_LSST_Jeff2EXP.png}
\caption{Retrieved $\tau$ and $\sigma$  parameters for simulated LCs. }
\label{fig:simLCresults1}
\end{figure*} 

\begin{figure*}
\plotone{figs/validate_macleod2011_Fig18_sdss-ps1-lsst_sim.png}
\caption{Comparison of retrieved parameters in relation to input parameters, shown as Fig.18 in \citet{macleod2011} }
\label{fig:simLCresults2}
\end{figure*} 


\subsection{Real data}

Since the simulations showed that we do decrease the bias by extending the  baseline, we use the combined SDSS-PS1  light curves and show the space occupied by quasars in $SF_{\infty} - \tau - \sigma$ space, which confirms the results of \citet{macleod2011} (Fig.~\ref{fig:realLCresults})


\begin{figure*}
\plottwo{figs/validate_macleod2011_Fig6.png}{figs/validate_macleod2011_Fig14.png}
\caption{Both relations are shown in the observed frame. The left panel is like Fig.6, and the right like Fig.14 in \citet{macleod2011}. }
\label{fig:realLCresults}
\end{figure*} 




%%%%%%%%%%%%%%%%%%%%%%%%%%%%%%%%%%%%%%%%%%%%%%%%%%

%%%%%%%%%%%%%%%%%%%% REFERENCES %%%%%%%%%%%%%%%%%%

% The best way to enter references is to use BibTeX:

\bibliographystyle{aasjournal} 
\bibliography{references}

%%%%%%%%%%%%%%%%%%%%%%%%%%%%%%%%%%%%%%%%%%%%%%%%%%

\end{document}

% End of file `sample62.tex'.
