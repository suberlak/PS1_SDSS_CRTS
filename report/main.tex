%%
%% Beginning of file 'sample62.tex'
%%
%% Modified 2018 January
%%
%% This is a sample manuscript marked up using the
%% AASTeX v6.2 LaTeX 2e macros.
%%
%% AASTeX is now based on Alexey Vikhlinin's emulateapj.cls 
%% (Copyright 2000-2015).  See the classfile for details.

%% AASTeX requires revtex4-1.cls (http://publish.aps.org/revtex4/) and
%% other external packages (latexsym, graphicx, amssymb, longtable, and epsf).
%% All of these external packages should already be present in the modern TeX 
%% distributions.  If not they can also be obtained at www.ctan.org.

%% The first piece of markup in an AASTeX v6.x document is the \documentclass
%% command. LaTeX will ignore any data that comes before this command. The 
%% documentclass can take an optional argument to modify the output style.
%% The command below calls the preprint style  which will produce a tightly 
%% typeset, one-column, single-spaced document.  It is the default and thus
%% does not need to be explicitly stated.
%%
%%
%% using aastex version 6.2
\documentclass[twocolumn]{aastex62}
\usepackage{natbib}
\usepackage[T1]{fontenc}

\let\oldAA\AA
\renewcommand{\AA}{\text{\normalfont\oldAA}}



%% The default is a single spaced, 10 point font, single spaced article.
%% There are 5 other style options available via an optional argument. They
%% can be envoked like this:
%%
%% \documentclass[argument]{aastex62}
%% 
%% where the layout options are:
%%
%%  twocolumn   : two text columns, 10 point font, single spaced article.
%%                This is the most compact and represent the final published
%%                derived PDF copy of the accepted manuscript from the publisher
%%  manuscript  : one text column, 12 point font, double spaced article.
%%  preprint    : one text column, 12 point font, single spaced article.  
%%  preprint2   : two text columns, 12 point font, single spaced article.
%%  modern      : a stylish, single text column, 12 point font, article with
%% 		  wider left and right margins. This uses the Daniel
%% 		  Foreman-Mackey and David Hogg design.
%%  RNAAS       : Preferred style for Research Notes which are by design 
%%                lacking an abstract and brief. DO NOT use \begin{abstract}
%%                and \end{abstract} with this style.
%%
%% Note that you can submit to the AAS Journals in any of these 6 styles.
%%
%% There are other optional arguments one can envoke to allow other stylistic
%% actions. The available options are:
%%
%%  astrosymb    : Loads Astrosymb font and define \astrocommands. 
%%  tighten      : Makes baselineskip slightly smaller, only works with 
%%                 the twocolumn substyle.
%%  times        : uses times font instead of the default
%%  linenumbers  : turn on lineno package.
%%  trackchanges : required to see the revision mark up and print its output
%%  longauthor   : Do not use the more compressed footnote style (default) for 
%%                 the author/collaboration/affiliations. Instead print all
%%                 affiliation information after each name. Creates a much
%%                 long author list but may be desirable for short author papers
%%
%% these can be used in any combination, e.g.
%%
%% \documentclass[twocolumn,linenumbers,trackchanges]{aastex62}
%%
%% AASTeX v6.* now includes \hyperref support. While we have built in specific
%% defaults into the classfile you can manually override them with the
%% \hypersetup command. For example,
%%
%%\hypersetup{linkcolor=red,citecolor=green,filecolor=cyan,urlcolor=magenta}
%%
%% will change the color of the internal links to red, the links to the
%% bibliography to green, the file links to cyan, and the external links to
%% magenta. Additional information on \hyperref options can be found here:
%% https://www.tug.org/applications/hyperref/manual.html#x1-40003
%%
%% If you want to create your own macros, you can do so
%% using \newcommand. Your macros should appear before
%% the \begin{document} command.
%%
\usepackage{hyperref}
\usepackage[capitalise]{cleveref}

\newcommand{\vdag}{(v)^\dagger}
\newcommand\aastex{AAS\TeX}
\newcommand\latex{La\TeX}

%% Reintroduced the \received and \accepted commands from AASTeX v5.2
\received{January 1, 2019}
\revised{January 17, 2019}
\accepted{February 1, 2019}%\today}
%% Command to document which AAS Journal the manuscript was submitted to.
%% Adds "Submitted to " the arguement.
\submitjournal{ApJ}

%% Mark up commands to limit the number of authors on the front page.
%% Note that in AASTeX v6.2 a \collaboration call (see below) counts as
%% an author in this case.
%
%\AuthorCollaborationLimit=3
%
%% Will only show Schwarz, Muench and "the AAS Journals Data Scientist 
%% collaboration" on the front page of this example manuscript.
%%
%% Note that all of the author will be shown in the published article.
%% This feature is meant to be used prior to acceptance to make the
%% front end of a long author article more manageable. Please do not use
%% this functionality for manuscripts with less than 20 authors. Conversely,
%% please do use this when the number of authors exceeds 40.
%%
%% Use \allauthors at the manuscript end to show the full author list.
%% This command should only be used with \AuthorCollaborationLimit is used.

%% The following command can be used to set the latex table counters.  It
%% is needed in this document because it uses a mix of latex tabular and
%% AASTeX deluxetables.  In general it should not be needed.
%\setcounter{table}{1}

%%%%%%%%%%%%%%%%%%%%%%%%%%%%%%%%%%%%%%%%%%%%%%%%%%%%%%%%%%%%%%%%%%%%%%%%%%%%%%%%
%%
%% The following section outlines numerous optional output that
%% can be displayed in the front matter or as running meta-data.
%%
%% If you wish, you may supply running head information, although
%% this information may be modified by the editorial offices.
\shorttitle{Improved DRW for S82 QSO}
\shortauthors{Suberlak et al.}
%%
%% You can add a light gray and diagonal water-mark to the first page 
%% with this command:
% \watermark{text}
%% where "text", e.g. DRAFT, is the text to appear.  If the text is 
%% long you can control the water-mark size with:
%  \setwatermarkfontsize{dimension}
%% where dimension is any recognized LaTeX dimension, e.g. pt, in, etc.
%%
%%%%%%%%%%%%%%%%%%%%%%%%%%%%%%%%%%%%%%%%%%%%%%%%%%%%%%%%%%%%%%%%%%%%%%%%%%%%%%%%

%% This is the end of the preamble.  Indicate the beginning of the
%% manuscript itself with \begin{document}.

% from Foreman2017  preamble : 
\newcommand{\project}[1]{\textsf{#1}}


\begin{document}

\title{Improving Damped Random Walk parameters for SDSS Stripe 82 Quasars with Pan-STARRS1}


%% LaTeX will automatically break titles if they run longer than
%% one line. However, you may use \\ to force a line break if
%% you desire. In v6.2 you can include a footnote in the title.

%% A significant change from earlier AASTEX versions is in the structure for 
%% calling author and affilations. The change was necessary to implement 
%% autoindexing of affilations which prior was a manual process that could 
%% easily be tedious in large author manuscripts.
%%
%% The \author command is the same as before except it now takes an optional
%% arguement which is the 16 digit ORCID. The syntax is:
%% \author[xxxx-xxxx-xxxx-xxxx]{Author Name}
%%
%% This will hyperlink the author name to the author's ORCID page. Note that
%% during compilation, LaTeX will do some limited checking of the format of
%% the ID to make sure it is valid.
%%
%% Use \affiliation for affiliation information. The old \affil is now aliased
%% to \affiliation. AASTeX v6.2 will automatically index these in the header.
%% When a duplicate is found its index will be the same as its previous entry.
%%
%% Note that \altaffilmark and \altaffiltext have been removed and thus 
%% can not be used to document secondary affiliations. If they are used latex
%% will issue a specific error message and quit. Please use multiple 
%% \affiliation calls for to document more than one affiliation.
%%
%% The new \altaffiliation can be used to indicate some secondary information
%% such as fellowships. This command produces a non-numeric footnote that is
%% set away from the numeric \affiliation footnotes.  NOTE that if an
%% \altaffiliation command is used it must come BEFORE the \affiliation call,
%% right after the \author command, in order to place the footnotes in
%% the proper location.
%%
%% Use \email to set provide email addresses. Each \email will appear on its
%% own line so you can put multiple email address in one \email call. A new
%% \correspondingauthor command is available in V6.2 to identify the
%% corresponding author of the manuscript. It is the author's responsibility
%% to make sure this name is also in the author list.
%%
%% While authors can be grouped inside the same \author and \affiliation
%% commands it is better to have a single author for each. This allows for
%% one to exploit all the new benefits and should make book-keeping easier.
%%
%% If done correctly the peer review system will be able to
%% automatically put the author and affiliation information from the manuscript
%% and save the corresponding author the trouble of entering it by hand.



\correspondingauthor{Krzysztof Suberlak}
\email{suberlak@uw.edu}

\author[0000-0002-9589-1306]{Krzysztof L. Suberlak}
\affiliation{Department of Astronomy \\ University of Washington \\ Seattle, WA 98195, USA}


\author{\v{Z}eljko Ivezi\'c}
\affiliation{Department of Astronomy \\ University of Washington \\ Seattle, WA 98195, USA}



\author{Chelsea MacLeod}
\affiliation{Harvard Smithsonian Center for Astrophysics \\ 60 Garden St, Cambridge, MA 02138, USA}




% Abstract of the paper
\begin{abstract}

We use the Panoramic Survey Telescope and Rapid Response System 1 Survey (Pan-STARRS1, PS1) data to extend  the Sloan Digital Sky Survey (SDSS) Stripe 82 quasar light curves. Combining PS1 and SDSS light curves provides 15 years baseline for 9248 quasars - 5 years longer than prior studies that used only SDSS. We fit the light curves with Damped Random Walk (DRW) model, and correlate the DRW model parameters -  asymptotic variability amplitude SF$_{\infty}$, and characteristic timescale $\tau$, with quasar physical properties - black hole mass, bolometric luminosity, and redshift. Using simulated light curves, we find that longer baseline allows us to better constrain DRW parameters. After adding PS1 data, the characteristic timescale $\tau$ is a stronger function of quasar luminosity, and has a weaker dependence on the black hole mass,
than with SDSS data alone. In addition, the variability amplitude dependence on the quasar luminosity is weaker. We also make predictions for the fidelity of DRW model parameter retrieval when light curves will be further extended with ZTF and LSST data. Finally, we show how updated DRW parameters lend an independent method of discovering Changing-Look Quasar Candidates (CLQSO).
\end{abstract}


%%%%%%%%%%%%%%%%% BODY %%%%%%%%%%%%%%%%%%%%%%%%%%%%%%%%%

\section{Introduction}

Quasars are variable at rest-frame optical wavelengths at the asymptotic root-mean-square (rms) level of about 0.2 mag. These distant galaxies harbor an actively accreting supermassive black hole - an active galactic nucleus. Although it is agreed upon that the majority of optical light originates from the thermal emission of the accretion disk, the detailed origin of variability has been debated for the past 50 years (see \citealt{sun2018} and references therein). Some favor a thermal origin of variability \citep{kelly2013}, related to the propagation of inhomogeneities (`hot spots') in the disk \citep{dexter2011, cai2016}, others suggested magnetically elevated disks \citep{dexter2019}, or X-ray reprocessing  \citep{kubota2018}.  Indeed, it may well be that the answer involves combination of these -  as \cite{sanchez2018} suggests, perhaps short-term variability (hours-days) is linked to the changes in X-ray flux, while long-term variability (months-years) is more intrinsic to the disk \citep{edelson2015,lira2015}. Nevertheless, quasar optical light curves have been successfully described using the Damped Random Walk (DRW) model \citep{kelly2009, macleod2010, kozlowski2010, zu2011, kasliwal2015a}, and the DRW parameters have been linked to the physical quasar properties (\citealt{macleod2010}, hereafter M10). 

Variability is also a classification tool, allowing to distinguish quasars from other variable sources that do not exhibit a stochastic variability pattern \citep{macleod2011}. This property is especially useful for selecting quasars in the intermediate redshift range, which overlaps the stellar locus in color-color diagrams \citep{sesar2007, yang2017}). Variability has also been used to increase the completeness in measurements of Quasar Luminosity Function (see \citealt{ ross2013, palanque2013, alsayyad2016, mcgreer2013, mcgreer2018}). 

Power spectral density (PSD) informs us about the distribution of variability across frequency range: from short timescales (high frequencies) to long timescales (low frequencies). Quasar, or more broadly speaking, AGN variability, exhibits a broken power law PSD, of the form 
$\log{P(f)} \propto \alpha \log{(f)}$, with $\alpha_l$ at low frequencies and $\alpha_h$ at high frequencies. For a pure DRW process,  $\alpha_{h}{=}-2$ and $\alpha_{l}{=}0$, so that:

\begin{equation}
P(f) = \frac{4\sigma^{2}\tau}{1+(2 \pi \tau f)^{2}}
\end{equation}
(with $\sigma = \mathrm{SF}_{\infty} / \sqrt{2}$, $\tau$ the characteristic timescale, $f$ the frequency), where $P(f) \propto f^{-2}$  at high frequencies $f > (2\pi \tau)^{-1}$, and levels to a constant value at lower frequencies ~\citep{kelly2014}. 

There is a debate in the literature about the exact shape of the quasar PSD, and of any possible  departures from the pure DRW model. Studies using quasar data from wide-field photometric surveys (OGLE, SDSS, PS1) benefit from relatively long baselines (several years), which constrain the low frequency part of the PSD. Overall, there is no evidence of a significant departure from DRW at these long timescales, i.e. $\alpha_{l} \approx 0$ \citep{zu2013, simm2016, kozlowski2016b, caplar2017, guo2017, sun2018}. However, these ground-based surveys suffer from a sparse sampling, which can be remedied by using a space-based telescope that can carry out near-continuous observations, like the Kepler mission ~\citep{borucki2010}. Studies using Kepler data, that focused on a smaller number of well-sampled AGN with short baselines (<hundred days), found a range of power-law slopes at high frequencies - from -1 to -3.2, which includes the DRW $\alpha_{h} \approx -2$, but further study is needed \citep{mushotzky2011,edelson2014,aranzana2018,smith2018}. However, in this paper the timescales probed are larger than several days, and thus we can assume that DRW is the best working description of quasar variability for available optical light curves. Furthermore, in this work we directly compare the results of SDSS light curves extended with PS1 to M10 who used pure DRW description (see discussion therein on a possible departure from DRW). Therefore, to allow a better comparison of our results with M10, we  use the DRW description of quasar PSD. 

Due to its stochastic nature, for an unbiased parameter retrieval of DRW process the light curve is requird to be several times longer than the characteristic timescale (\citealt{kozlowski2010, kozlowski2017a}, hereafter K17). This means that DRW parameters recovered for short light curves (compared to the recovered timescale) may be biased, which in turn affects the correlations with physical parameters (black hole mass, Eddington ratio, absolute luminosity). 

For this reason, while some studies have restricted the probed redshift range, limiting the quasar sample to where one would expect only shorter timescales based on previous studies \citep{sun2018, guo2017, kelly2013,simm2016}, some have elected not to study timescales at all \citep{sun2018,sanchez2018}, or use the timescales recovered from short light curves primarily for classification \citep{hernitschek2016}.

By extending available quasar light curves, we are able to better recover DRW timescales. Since almost a decade ago, when M10 published their study based on SDSS Stripe 82 data, new datasets (PS1,PTF,CRTS) have become available. They can extend the quasar light curves by almost 50\%.  Indeed, \citet{li2018} combined SDSS and Dark Energy Camera Legacy Survey (DECaLS) data, to provide a 15 year baseline, but by using the entire SDSS footprint suffered from poor sampling and had to use the ensemble structure function approach. %. However, by focusing on a large area to encompass as many quasars as possible (119,305 up to z=4.89) suffered from poor sampling which lends itself better to an ensemble structure function approach rather than direct light curve modeling. 

% Kepler :  not relevant in this context 
%On the short timescales there are Kepler-based studies, with extremely well-sampled light curves (30 min cadence) of a small number of quasars, with baselines of up to 4 years \citep{mushotzky2011, edelson2014, kasliwal2015a, kasliwal2017, smith2018, aranzana2018}. 

 Unlike previous studies, in this work, by combining SDSS and PS1 data for the well-observed Stripe 82, we afford both an extended baseline (15 years), a large number (9000) of quasars, and  a good cadence (N > 60 epochs) to which we fit the DRW model. The layout of this paper is as follows: first we confirm in Section~\ref{sec:methods} that extending the quasar baseline is an important improvement in providing unbiased estimates of the DRW model parameters (K17);  in Section~\ref{sec:data} we describe the datasets employed, and their combination onto a common photometric system;  in Section~\ref{sec:simulation} we simulate the improvement in the recovery of DRW parameters with baseline extension and realistic cadence; finally in Section~\ref{sec:results} we describe the main results analyzing correlations between physical parameters and variability, in Section~\ref{sec:discussion} we discuss the physical meaning of relevant timescales, and in Section~\ref{sec:conclusions} we summarize the main conclusions. In this work we adopt a $\Lambda $CDM cosmology with $h_{0} = 0.7$ and $\Omega_{m} = 0.3 $. 
%
%
%
%
\section{Methods}
\label{sec:methods}
\subsection{DRW as a Gaussian Process}
Damped Random Walk (Ornstein-Uhlenbeck process, \citealt{rasmussen2006}) can be modeled as a member of a class of Gaussian Processes (GP). Each GP is described by a mean and a kernel - a covariance function that contains a measure of correlation between two points: $x_{n}$ and $x_{m}$, separated by $\Delta t_{nm}$ (autocorrelation). For the  DRW process, the covariance  between two observations spaced by  $\Delta t_{nm}$ is: 

\begin{eqnarray}
k(\Delta t_{nm}) &=& \sigma^{2}\exp{(-\Delta t_{nm} / \tau)}  \nonumber \\
                 &=& \sigma^{2} \mathrm{ACF}(\Delta t_{nm})\nonumber
\end{eqnarray} 

Here $\sigma^{2}$ is an amplitude of correlation decay as a function of $\Delta t_{nm}$,  while $\tau$ is the characteristic timescale over which correlation drops by $1/e$. For a DRW,  the correlation function $k(\Delta t_{nm})$ is also related to the autocorrelation function ACF. 

Not explicitly used in this paper, but of direct relevance to the DRW modeling, is the structure function (SF). SF can be found from the data as the root-mean-square scatter of  magnitude differences $\Delta m$  calculated as a function of temporal separation $\Delta t$ (we drop subscripts $n$,$m$ for brevity). SF is directly related to a DRW kernel $k(\Delta t)$:

\begin{equation}
\mathrm{SF}(\Delta t) = \mathrm{SF}_{\infty} (1-\exp{(-|\Delta t|/\tau)})^{1/2}
\end{equation}

For quasars SF follows approximately a power law: $\mathrm{SF} \propto \Delta t^{\beta}$,  and for large time separation $\Delta t$, as epochs in the light curve cease to be correlated, it levels out to a constant value $\mathrm{SF}_{\infty}$ - the asymptotic SF.  Note that above $\mathrm{SF}_\infty = \sqrt{2} \sigma$ (also see \citet{macleod2012, bauer2009, graham2015a} for an overview).


\subsection{Fitting}
We evaluate the likelihood of the DRW model with a particular set of $\tau,\sigma$ given the data with \project{celerite}  - a fast GP solver \citep{foreman2017}. The underlying matrix algebra is similar to that used by \cite{rybicki1992}, \cite{kozlowski2010}, and M10. Also, as in previous work, we use a prior on DRW parameters that is  uniform  in log space :  $1 / (\sigma \tau)$. The main difference in our approach is that rather than adopting the Maximum A-Posteriori (MAP) as the `best-fit' value for the DRW parameters (as in \citealt{kozlowski2010}, K17, \citealt{kozlowski2016b}, M10, \citealt{macleod2011}),  we find the expectation value of the marginalized posterior. This is advantageous because of a non-Gaussian shape of the posterior - otherwise, if the posterior was a 2D normal distribution, the expectation value would coincide with the maximum of the posterior (MAP solution). 

\subsection{The impact of light curve baseline}\label{sec:baseline}

K17 reports that one cannot trust any results of DRW fitting unless the light curve length is at least ten times longer than the characteristic timescale. In this section we revisit relationship between recovered and input timescales as a function of light curve baseline by following K17 setup. We confirm that the bias in retrieved DRW timescale depends on how many times the light curve is longer than the timescale. However, we find that the baseline does not have to be as many as ten times longer to provide meaningful, rather than unconstrained, results. Assuming fixed baseline of $\Delta T = 8$ years, we simulate 10 000 light curves, exploring 100 values of input timescales, but identical $\mathrm{SF}_{\infty}=0.2$ mag, with either SDSS ($N=60$), or OGLE-like ($N=445$) cadence. Defining $\rho$ as the ratio of input timescale to baseline, we probe a range of $\rho$ between 0.01 and 15, uniform in a logarithmic grid.



\begin{figure*}  % code/PLOT_Fig2_George_Celerite.ipynb
\plottwo{figs/celerite_SDSS_Jeff1_expectation-190208_results_celerite_R.png}{figs/celerite_OGLE_Jeff1_expectation-190208_results_celerite_R.png}
\caption{Recovery of the inpyt DRW timescale, with baseline fixed to $\Delta T = 8$ years. We explore 100 logarithmically-spaced values of $\rho \equiv \tau / \Delta T$, simulating 100 realizations of DRW process at each $\rho$. The impact of photometric uncertainties and cadence is small in this case: the left panel (SDSS, N=60 epochs) does not significantly differ from the right panel (OGLE, N=445 epochs). The dotted horizontal and solid vertical lines mark $\rho = 0.1$, i.e. the baseline being ten times longer than the timescale. The solid diagonal line corresponds to a perfect recovery of DRW parameters (where $\rho_{in} =\rho_{out} $). For any quasar, extending its light curve  moves it towards the upper-right (well-constrained) portion of the diagram, since for a fixed $\tau_{in}$, increasing $\Delta T$ decreases $\rho$.} 
\label{fig:rho_space}
\end{figure*}


For each light curve we simulate the underlying DRW signal $s(t)$ by iterating over the array of time steps $t$.  At each step, we draw a point from a Gaussian distribution, for which the mean and standard deviation are re-calculated at each timestep (see eqs. A4 and A5 in \citealt{kelly2009}, Sec. 2.2 in M10, and K17). Initially, at $t_{0}$ the signal is equal to the mean magnitude, $s_{0} = \langle m \rangle$. After a timestep $\Delta t_{i} = t_{i+1} - t_{i}$, the signal $s_{i+1}$ is drawn from  a normal distribution $\mathcal{N}(loc, stdev)$, with : 

\begin{equation}
loc = s_{i} e ^ { - r  }  + \langle m \rangle \left( 1 - e ^{ - r }\right)
\end{equation}

and 

\begin{equation}
stdev^{2} =  0.5  \, \mathrm{SF}_{\infty}^{2} \left( 1 - e ^{  - 2 r  }  \right)
\end{equation}

where  $r = \Delta t_{i} / \tau$, and $\tau$ is the damping timescale.


Like K17, we add to the true underlying signal with zero mean $s(t)$, a mean magnitude  ($\mathrm{r_{SDSS}}=17$ mag and $\mathrm{I_{OGLE}}=18$ mag), and calculate magnitude-dependent estimate of photometric uncertainty:

\begin{eqnarray}
\sigma_{\mathrm{SDSS}}^{2} &=& 0.013^{2} + \exp{[2 (\mathrm{r_{SDSS}}-23.36)]} \\
\sigma_{\mathrm{OGLE}}^{2} &=& 0.004^{2} + \exp{[1.63 (\mathrm{I_{OGLE}} - 22.55)]}
\end{eqnarray}

To simulate observational conditions  we add a Gaussian noise $n(t) = \mathcal{N}(0,\sigma(t))$:

\begin{equation}
y(t) = s(t) + n(t) 
\end{equation}

The resulting distribution of fitted timescales  as a function of input timescales, scaled by the  8 year baseline: $\rho_{out}$ vs $\rho_{in}$, is shown in Fig.~\ref{fig:rho_space}. We confirm the findings of K17: for short light curves, the best-fit $\tau$ becomes $\sim 1/5$ of light curve length (where $\log_{10}{(\rho_{out})} \approx -0.7$, the `unconstrained' region, lower-left of each panel). However, as long as the light curve is several times longer than the timescale ($1/\rho \gtrapprox 3, i.e. \log_{10}{(\rho)} \lessapprox 0.5$), we can recover the timescale without substantial bias (the dashed line approaches the solid diagonal line on both panels). Therefore, by extending the baseline we can move from the biased region (bottom left) to the unbiased regime (top right). This is the basis for this study,  in which we extend the baselines of quasar light curves from SDSS-only (10 years) to combined SDSS-PS1 (15 years). 


%
%
%
%
\section{Data}
\label{sec:data}

We focus on the data pertaining to a 290 deg$^{2}$ region of the southern sky known as Stripe 82 (S82), repeatedly observed by the Sloan Digital Sky Survey (SDSS) between 1998 and 2008. Originally aimed at supernova discovery, objects in this area were re-observed on average 60 times (see \citealt{macleod2012}, Sec. 2.2 for overview, and \citealt{annis2014} for details). Availability of well-calibrated \citep{ivezic2007}, long-baseline light curves spurred variability research \citep{sesar2007}. The DR9 catalog \citep{schneider2008} contains 9258 spectroscopically confirmed quasars within S82. Within  $0.5 ''$  we find matching data for 9248 quasars from  PanSTARRS (PS1) DR2 \citep{chambers2011,flewelling2018}, 7737 from Catalina Real-Time Transient Survey (CRTS, \citealt{drake2009}), and  6455 from Palomar Transient Factory (PTF, \citealt{rau2009}). Fig.~\ref{fig:lc_extent} illustrates the improvement in baseline coverage when combining various surveys. The width of each rectangle corresponds to the duration of each survey (survey baseline), and the height to the area covered by each survey. The lower edge of each rectangle marks the $5\sigma$ depth in the r-band (or equivalent). LSST stands out in that it will provide the best extension of SDSS baseline and depth.  



\begin{figure*} % code2/Stats_simulation.ipynb 
\plotone{figs/lightcurves_extent.png}
\caption{An illustration of survey baseline, sky area covered, and depth. The width of each rectangle corresponds to the extent of real or simulated light curves for Stripe 82 quasars for each survey. This includes SDSS DR7, CRTS DR2, PS1 DR2, PTF DR2, ZTF DR1, and for LSST the full 10-year survey. The lower edge of each rectangle (marked by a solid thick line) corresponds to the $5\sigma$ limiting magnitude (SDSS $r$, PS1 $r$, PTF $R$, ZTF $r$, LSST $r$, CRTS $V$). The vertical extent of each rectangle corresponds to the total survey area (for SDSS, up to DR15).  Note how PS1 and PTF extend the baseline of SDSS by approximately $50\%$, and how inclusion of LSST roughly triples the SDSS baseline. For reference, the area covered by LSST is $25 000$ sq.deg., corresponding to  $60\%$ of the whole sky (which has the area of $4\pi$ steradians, i.e. 41253 sq.deg.).}
\label{fig:lc_extent}
\end{figure*} 


Combining data from different photometric standards requires applying color transformation, or photometric offsets. We first seek to combine PS1 $gri$,  PTF $gR$, and CRTS $V$, into a common SDSS $r$-band (best photometry). To this end we calculate  color terms using the SDSS standard stars catalog~\citep{ivezic2007}. Focusing on a 100 000 randomly chosen stars, we find their CRTS, PS1, and PTF matches~\footnote{CRTS from B.Sesar, priv.comm., PS1 from MAST (\url{http://panstarrs.stsci.edu}), and PTF from IRSA PTF Object Catalog (\url{https://irsa.ipac.caltech.edu/})}. 

The difference  between the target (SDSS) and source (eg.PS1) photometry can be written as a function of the mean SDSS $g-i$ color: 

\begin{equation}
m_{\mathrm{PS1}} - m_{\mathrm{SDSS}} = f(g-i)
\end{equation}

Some authors (eg. \citealt{li2018}) allow the transformation to be a higher-order polynomial, but as Fig.~\ref{fig:quasarColors} shows, quasars occupy a relatively narrow region of $g-i$ color space, and we find that the linear fit is sufficient. The derived linear coefficients for photometric transformations between SDSS $r$ and PS1 $gri$, PTF $gR$, CRTS $V$,  as a function of SDSS $g-i$ color, are listed in Table~\ref{tab:offsets}. We illustrate the process showing in Fig.~\ref{fig:offsetsPS1} the SDSS-PS1 standard stars data used to calculate the offsets. Note that the PS1 $r$ (middle panel) is very close to the SDSS $r$ - within  $0.01$ mag (1\% level) across the $g-i$ color range. We focus on SDSS $r$ - PS1 $r$ offset as a function of magnitude in Fig.~\ref{fig:offsetPS1mag} - the near-equivalence of bandpass coverage is valid at 1\% level up to $r < 20.5$. %Because of that, we also combine PS1 r-band and SDSS r-band without applying any offsets. 

In selecting the most beneficial datasets to complement SDSS $r$ we also consider the associated photometric uncertainties (aka `errors'). As shown in Fig.~\ref{fig:lc_extent},  PTF and CRTS are shallower than SDSS or PS1. Therefore for faint objects, like quasars (for S82 sample the population median is SDSS $r \sim 20$ mag) PTF and CRTS have larger photometric uncertainties than SDSS or PS1. Indeed, as Fig.~\ref{fig:lc_errors} shows, the distribution of median errors for PTF, CRTS, and ZTF quasar data is wider than the corresponding SDSS and PS1 data. As simulations show (Sec.~\ref{sec:simulation}), although PTF and CRTS data do extend the SDSS baseline, their error properties decreases their utility in complementing the SDSS dataset. After all, the SDSS baseline extension afforded with PTF and CRTS is comparable to that achieved with PS1 data alone (Fig.~\ref{fig:lc_extent}). 

Furthermore, to mitigate problems that could arise when applying photometric transformations (such as spurious variability due to incorrect offsets, or color-dependent variability), we choose to combine SDSS $r$ with only PS1 $r$, since as Figs.~\ref{fig:offsetsPS1},~\ref{fig:offsetPS1mag} show, SDSS $r$ and PS1 $r$ are sufficiently similar (at 1\% level up to 20.5 mag) that no photometric transformation is required. 

Finally, we clean the combined SDSS $r$ - PS1 $r$ quasar light curves using standard procedures of  $\sigma$-clipping in magnitude and error space, and error-weighted day-averaging, to mitigate the impact of bad photometry, and average out the intra-night variability (as in \citealt{charisi2016,suberlak2017}). Of 9248 SDSS-PS1 quasars, 8516 have the PS1 $r$ data with 662 092 epochs. Of these,  $5\sigma$ clipping removes  14 156 epochs, and day-averaging brings that number down by 67 615 epochs (of multiple observations per night that were averaged). To avoid unphysically small errors,  we add in quadrature $0.01$ mag if combined nightly error is $<0.02$ mag. In the final sample there are 580 321 epochs.


 

% QSO COLORS 
\begin{figure*} % code/AA_quasar_colors.ipynb 
\plotone{figs/SDSS_S82_CMD_qso_stars_2.png}
\caption{Regions of color-color (upper left, upper right, bottom left), and color-magnitude (bottom right)  space occupied by SDSS S82 quasars (color) and stars (contours). We use quasar median photometry from \citet{schneider2010}, and standard stars catalog of \citet{ivezic2007}, showing a random subset of 10 000 stars. As seen in the bottom-left panel, quasars occupy a particular range of SDSS $g-i$ color. Therefore in fitting the linear color transformations we limit the color range to  $-0.35<(g-i)<0.75$ (vertical dashed lines in Fig.\ref{fig:offsetsPS1}). Quasars also overlap other variable sources (eg. RR Lyrae), not shown here \citep{sesar2007}. }
\label{fig:quasarColors}
\end{figure*} 



% PS1 
\begin{figure*}% code/AC_SDSS_PS1_offsets.ipynb 
\epsscale{1.2}
\plotone{figs/Offsets_PS1-SDSSr_SDSSgi_ext-NO.png}
\caption{The SDSS-PS1 offsets, derived with the SDSS standard stars~\citep{ivezic2007}. From randomly chosen subset of 100 000 SDSS stars, 95 000 have PS1 DR2 data. To minimize scatter due to larger errors  we select a subset of 40 000 stars with  $r < 19$ mag. Vertical dashed lines mark the region in the SDSS color space occupied by quasars (see Fig.~\ref{fig:quasarColors}), used to fit the stellar locus with a first order polynomial, marked by the solid red line. The best-fit slopes: 0.619, -0.04, -0.283 for PS1 $g$, $r$, $i$, respectively, are listed as $B_{1}$ in Table~\ref{tab:offsets}.}
\label{fig:offsetsPS1}
\end{figure*} 



\begin{deluxetable}{c|cc}
%\tablewidth{\textwidth}
% solution from https://tex.stackexchange.com/questions/370268/aastex61-single-column-deluxetable-cannot-set-width 
\tablecaption{Color terms (photometric offsets) between CRTS, PTF, PS1 passbands and  SDSS, using the SDSS mean $g-i$ color to spread the stellar locus. Thus the SDSS $r$ synthetic magnitude, $r_{s}$, can be found as $r_{s} = x-B_{0}-B_{1}(g-i)$. This linear trend is illustrated in Fig.\ref{fig:offsetsPS1}, where we plot $(x-r_{\mathrm{SDSS}})$ as a function of $(g-i)_{\mathrm{SDSS}}$ for $x=g_{\mathrm{P1}}, r_{\mathrm{P1}},i_{\mathrm{P1}}$.\label{tab:offsets}}  

\tablehead{
\colhead{Band (x)} & 
\colhead{\hspace{1.15cm}$B_{0}$}\hspace{1.05cm} & 
\colhead{\hspace{1.15cm}$B_{1}$}\hspace{1.05cm}
}
\startdata
CRTS V & -0.0464  & -0.0128 \\
PTF g &  -0.0294  &  0.6404 \\
PTF R &  0.0058   & -0.1019 \\
PS1 g &  0.0174   &  0.6194 \\
PS1 r &  0.0065   & -0.0044 \\
PS1 i &  0.0260   & -0.2830 \\
\enddata

\tablecomments{To derive the color terms we used a subset of 100 000 stars randomly chosen from the SDSS standard stars catalog \citep{ivezic2007}. To minimize scatter we selected bright stars with $r<19$ mag.}
\end{deluxetable}



% removed as duplicating the SDSS-PS1 offsets middle panel ... 
% \begin{figure}%[ht!] % ../code2/SDSS_PS1_DR2_stellar_offsets.ipynb
% \plotone{figs/PS1_DR2_detections_SDSS_Stripe82Calib_g-i.png}
% \caption{The offset $\Delta m_{r}$  between PS1 r-band and SDSS r-band plotted as a function of (g-i) SDSS color. $\Delta m_{r}$ is stable across a range of (g-i) color occupied by quasars, and illustrates that the we can justify using no photometric transformation between SDSS r-band and PS1 r-band. }
% \label{fig:SDSS_PS1_offset_gi}
% \end{figure} 


\begin{figure}%[ht!] % ../code2/SDSS_PS1_DR2_stellar_offsets.ipynb
\epsscale{1.2}
\plotone{figs/PS1_DR2_detections_SDSS_Stripe82Calib.png}
\caption{PS1 $r$ vs SDSS $r$ as a function of SDSS $r$ for 100 000 randomly selected standard stars from \citet{ivezic2007} catalog. Almost 95\% of SDSS stars have PS1 DR2 photometry. The filled magenta circles represent the median offset - a slight slope at 1\% (0.01 mag) level up to $r<20.5$ mag.}
\label{fig:offsetPS1mag}
\end{figure} 




\begin{figure}%[ht!]  %code2/Subsample_master_LC.ipynb 
\epsscale{1.2}
\plotone{figs/sim_subsampled2_stats_error_linear.png}
\caption{Distribution of median photometric uncertainties (`errors') in r-band real light curves. The PTF and ZTF segments have much larger errors than SDSS and PS1 due to shallower depth data. CRTS errors (not shown) are on average 50\% larger than PTF.}
\label{fig:lc_errors}
\end{figure} 

%
%
%
%
%

\section{Simulations : lessons learned}\label{sec:simulation}

We simulate the theoretical improvement of the DRW parameter retrieval in extended light curves. We generate long and well-sampled  `master' light curves, all with input $\tau = 575 $ days, $SF_{\infty} = 0.2$ mag (the median of S82 quasar distribution in M10), with 0 mean.  We subsample at real observed epochs for SDSS and PS1, and at predicted cadences for ZTF and LSST  (see Fig.~\ref{fig:lc_simulated}). To each simulated light curve we add a magnitude offset corresponding to the mean of the combined SDSS-PS1 light curve. That way the magnitude distribution of simulated light curves is similar to that of the observed SDSS-PS1 data. For the LSST 10-year segment (finishing in 2031) we assumed 50 randomly distributed  epochs per year, with the following error model:

\begin{eqnarray}
\label{eq:errorModel}
\sigma_{LSST}(m)^{2} &=& \sigma_{sys}^{2} + \sigma_{rand}^{2} \,\, \mathrm{(mag)}^{2} \\
\sigma_{rand}^{2} &=& (0.04-\gamma)x + \gamma x^{2} \nonumber \\
x &=& 10^{0.4(m-m_{5})} \nonumber
\end{eqnarray}

with  $\sigma_{sys} = 0.005$, $\gamma=0.039$, $m_{5} = 24.7$ (see \citealt{ivezic2019}, Sec. 3.2).
For the ZTF 1-year segment (Spring 2019 ZTF DR1 including the data from 2018) we assumed 120 observations (every three nights) in $g_{\mathrm{ZTF}}$ and $r_{\mathrm{ZTF}}$, deriving the magnitude-dependent error model by plotting best mag rms as a function of best median magnitude for ZTF matches to S82 standard stars in Fig.~\ref{fig:ztf_errors}. We find that the LSST error model (Eq.~\ref{eq:errorModel}) with $\gamma = 0.05$, $\sigma_{sys} = 0.005 $, and $m_{5} = 20.8$ adequately describes the ZTF photometric uncertainty. 


\begin{figure}%[ht!]  % code2/Simulate_ZTF_cadence_error.ipynb
\plotone{figs/ZTF_error_curve.png}
\caption{The best mag rms plotted as a function of magnitude for ZTF non-variable stars with over 100 observations. We overplot the adopted error model, with $\gamma = 0.05$, $\sigma_{sys} = 0.005 $, and $m_{5} = 20.8$ (see Eq.~\ref{eq:errorModel}). Properties of ZTF photometric uncertainties are largely similar to the PTF uncertainties.}
\label{fig:ztf_errors}
\end{figure} 



\begin{figure*}%[ht!] % code2/Subsample_master_LC.ipynb
\plotone{figs/Simulated_DRW-0008_sampled-3537034_fit.png}
\caption{Simulated well-sampled underlying DRW process - one of `master' light curves ($\tau=575$d, SF$_{\infty} = 0.2$ mag, 4 points per day) shown with  small black dots. To simulate observations, the cadence is degraded (subsampled) to match the ground-based cadence corresponding to real quasar data from SDSS (red), PS1 (green) segments, and simulated LSST (blue) epochs (here we use SDSS-PS1 epochs for quasar dbID=3537034). The orange `error snake' is an envelope marking the standard deviation of the fit to the data using a Gaussian process  with DRW kernel (Sec.~\ref{sec:simulation}).}
\label{fig:lc_simulated}
\end{figure*} 



To mirror observational conditions we add to the true underlying DRW signal a Gaussian noise, with variance defined by photometric uncertainties for corresponding surveys. Fig.~\ref{fig:lc_simulated} illustrates the simulated `master' light curve (black dots, 4 per day), subsampled at SDSS (red), PS1 (green), and LSST (blue) cadence. While PS1 provides a 50\% improvement of the SDSS baseline, LSST will nearly triple it. Fig.~\ref{fig:lc_simulated_results} shows how the simulated distribution of DRW parameters $\sigma$, $\tau$, changes as the SDSS quasar light curves are extended with PS1, ZTF, and LSST data. In the future (after more data has been assembled and re-calibrated), ZTF will help, but not as dramatically as LSST. Note that ZTF, due to larger errors (Fig.\ref{fig:lc_errors}), causes a widening of the recovered $\tau$ distribution. Using PS1 data with its excellent  deep photometry (as compared to ZTF or PTF) is the best improvement over existing SDSS results.  For this reason we use only SDSS-PS1 portion of quasar light curves, as the best tradeoff between adding more baseline vs introducing more uncertainty with noisy data.


\begin{figure*}  % code2/Explore_simulation_results.ipynb 
\plotone{figs/Simulated_Jeff1-EXP-190401.png}
\caption{The ratio of DRW parameters fitted with \project{celerite}: $\tau$ and $\sigma$, to the input $\tau_{in} = 575 $d, $\sigma_{in} = 0.2 / \sqrt{2} {\sim} 0.14$  ($SF_{\infty}=0.2$ mag). We simulated 9258 `master' light curves, and  subsampled at real SDSS r-band or PS1 r-band cadence and photometric uncertainties, and simulated ZTF and LSST cadence. To simulate observing conditions, the underlying DRW signal was convolved with a Gaussian noise corresponding to epochal errors. 
For each light curve we start with SDSS segment only, and as we add more segments (PS1, ZTF, LSST), we refit for DRW model parameters with \project{celerite}. Thus
each distribution corresponds to a different segment of simulated  combined SDSS-PS1-ZTF-LSST light curves. Extending the baseline shifts the distribution of recovered DRW parameters towards unbiased regime - vertical dashed line marks input matching the output. This corresponds to the upper-right (well-constrained) portion of Fig.~\ref{fig:rho_space}.}
\label{fig:lc_simulated_results}
\end{figure*} 


%
%
%
%
%
%

\section{Results: variability parameters for S82 Quasars}\label{sec:results}

We extend Stripe 82 quasar light curves by combining the SDSS r-band data with  the PS1 $r$-band data, without any photometric offets. For each quasar we fit the SDSS and SDSS-PS1 segments  with the DRW model. This yields two sets of DRW parameters per quasar: $(\tau_{SDSS}, \sigma_{SDSS})$, and $(\tau_{SDSS-PS1},\sigma_{SDSS-PS1})$. Because variability is inherent to the quasar, for the remaining analysis we shift all fitted timescales to quasar rest frame, and implicitly assume that the DRW timescales are considered in rest frame: $\tau_{\mathrm{RF}} = \tau_{\mathrm{OBS}} / (1+z)$.


\begin{figure} % code2/Compare_Celerite_Chelsea_real_fits.ipynb 
\plotone{figs/MacLeod2010_Fig3_restframe_NEW_.png}
\caption{Comparison of distributions of the rest-frame variability timescale $\tau_{RF}$ against the  asymptotic variability amplitude $SF_{\infty}$, for M10 SDSS r-band,  and \project{celerite} fits using  SDSS or SDSS-PS1 segments of combined S82 quasar light curves. The M10 (red, solid lines) and this work, using only SDSS (dashed, blue), overlap, as we recover the same underlying distributions. }
\label{fig:tau_sf_dist}
\end{figure} 


In this section we first correct fitted $\tau$, $\sigma$ for wavelength dependence. Then we show consistency with M10 results,  and consider the trends between DRW parameters and physical quasar properties: black hole mass $M_{\mathrm{BH}}$, absolute i-band  magnitude $M_{i}$, or redshift $z$.  


\subsection{Comparison to M10}
The DRW parameters recovered with \project{celerite} are  broadly consistent with M10 - Fig.~\ref{fig:tau_sf_dist} shows the rest-frame  $\tau$, and $SF_{\infty}$ distributions for our results for the SDSS segment (blue dashed contours),  SDSS-PS1 combined light curves (green dot-dashed contours), and  M10 SDSS for r-band only (red solid contours). When using exactly the same data as M10 (SDSS), our results agree. The offset of 0.05 dex  between our and M10 results for SDSS, seen on the left panel of Fig.~\ref{fig:sigma_tau_ratios_M10}, can be attributed to data cleaning and software differences. The right panel of Fig.~\ref{fig:sigma_tau_ratios_M10} shows the same distribution in terms of $K-\hat{\sigma}$ space, orthogonal to $\tau-\sigma$, where $\hat{\sigma} = \sigma\sqrt{2 / \tau}$, and $K = \tau \sqrt{\sigma} 2^{1/4} $. 


\begin{figure*} % code2/Compare_Celerite_Chelsea_real_fits.ipynb  
%\epsscale{1.2}
\plottwo{figs/Compare_Chelsea_tau_sigma_ratios_r.png}{figs/Compare_Chelsea_K-sigma-hat_ratios.png}
\caption{Comparison of \project{celerite} fits using only the  SDSS r-band segments of S82 quasars ($\sigma_{\mathrm{SDSS}}, \tau_{\mathrm{SDSS}}$), against M10 results for SDSS r-band ($\sigma_{M10}, \tau_{M10}$), object-by-object. The small offset ($<0.05 $ dex) can be attributed to software differences. See Fig.~\ref{fig:tau_sf_dist} for a comparison of rest-frame $\tau$ and $SF_{\infty}$ distributions. This is similar to Fig.3 in M10, except we plot only the r-band SDSS results. The right-hand panel shows the comparison in an orthogonal $K-\hat{\sigma}$  space. } 
\label{fig:sigma_tau_ratios_M10}
\end{figure*} 



\begin{figure*} % both :  code2/Compare_Celerite_Chelsea_real_fits.ipynb
\epsscale{1.1}
\plottwo{figs/Compare_Celerite_tau_sigma_ratios_r.png}{figs/Simulation_Celerite_tau_sigma_ratios_r.png}
\caption{Ratios of fitted DRW parameters  ($\tau$, $\sigma$), comparing the value of parameter recovered using the combined light curve length (SDSS-PS1) to the shorter, SDSS-only light curve. The left panel shows the results for S82 quasars using real data, whereas the right panel shows the simulated quasars with realistic cadence, with  $\tau_{in}=575$ days and  $SF_{\infty}=0.2$ (right). The general trend when using the real data (despite having a range of underlying timescales and amplitudes) is similar to that when using simulated data: the diagonal scatter is along the lines of constant $\hat{\sigma}$, and there is much less scatter in the perpendicular direction of $\mathrm{K}$ (see Fig.~\ref{fig:K_sigma_ratios}. There is no major change of shape of distribution as a function of mean quasar magnitude. The red rectangle marks the outliers with $\log{(\tau_{\mathrm{SDSS-PS1}} /  \tau_{\mathrm{SDSS}})} > 1$   and   $\log{(\sigma_{\mathrm{SDSS-PS1}} / \sigma_{\mathrm{SDSS}})}   > 0.4  )$, discussed in Sec.~\ref{sec:outliers}.)
}
\label{fig:sigma_tau_ratios}
\end{figure*}

\begin{figure*} % both :  code2/Compare_Celerite_Chelsea_real_fits.ipynb
\epsscale{1.1}
\plottwo{figs/Compare_Celerite_K-sigma-hat_ratios.png}{figs/Simulation_Celerite_K-sigma-hat_ratios.png}
\caption{As Fig.~\ref{fig:sigma_tau_ratios}, but in $K-\hat{\sigma}$ space, which is orthogonal to the $\tau-\sigma$ space, since $K = \tau \sqrt{\mathrm{SF}_{\infty}} = \tau \sqrt{\sigma} 2^{1/4}$ and $\hat{\sigma} = \mathrm{SF}_{\infty} / \sqrt{\tau} = \sigma \sqrt{2/\tau}$.
}
\label{fig:K_sigma_ratios}
\end{figure*}


\subsection{Outliers: possible CLQSO candidates}
\label{sec:outliers}
Fig.~\ref{fig:sigma_tau_ratios} shows the change in recovered DRW parameters between SDSS and combined SDSS-PS1 light curves. The distribution of $f_{\sigma} \equiv \log_{10}{(\sigma_{\mathrm{SDSS-PS1}} / \sigma_{\mathrm{SDSS}})}$ and $f_{\tau} \equiv \log_{10}{(\tau_{\mathrm{SDSS-PS1}} / \tau_{\mathrm{SDSS}} )}$ for real light curves (left), matches the predicted distribution for simulated light curves (right). Studies show that about 0.1\% quasars will exhibit large variability (in excess of 0.5 mag rms - see Fig.18 in \citealt{macleod2012}). Visual inspection of light curves in the upper-right region of the left panel of Fig.~\ref{fig:sigma_tau_ratios} (marked by the red rectangle) reveals large changes in brightness, similar to those seen in Changing-Look Quasars
~\citep{macleod2019,ruan2019, sheng2019, frederick2019,trakhtenbrot2019,shen2019,stern2018,ross2018,lawrence2018,yang2018,gezari2017, stern2017, sheng2017, blanchard2017, ruan2016, runnoe2016, guo2016, lamassa2015,schawinski2015, elitzur2014}. The light curves and properties of 38 CLQSO candidates for which $f_{\tau} > 1$, $f_{\sigma} > 0.4$, and  $\langle r \rangle < 20.5 $ are discussed in Appendix~\ref{app:clqso_cands}.

Such large differences in timescales and amplitude of variability can also be inferred directly from the light curves. Consider the difference in magnitude and scatter between the SDSS portion of the light curve (spanning approximately 10 years between 1998 and 2008), and the PS1 portion (spanning $\sim$5 years between, 2009-2014 - see Fig.~\ref{fig:lc_extent}). We measure the median magnitudes offset as $\Delta(\mathrm{median}) =  \mathrm{median}(SDSS) - \mathrm{median}(PS1)$, and the scatter difference as $\Delta(\sigma_{G}) = \sigma_{G}(SDSS)-\sigma_{G}(PS1)$. The resulting distributions of  $\Delta(\mathrm{median}) $ and $\Delta(\sigma_{G})$ for S82 quasars are shown in Fig.~\ref{fig:median_offsets}. Indeed, when plotting $\Delta(\mathrm{median})$ as a function of $f_{\tau}$ and $f_{\sigma}$ there is a gradient indicating that the  CLQSO candidates - outliers in $(f_{\tau}, f_{\sigma})$ space, are also outliers in $\Delta(\mathrm{median})$- $\Delta(\sigma_{G})$ space. Thus the by-product of extending light curves to recalculate the DRW parameters with increased fidelity is an independent method to discover the CLQSO. 


\begin{figure} % ../code2/Explore_SDSS_PS1_lightcurves.ipynb
\epsscale{1.2}
\plotone{figs/SDSS_PS1_DR2_cleaned_offsets_cdf_NEW.png}
\caption{The differences between SDSS and PS1 segments of combined quasar r-band light curves. First, the difference between median SDSS and median PS1 portion, plotted as a histogram (upper-left panel), and cumulative distribution function (lower-left panel). Then, the difference between $\sigma_{G}$ calculated for each portion of the light curve (
$\sigma_{G}$ is a robust estimate of the standard deviation, and is related to the difference between 75th and 25th percentile : $\sigma_{G} = 0.7413(Q_{75} - Q_{25})$). The outliers in the median offset space are also outliers in the DRW parameter space (eg. objects with $\log{(\tau_{\mathrm{SDSS-PS1}} /  \tau_{\mathrm{SDSS}})} > 1$   and   $\log{(\sigma_{\mathrm{SDSS-PS1}} / \sigma_{\mathrm{SDSS}})}   > 0.4  )$  and $r > 20.5$  have    $\Delta(\mathrm{median}) > 0.1$).}
\label{fig:median_offsets}
\end{figure}




\subsection{Rest-frame Wavelength Correction}

Objects at cosmological distances are embedded in Hubble flow due to the expansion of the Universe~\citep{riess2019}. Therefore light observed from a distant quasar would have been emitted at shorter wavelength in quasar's rest-frame: $\lambda_{RF} = \lambda_{obs} / (1+z)$, where $z$ is the cosmological redshift. Quasars at different redshifts probe different regions of rest-frame spectra (see Fig.7 in \citealt{shen2018}). Thus before correlating the DRW parameters with quasar properties we correct $\sigma, \tau$ for the $\lambda_{RF}$ dependence, studied by M10 with SDSS $ugriz$ light curves. We plot in Fig.~\ref{fig:lambda_dependence} the DRW parameters: $\mathrm{SF}_{\infty}$ and $\tau$, as a function of $\lambda_{RF}$. A solid line marks the M10 best-fit power law trend:
\begin{equation}
\label{eq:lambda}
f \propto \left( \frac{\lambda_{RF}}{4000 \mbox{\AA}} \right)^{B}
\end{equation}

with  $B=-0.479$ and $0.17$ for $SF_{\infty}$ and $\tau$, respectively.

  
\begin{figure} % code2/Compare_Celerite_Chelsea_real_fits.ipynb 
\epsscale{1.2}
\plotone{figs/macleod2010_Fig13_kdeplot_Chelsea_cut_NEW_.png}
\caption{Rest-frame timescale $\tau$ (top panel), and asymptotic structure function $SF_{\infty}$ (bottom panel), as a function of rest-frame wavelength $\lambda_{RF}$. The background contours show M10 SDSS $ urz $  data, and the foreground contours  denote our results using  SDSS (red) and SDSS-PS1 (orange) segments. The red line indicates the best-fit power law to M10 data, with $B=0.17$ an $-0.479$ for $\tau_{RF}$, and $SF_{\infty}$, respectively. We take the center of each bandpass to approximate the  observed wavelength: that is, for SDSS $urz$ bandpasses,  $\lambda_{obs} = 3520$, $6250$, $9110$ $\mbox{\AA}$, respectively, and given the redshift of each quasar, find $\lambda_{RF}=\lambda_{obs} / (1+z)$.}
\label{fig:lambda_dependence}
\end{figure} 




\subsection{Trends with Black Hole Mass, Absolute Luminosity}

In the era of large synoptic surveys such as ZTF or LSST, the large increase in the number of discovered quasars means that due to limited observational resources we will afford a spectroscopic follow-up for only a few percent of AGN with optical time-series~\citep{ivezic2019}. Therefore, a relationship between quasar variability parameters ($\tau, \sigma$), and physical properties $M_{BH}$, $M_{i}$ could provide an estimate of the latter for millions of quasars. We inspect correlations between $\tau, \sigma$  and $M_{BH}$, $M_{i}$, using \cite{shen2011} catalog, based on single-epoch SDSS spectra. $M_{i}$ is K-corrected to  $z= 2$, corresponding to the peak of quasar activity~\citep{richards2006a}. For details, see Appendix~\ref{app:measureBHmass}.

On Fig.~\ref{fig:quasar_properties} we examine the distribution of $M_{BH}$, $M_{i}$, as a function of $z$ for S82 quasars. The upward gradient in the top two panels reflects the selection effect that higher redshift quasars have to be brighter to be included in the magnitude-limited sample (luminosity-redshift degeneracy: see Sec.5, Fig.12 in M10, and \citealt{dong2018}). Higher redshift quasars are also more active  and have higher black hole masses due to cosmological downsizing (see \citealt{babic2007,labita2009, mclure2004}). The distribution in the lower-left panel of Fig.~\ref{fig:quasar_properties} is peaked at $z=2$ which corresponds to the peak of quasar activity. 

\begin{figure*} % ../code2/Trends_quasar_properties.ipynb 
\plotone{figs/macleod2010_Fig12_Shen2011.png}
\caption{Distribution of quasars as a function of  redshift, observed i-band magnitude, absolute i-band magnitude (K-corrected to z=2), and virial black hole mass. All data from \cite{shen2011}.}
\label{fig:quasar_properties}
\end{figure*} 

Fig.~\ref{fig:quasar_trends} shows the DRW parameters for S82 quasars: $\tau$ and $SF_{\infty}$, plotted as a function of quasar physical properties $M_{BH}$, $M_{i}$, and $z$. Upper-left and lower-left panels in Fig.~\ref{fig:quasar_trends} contain a gradient of $SF_{\infty}$  with 
$M_{i}$,$z$ - brighter quasars have lower variability amplitude, largely independent of black hole mass. 


\begin{figure*} % ../code2/Trends_quasar_properties.ipynb  
\plotone{figs/macleod2010_Fig14_Shen2011_sdss-ps1.png}
\caption{Long-term variability $SF_{\infty}$, and characteristic timescale $\tau$ for SDSS-PS1 $r$ band data, as a function of the absolute i-band magnitude ($M_{i}(z=2)$,  a proxy for bolometric luminosity), virial black hole mass $M_{BH}$, and redshift $z$, from \cite{shen2011}. }
\label{fig:quasar_trends}
\end{figure*} 


We investigate these relations in more detail by fitting $f$ ( $\tau$ or $SF_{\infty}$) as a power-law function of $M_{BH}$, $M_{i}$, $z$ :  


\begin{eqnarray}
\label{eq:powlawmodel}
\log_{10}{f} = &A& + B \log_{10}\left( \lambda_{RF} / 4000 \mbox{\AA} \right) + C (M_{i} + 23) \nonumber \\
&+& D \log_{10}{\left( M_{BH} / 10^{9} M_{\odot}  \right)} 
\end{eqnarray} 

using a Bayesian linear regression method that incorporates measurement uncertainties in all latent variables~\citep{kelly2007b}. 

We compare the change in retrieved fit coefficients caused by adding PS1 data to SDSS against M10 SDSS-only study. Note that M10 fitted  DRW model treating each of the 5 SDSS bands as a separate light curve, resulting in over thirty thousand values of $\tau, \mathrm{SF}_{\infty}$ for nine thousand S82 quasars. Grouping fitted quasar parameters by band, they were correlated to quasar physical parameters with Eq.~\ref{eq:powlawmodel}. Fig.~\ref{fig:ugriz_drw_M10} shows the posterior samples for fitting  Eq.~\ref{eq:powlawmodel} to $f=\mathrm{SF}_{\infty}$ for quasar data separately for each SDSS bandpass.  Each band yields a slightly different fit coefficient. M10 reported as the fit result the band-mean (red vertical dashed line). Since this study uses only $r$ band data, we compare the fit coefficients to M10 SDSS $r$ data (green solid vertical line in Fig.~\ref{fig:ugriz_drw_M10}). We show the results of fitting Eq.~\ref{eq:powlawmodel} to new SDSS and SDSS-PS1 parameters in Figs.~\ref{fig:drw_tau_posterior},~\ref{fig:drw_sf_posterior}. First, with $f=\tau$ in Eq.~\ref{eq:powlawmodel} (Fig.~\ref{fig:drw_tau_posterior}), the SDSS-PS1 data confirms M10 for luminosity dependence (the posterior MCMC samples overlap), but the $M_{BH}$ dependence on $\tau$ is weaker by 0.1 dex. Second, in Fig.~\ref{fig:drw_sf_posterior} with  $f=\mathrm{SF}_{\infty}$, $\mathrm{SF}_{\infty}$ has a slightly weaker dependence on $M_{BH}$ (by 0.05 dex compared to M10). The difference between \project{celerite} results and M10 for SDSS can be attributed to data cleaning ($5\sigma$ clipping, day-averaging), that was not performed by M10 on SDSS, and software differences. Each distribution from 
Figs.~\ref{fig:drw_tau_posterior} and ~\ref{fig:drw_sf_posterior} is summarized in Table~\ref{tab:coefficients}, with the uncertainty in A,C,D fit coefficients estimated from  the standard deviation of the posterior samples. 


\begin{figure*}  % ../code2/Read_IDL.ipynb     %../code2/compare
\plotone{figs/Chelsea_ugriz_Shen2011_SF.png}
\caption{Posterior MCMC draws for fitting Eq.~\ref{eq:powlawmodel} with M10 variability amplitude $\mathrm{SF}_{\infty}$ against $M_{BH}$, $M_{i}$, $z$ \citep{shen2011}. Since M10 treated the near-simulaneous SDSS $ugriz$ data for 9258 quasars, independently for each band, this resulted in  DRW fit parameters for 7014 $u$,  7408 $g$,  6871 $r$, 6814 $i$, and 5111 $z$-band SDSS quasar light curves that fulfilled M10 quality of DRW fit selection criteria. M10 values for $\mathrm{SF}_{\infty}$ are corrected to 4000 $\mbox{\AA}$ using Eq.~\ref{eq:lambda}, with the power-law coefficient $B=-0.479$. Each distribution corresponds to a different SDSS band. We compare the results of fitting SDSS-PS1 $r$-band directly against M10 results for SDSS $r$-band (solid green). Note that Table 1 in M10 reported band-averaged values for A,C,D coeffieicnts (vertical dashed red line), while we cite in Table~\ref{tab:coefficients} the maen for $r$-band (vertical solid green line).}
\label{fig:ugriz_drw_M10}
\end{figure*} 

% https://github.com/AASJournals/AASTeX60/issues/61 
\begin{deluxetable*}{cc|CCCC}
%\tablewidth{\pagewidth}
\tablecaption{Comparison of best-fit coefficients for Eq.~\ref{eq:powlawmodel} using M10 results, and this work (S18). B is fixed to $0.17$ or $0.479$ from fitting a power law between $\lambda_{RF}$ and $\tau$, $\mathrm{SF}_{\infty}$ (see Fig.~\ref{fig:lambda_dependence}). For $f=\tau$, C is almost the same between M10 and this work for SDSS-PS1 (rows 1 and 3). However, D based on SDSS-PS1 data is smaller than M10 by 0.05 dex (row 3). For $f = SF_{\infty}$, SDSS-PS1 based C is within 0.002 dex from M10 (rows 4,6), but D based on SDSS-PS1 data is smaller than M10 by 0.02 dex. \label{tab:coefficients} } 

\tablehead{\colhead{$f$} & \colhead{Source} & \colhead{$A$(offset)} & \colhead{$B(\lambda_{RF})$} & \colhead{$C (M_{i})$} & \colhead{$D (M_{\mathrm{BH}})$} }
\startdata
$\tau$ & M10, SDSS & $2.5\pm0.027$ & $0.17\pm0.02$ & $0.03\pm0.009$ & $0.178\pm0.027$ \\
 & S19, SDSS & $2.508\pm0.019$ & $0.17\pm0.02$ & $0.04\pm0.007$ & $0.119\pm0.019$ \\
 \tableline
 & S19, SDSS-PS1 & $2.585\pm0.019$ & $0.17\pm0.02$ & $0.032\pm0.006$ & $0.128\pm0.019$ \\
  \tableline
SF$_{\infty}$ & M10, SDSS & $-0.486\pm0.012$ & $-0.479\pm0.005$ & $0.119\pm0.004$ & $0.121\pm0.012$ \\
 & S19, SDSS & $-0.548\pm0.009$ & $-0.479\pm0.005$ & $0.125\pm0.003$ & $0.101\pm0.008$ \\
  \tableline
 & S19, SDSS-PS1 & $-0.491\pm0.008$ & $-0.479\pm0.005$ & $0.117\pm0.003$ & $0.11\pm0.008$ \\
 \enddata
\end{deluxetable*}


\begin{figure*}
\plotone{figs/posterior_IDL_Chelsea_Celerite_CD_MI_Z2_tau1.png}
\caption{Distribution of MCMC posterior draws fitting Eq.~\ref{eq:powlawmodel} for characteristic timescale ($f=\tau$), based on M10 SDSS $r$-band results (solid blue line), new cleaned SDSS $r$-band results (dashed green line), and new SDSS-PS1 combined $r$-band results (dot-dashed yellow line). Of 9258 spectroscopically confirmed quasars in S82, M10 selection criteria keep 6871 (see M10, Sec 2.2). There are PS1 matches to 8516/9258,  and of these 6371 overlap with the M10 subset. Thus we keep the same subset of 6371 quasars for all fits illustrated here.  The results from SDSS-PS1 light curves are consistent with M10 for the SDSS $r$ band. }
\label{fig:drw_tau_posterior}
\end{figure*} 



\begin{figure*}
\plotone{figs/posterior_IDL_Chelsea_Celerite_CD_MI_Z2_SF1.png}
\caption{Same as Fig.~\ref{fig:drw_tau_posterior},  but fitting quasar absolute magnitude $M_{i}$, and black hole mass $M_{BH}$ in Eq.~\ref{eq:powlawmodel} as a function of the asymptotic amplitude ($f = \mathrm{SF}_{\infty}$). New data from PS1 is consistent with earlier results of M10 on luminosity dependence, but supports slightly weaker dependence of $\mathrm{SF}_{\infty}$ on $M_{BH}$ (by 0.06 dex).}
\label{fig:drw_sf_posterior}
\end{figure*} 



\subsection{Comparison to other studies: Eddington ratio}

Eddington ratio $(f_{Edd} {=} L_{Bol}/L_{\mathrm{Edd}})$ encodes accretion strength: the proximity of quasar bolometric luminosity to the theoretical Eddington limit, where $L_{\mathrm{Edd}} {=} 1.26 {\times} 10^{38} (M_{\mathrm{BH}} / M_{\odot})$ erg/s \citep{shen2011}. Since $\tau$ and $\mathrm{SF}_{\infty}$ depend on $M_{i}$ and $M_{BH}$, we investigate the possibility of the Eddington ratio being the driver of these observed trends. On Fig.~\ref{fig:eddington_ratio} we show $f_{Edd} $ as a function of $M_{i}$, $M_{BH}$, and $\mathrm{SF}_{\infty}$. The left two panels depict  $f_{Edd} $ and $\mathrm{SF}_{\infty}$ binned as a function of $M_{i}$ and $M_{BH}$. Third panel shows the quasar counts, and the fourth panel the bin means (black dots). The means are further binned along  $f_{Edd} $ (as in M10). Combined SDSS-PS1 data supports $\mathrm{SF}_{\infty}$  being inversely related to  $f_{Edd} $, with power-law slope of $-0.206 \pm 0.03$, consistent with  $-0.23 \pm 0.03$ reported by M10. 

\begin{figure*} % ../code2/Trends_SFinf_Eddington_ratio.ipynb
\epsscale{1.2}
\plotone{figs/fEdd_SFinf_celerite_Shen2011_sdss-ps1_gridsize_15_sqr_limN.png}
\caption{First panel: Eddington ratio $f_{Edd} = L/L_{Edd}$ plotted as a function of $M_{BH}$  vs $M_{i}$(from \citealt{shen2011}).  Second panel: $\mathrm{SF}_{\infty}$  binned on the same grid of $M_{BH}$, $M_{i}$. Third panel:  counts of quasars in each bin. We only plot bins with $N>5$ quasars. Fourth panel : the median $f_{Edd}$ and $\mathrm{SF}_{\infty}$ per  bin are plotted as  green crosses. These are aggregated along $f_{Edd}$, in 10 bins of such  width that each has the same number of points. Open circles mark the median $SF_{\infty}$ per $f_{Edd}$ bin, with errors defined as $\sigma_{y} = 1.25  \sigma_{G}(bin) / N$, where $N$ is number of points per bin, and $\sigma_{G}$ is the robust estimate  of the standard deviation ($\sigma_{G} = 0.7413 (Q_{75}-Q_{25})$). We assume the uncertainty along $f_{Edd}$ as $\sigma_{x} = w/\sqrt{12} $, with $w$ denoting the bin width \citep{ivezic2014}. The solid orange line is the best-fit slope:  $-0.206 \pm 0.036$, with the slope uncertainty estimated from the standard deviation of the posterior samples. The best-fit slope agrees with M10 results ($-0.23 \pm 0.03$), plotted as dashed magenta line.}
\label{fig:eddington_ratio}
\end{figure*} 

%Additional PS1 data extending SDSS quasar light curves suggests that characteristic variability timescale is more strongly dependent on luminosity. This is consistent with \citet{sun2018}, who concluded with Structure Function analysis of their luminosity-matched quasar sample, that $\tau$ depends mostly on the bolometric luminosity. 


%
%
%
%

\section{Discussion}
\label{sec:discussion}
\subsection{Trends with Eddington ratio}
Anticorrelation of  variability amplitude with Eddington ratio  has a variety of possible theoretical explanations. In the thin disk theory \citep{shakura1973, frank2002, netzer2013}, radius of the emission region at given wavelength increases with Eddington ratio, and is inversely proportional to temperature \citep{rakshit2017}. Thus a hotter disk means that the emission observed in a given bandpass is emitted from a larger radius. From causality, a smaller region can be more variable than a larger one. Therefore, a  hotter disk would be less variable at a given wavelength than a colder one, and  the variability amplitude as studied in a particular bandpass (here, SDSS r-band) would be anticorrelated with Eddington ratio \citep{fausnaugh2016,edelson2015}. 

On the other hand, in the strongly inhomogeneous disk model independent temperature fluctuations in $N$ zones drive the variability \citep{dexter2011}. In that framework the inverse trend of variability amplitude against $L/L_{Edd}$  and $L_{Bol}$  can be understood qualitatively if more luminous quasars also have higher mass accretion rate, and thus greater number of disk inhomogeneities, resulting in smaller flux variability \citep{simm2016}. The inhomogeneous disk model was consistent with mean SDSS spectral analysis in \citet{ruan2014}, but was not a preferreed explanation for \citet{kokubo2015}. 

Both \citet{rumbaugh2018} (with Dark Energy Survey structure function study) and  \citet{sun2018}  (with a low-z subsample of S82 SDSS quasars) confirm the anti-correlation between quasar variability and luminosity. However, \citet{graham2019} do not find support for this trend with the sample of extremely variable quasars (EVQs) in the CRTS dataset,  but when selecting for lower luminosity sources ($M_{V} < -23$), the anti-correlation is recovered. This agrees with an interpretation that a dwindling fuel supply may correspond to higher variability. Furthermore, \citet{sanchez2018} combined the SDSS spectra with 5 year light curves of 2345 quasars obtained with Quasar Equatorial Survey Team (QUEST)-La Silla AGN Variability Survey, and  using the Bayesian parametrization of Structure Function \citep{schmidt2010} they also found that the amplitude of variability $A$ is anti-correlated with rest-frame emission wavelength,  and Eddington ratio (also see \citealt{simm2016}, \citealt{rakshit2017}).

Indeed, $f_{Edd}$ is a proxy for the strength of accretion, which together with orientation may be the key to explaining quasar main sequence (QMS: \citealt{shen2014, marziani2018}). The QMS, defined by so-called Eigenvector-1, is the anti-correlation between the broad line Fe{\sc ii} emission, and the strength of the narrow O{\sc iii} ($5007$ $\mbox{\AA}$) line \citep{wang1996}. An analysis of quasar clustering by \citet{shen2014}, later confirmed by \citet{sun2015} with measurements of  black hole mass from the quasar host galaxy stellar dispersion \citep{ferrarese2000, kormendy2013}, showed that the entire diversity of quasars in  QMS can be explained by the variation in accretion (affecting $R_{\mathrm{Fe  II}}$ - the ratio of the  Fe{\sc ii} EQ  between $4435-4685$ $\mbox{\AA}$ and H$\beta$), or orientation effects (affecting the FWHM of the H$\beta$). However, \citet{panda2019a, panda2019b} found that these are insufficient, and variations in metallicity, as well as a range of cloud densities, and turbulences are required.  \cite{jiang2016} also found that metallicity, and in particular the iron opacity bump, may have a strong influence on the stability of an accretion disk, and thus linking metallicity to AGN variability. This is also consistent with findings of \cite{sun2018}: quasars with high  Fe{\sc ii} strength have higher metallicity, and have more stable disks. 



\subsection{Variability Timescales}

In the era of changing-look active galaxies (including initially distinct classes of Changing-Look Quasars \citep{lamassa2015, macleod2019}, Changing-Look AGNs \citep{marchese2012, bianchi2009,risaliti2009}, Changing-Look LINERS \citep{frederick2019} to name a few) there is a revived interest in possibly linking the behavior of stellar-sized accreting systems (eg. Black Hole Binaries),  to that of galactic scale (eg. AGN, QSO, LINERS  - \citealt{noda2018, ruan2019}). 

Several relevant timescales are involved, and there are various interlinked mechanisms that could drive the variability. A standard optically thick, geometrically thin, $\alpha$-disk model has a hierarchy of timescales: dynamical, thermal, front, viscous, with   $t_{dyn} < t_{th} < t_{front}  < t_{visc} $ \citep{netzer2013, frank2002}. We proceed to describe briefly each timescale, concluding with our interpretation of the mechanism that could drive the variability observed from the data. 

The dynamical, or gas orbital, timescale is simply  an inverse of the Keplerian orbital angular frequency $ \Omega$  at radius $R$: 

\begin{equation}
t_{dyn} {\sim}  1 / \Omega = \left( \frac{GM}{R^{3}}\right)^{-1/2}
\end{equation}


The main parameter  describing the accretion disk is $\alpha$ - the ratio of the (vertically averaged) total stress to thermal (vertically averaged) pressure: 

\begin{equation}
\alpha= \frac{\langle \tau_{r\varphi}  \rangle_{z} }{\langle P \rangle _{z}} 
\end{equation}


After \cite{lasota2016},  the hydrodynamical stress tensor (corresponding to  kinematic viscocity $\nu$) is:

\begin{equation}
\tau_{r\varphi } = \rho \nu \frac{\partial v_{\varphi}}{\partial R} = \rho \nu \frac{d \Omega}{d \ln{R}} = \frac{3 \rho \nu \Omega}{2}  
\end{equation}

so  with  $c_{s}$ -  local sound speed at radius $R$ (isothermal sound speed is $c_{s} = \sqrt{P/\rho}$),

\begin{equation}
\alpha  =   \frac{3 \rho \nu \Omega}{2 P} =  \frac{3 \Omega \nu}{2 c_{s}^{2}}
\end{equation}



This means that smaller $\alpha$ corresponds to less viscous disks. 


The thermal timescale, related to the time needed for re-adjustment to the thermal equilibrium (derived in detail in \citealt{frank2002}), is the ratio of heat content per unit disk area to dissipation rate per unit disk area: $(dE / A) / (dE/dt /  A) = dt $.  The heat content per unit volume is ${\sim} \rho k T / \mu m_{p} {\sim} \rho c_{s}^{2}$, and heat content per unit area is  ${\sim} \rho c_{s}^{2} / h {\sim} \Sigma c_{s}^{2}$. Meanwhile, the dissipation rate per unit area, $D(R)$, is 

\begin{equation}
D(R) = \frac{9}{8} \nu \Sigma R^{-3} G M
\end{equation}

(eq. 4.30 in \citealt{frank2002}), so :

\begin{equation}
t_{th} {\sim} \frac{c_{s}^{2}R^{3}}{G M \nu } = \frac{c_{s}^{2}}{\nu \Omega} = \frac{t_{dyn}}{\alpha}
\end{equation}

Thus if the disk were inviscid ($\nu \rightarrow 0$), then $t_{th}\rightarrow\infty$ i.e. there is no contact with adjacent disk elements. 

The cooling and heating fronts propagate through the disk at  $\alpha c_{s} $ \citep{hameury2009}  - in that description  with no viscosity there is no communication between neighboring disk annuli, and thus no front propagation \citep{balbus1998, balbus2003}. Following \cite{stern2018}, if we define as $h/R$  the the disk aspect ratio, with the disk height $h = c_{s} / \Omega$, the characteristic time for front propagation is:

\begin{equation}
t_{front} {\sim} (h/R) ^ {-1} t_{th}
\end{equation}


The viscous timescale is the characteristic time it would take for a parcel of material to undergo a radial transport due to the viscous torques from the radius $R$ to the black hole \citep{czerny2006}. Note that while viscosity has probably magnetic origin \citep{eardley1975, grzedzielski2017}, in this simplistic order of magnitude estimate we use a hydrodynamical description of accretion flow.  With $\nu = \eta / \rho$ (kinematic viscosity being the ratio of dynamical viscosity to density), \cite{frank2002} shows (Chap.5.2) that 

\begin{equation}
t_{visc} {\sim} R^{2} / \nu {\sim}  R / v_{R} = (h/R)^{-2} t_{th}
\end{equation}

 We can parametrize each timescale for a black hole mass $M_{BH} = 10^{8} M_{\odot}$, at $R {\sim} 150 r_{g}$, with the gravitational radius $r_{g} = GM_{BH} / c^{2} {\sim} 4 \mathrm{au}$, using Eqs.5-8 in \cite{stern2018} : 


 \begin{equation}
 t_{dyn} {\sim} 10  \mathrm{days} \left(\frac{M_{\mathrm{BH}}}{10^{8} M_{\odot}} \right) 
 \left( \frac{R}{150 r_{g}}\right) ^{3/2} 
 \end{equation}

 \begin{equation}
 t_{th}   {\sim} 1 \,\mathrm{year} \left( \frac{\alpha}{0.03}\right)^{-1}  
 \left( \frac{M_{\mathrm{BH}}}{10^{8} M_{\odot}}\right) \left( \frac{R}{150 r_{g}}\right)^{3/2} 
 \end{equation}

  \begin{eqnarray}
  t_{front} {\sim} 20 \,\mathrm{years} \left( \frac{h/R}{0.05}\right)^{-1}   \left( \frac{\alpha}{0.03}\right)^{-1}  \nonumber  \\ 
  \left( \frac{M_{\mathrm{BH}}}{10^{8} M_{\odot}}\right)     \left( \frac{R}{150 r_{g}}\right) ^{3/2} 
 \end{eqnarray}

  \begin{eqnarray}
  t_{visc}  {\sim} 400 \, \mathrm{years} \left( \frac{h/R}{0.05}\right)^{-2}   \left( \frac{\alpha}{0.03}\right)^{-1} \nonumber  \\  
  \left(\frac{M_{\mathrm{BH}}}{10^{8} M_{\odot}} \right)     \left( \frac{R}{150 r_{g}}\right) ^{3/2}  
 \end{eqnarray}


In summary,  of  considered timescales only thermal and dynamical are short enough to be related to  the observed short-term stochastic variability. It may be that the variability on the scale of days is driven by local changes, and on the longer scale (perhaps hundreds of days) by a different mechanism \citep{kokubo2015}. The other time scales may be more related to the dramatic changes in brightness of the continuum as observed in changing-look AGN. Indeed, \citet{noda2018} favor a change in mass accretion rate, followed by a propagation of the cooling front \citep{lawrence2018, simm2016}. \citet{noda2018} also suggest that perhaps some short-term variability could be related to the amount of the disk swept by the thermal front propagation due to Hydrogen ionization instability, similarly to white dwarf systems (also, see \citealt{ruan2019, ross2018, sniegowska2019}). 


The variability on several years timescale could also be explained by the X-ray reprocessing model \citep{kokubo2015, kubota2018}, assuming that the AGN UV-optical variability is a result of reprocessing of X-ray or far-UV emission \citep{krolik1991}.  The idea of X-ray reprocessing  over time has gained more and more support, with evidence from simultaneous X-ray-UV-optical AGN time series \citep{edelson2014, mchardy2018,  zhu2018}. In particular, the accretion disk blackbody emission is insufficient to explain the broadband AGN spectrum. The total SED with a soft X-ray excess, and a hard X-ray tail, requires additional sources of emission. A recent model by \citet{kubota2018} divides the flow into blackbody emission, warm Comptonization region (the disk), and hard X-ray hot Comptonization component (corona, or a hot material filling the region close to the black hole below the truncation radius).  Since the soft X-rays are correlated with the hard X-rays, at least part of the picture consists of reflection or reprocessing of hard X-rays by the disk \citep{lawrence2018}. This model predicts an increase of variability amplitude ($SF_{\infty}$) with $M_{BH}$, and adds an insight that the observed slope is due to changes in accretion rate $\dot{m}$, explaining that smaller $\dot{m}$ corresponds to highest variability. This qualitatively agrees with the picture that dwindling fuel supply makes the flow more variable. Previous worries about X-ray reprocessing concerned the seemingly insufficient solid angle subtended by the source of the hard X-rays to cause the observed soft X-ray and optical response. This is addressed by realizing that reprocessing could be taking place in the extended region \citep{gardner2017}, such as an inflated inner disk (corresponding to warm Comptonizing region in \citealt{kubota2018}, or even the BLR region serving as an additional `complex reprocessor'\citep{mchardy2018}. Also, for \citet{panda2019a} warm corona helps decrease the dependence of $R_{\mathrm{Fe  II}}$ on $f_{Edd}$. 

Thus while CLAGN may be related to the state-change to ADAF flow \citep{sniegowska2019}, similar to that of XRBs \citep{noda2018,ruan2019}, with cooling and heating fronts \citep{ross2018}, the short-timescale variability requires approximately three distinct emission regions \citep{kubota2018}, with extended reprocessor (such as diffuse, hot, puffed-up inner disk, and BLR - \citealt{mchardy2018}), that reverberates the rapid hard X-ray variability in soft X-rays to optical via UV \citep{fausnaugh2018}. Some emission (especially soft X-rays) seems to require the warm Comptonizing corona \citep{kubota2018}. The warm corona, coupled with metallicity changes, and variation in turbulence level and cloud density, also helps explain the Quasar Main Sequence in the optical \citep{panda2019a,panda2019b}. Finally, the \citet{kubota2018} model, apart from being consistent with other mechanisms \citep{mchardy2018, panda2019a, sniegowska2019, lawrence2018, ross2018, ruan2019}, explains the observed correlation of variability amplitude with black hole mass as corresponding to variations in mass accretion rate. 


%
%
%
%

\section{Summary and Conclusions}
\label{sec:conclusions}
We model the optical variability of ${\sim} 9000$ Stripe 82 quasars as the Damped Random Walk (DRW, \citealt{kelly2009}). We treat the DRW as a Gaussian Process (GP), described by two parameters - characteristic timescale $\tau$ (representing decorrelation timescale, or light curve smoothness), and the asymptotic amplitude $SF_{\infty}$ (which relates to the amplitude of variability). We fit observed and simulated light curves with \project{celerite} - a fast GP solver \citep{foreman2017}. By simulating $10 000$  DRW light curves spanning length of 8 years, with 60 or 445 epochs, we explore the impact of the ratio of input timescale to the light curve baseline. We find that the light curve needs to be several times the length of input timescale to allow unbiased timescale retrieval, confirming K17. Motivated by this result we consider extending SDSS with PS1, PTF, CRTS, and ZTF data.  We calculate appropriate photometric offsets (color terms) to relate PS1 gri, PTF gR,  CRTS V, and ZTF r to SDSS r-band. However, due to larger photometric uncertainties of PTF, ZTF ,CRTS at faint magnitudes of SDSS quasars, we decide to use only PS1 r-band data. Furthermore, SDSS r-band and PS1 r-band are sufficiently similar that no photometric transformation is required.  Thus by extending the SDSS r-band light curves with PS1 DR2 r-band data we improve on the fidelity of recovered DRW parameters. 

We identify 38 objects which exhibit tenfold increase in variability timescale when using the SDSS-PS1 dataset as compared to timescale inferred from SDSS alone. Visual inspection of their light curves show characteristics of changing-look quasars (magnitude difference larger than 0.5 mag). Of these, 5 are confirmed in the literature ~\citep{macleod2019, lamassa2015}. 

We test the correlation of quasar physical properties, such as black hole mass $M_{BH}$ and absolute i-band magnitude $M_{i}$,  with DRW model parameters. The SDSS-PS1 data, coupled with \citet{shen2011} quasar catalog, imply that the damping timescale $\tau$  is correlated with $M_{BH}$ with a power-law index of $0.128 \pm 0.019$, and almost independent of quasar bolometric luminosity, as in M10, \citet{wilhite2008}, and \citet{vandenberk2004}. The asymptotic variability amplitude $\mathrm{SF}_{\infty}$ is anti-correlated with luminosity $M_{i}$ with power law index $0.117\pm0.003$, and correlated with $M_{BH}$ with slope $0.11\pm 0.008$. This can be explained if the driving principle was Eddington ratio $f_{Edd}$ \citep{wilhite2008}. Indeed, there is an anti-correlation of $\mathrm{SF}_{\infty}$ and $f_{Edd}$, with power law slope of $-0.206 \pm 0.036$ (similar to M10).  As suggested by \citet{kubota2018}, this gradient of $\mathrm{SF}_{\infty}$ in the plane of $M_{BH}$ vs $M_{i}$ could be explained if the lower mass accretion rate corresponds to higher variability, so that when the supply of fuel decreases, the flow becomes less stable, more clumpy, and more inhomogeneous ~\citep{rakshit2017, kokubo2015, dexter2011}.  This is also consistent with the X-ray reprocessing model, whereby the hard X-ray variability of the inner disk is reflected/reprocessed by the extended warm Comptonization region (inflated disk), and perhaps a complex reprocessor, including the clouds of the broad line region~\citep{kubota2018, panda2019b}. Changes on recovered timescales are too fast to be driven by changes in disk viscosity, or thermal front propagation alone - thermal or dynamical timescale of response to the changes in X-ray emission seems most consistent with our results ~\citep{stern2018}.



% perhaps skip, about the impact of PSD shape on the RM studies  .... 
%The exact shape of quasar PSD would affect other areas of study. One example is reverberation mapping (RM), which was used to provide  accurate black hole mass estimates. RM is based on measuring the lags between light curves observed at different wavelengths. The two most widely used approaches to measuring interband lags are interpolated cross-correlation function (ICCF, \citealt{gaskell1987, peterson2004})  and  light curve modeling via Markov chain Monte Carlo approach (JAVELIN, \citealt{zu2011}). The PSD of the DRW comes explicitly in the latter,  which assumes that the higher energies drive lower energies (whether due to x-ray reprocessing or other mechanism), and that the driving light curve is well-modeled by a DRW (with PSD equal to  $-2$ or flatter).  In this scenario other light curves (at longer wavelength) are related to it via transfer function \cite{edelson2019}. Using a wrong PSD means at best that errors are underestimated. Therefore a convincing detection of a departure of quasar PSD from that of the DRW  would have  a direct impact on RM studies that use  JAVELIN or other light curve modeling tools. 


More data extending the light curves would help improve the DRW fit coefficients, decreasing the scatter in observed correlations. Moreover, given that the uncertainty in black hole mass is one of the biggest sources of error, better measurements of quasar properties would be of high utility~\citep{shen2011}.  This will be possible by the upcoming AGN reverberation mapping campaigns (eg. SDSS-V Black Hole Mapper), providing better calibration for line width-based methods of estimating black hole masses~\citep{kollmeier2017}. All quasars in this study were spectroscopically confirmed, but some spectra had low signal-to-noise, resulting in higher likelihood of incorrect redshift measurement. Better spectroscopy and follow up of S82 quasars, afforded by SDSS-V panoptic spectroscopy, would not only help improve on the spectrum-based  properties (redshift, absolute magnitude, black hole masses), but also allow to study spectral changes, and further new CLAGN discoveries~\citep{macleod2019}. 

If this study were to be expanded onto a sample of quasars with good photometry over sufficiently long baselines, but lacking spectral information, the required physical information on quasars could be obtained by indirect methods of estimating the coarse spectral information from broad-band photometry~\citep{kozlowski2015}. This would benefit from better catalogs of existing spectroscopically-confirmed quasars (SDSS DR14) to improve the calibration, as well as better methods of estimating the redshift based on photometry alone (eg. photo-z, \citealt{graham2018}).  This will be possible in short term with the ZTF \citep{bellm2018}, and in the long term with LSST~\citep{ivezic2019}. Occasional coverage adding few epochs to some quasars may be possible with other surveys (eg. TESS, \citealt{ricker2014}), but to improve the statistics of an entire sample of S82 quasars would require longer baselines. Combining SDSS and PS1 with LSST would provide unprecedented 35-year baseline, which assuming timescales between 10-1000 days is over 10 times longer, allowing unbiased DRW parameter retrieval, which coupled with correlations with quasar properties, would provide an estimate of black hole masses and bolometric luminosities for millions of quasars~\citep{ivezic2019}. 


\section{Acknowledgements}

% FROM https://panstarrs.stsci.edu 
The Pan-STARRS1 Surveys (PS1) and the PS1 public science archive have been made possible through contributions by the Institute for Astronomy, the University of Hawaii, the Pan-STARRS Project Office, the Max-Planck Society and its participating institutes, the Max Planck Institute for Astronomy, Heidelberg and the Max Planck Institute for Extraterrestrial Physics, Garching, The Johns Hopkins University, Durham University, the University of Edinburgh, the Queen's University Belfast, the Harvard-Smithsonian Center for Astrophysics, the Las Cumbres Observatory Global Telescope Network Incorporated, the National Central University of Taiwan, the Space Telescope Science Institute, the National Aeronautics and Space Administration under Grant No. NNX08AR22G issued through the Planetary Science Division of the NASA Science Mission Directorate, the National Science Foundation Grant No. AST-1238877, the University of Maryland, Eotvos Lorand University (ELTE), the Los Alamos National Laboratory, and the Gordon and Betty Moore Foundation.

% FROM  https://www.sdss.org/collaboration/citing-sdss/
Funding for the Sloan Digital Sky Survey IV has been provided by the Alfred P. Sloan Foundation, the U.S. Department of Energy Office of Science, and the Participating Institutions. SDSS-IV acknowledges
support and resources from the Center for High-Performance Computing at
the University of Utah. The SDSS web site is www.sdss.org.
SDSS-IV is managed by the Astrophysical Research Consortium for the 
Participating Institutions of the SDSS Collaboration including the 
Brazilian Participation Group, the Carnegie Institution for Science, 
Carnegie Mellon University, the Chilean Participation Group, the French Participation Group, Harvard-Smithsonian Center for Astrophysics, 
Instituto de Astrof\'isica de Canarias, The Johns Hopkins University, Kavli Institute for the Physics and Mathematics of the Universe (IPMU) / 
University of Tokyo, the Korean Participation Group, Lawrence Berkeley National Laboratory, 
Leibniz Institut f\"ur Astrophysik Potsdam (AIP),  
Max-Planck-Institut f\"ur Astronomie (MPIA Heidelberg), 
Max-Planck-Institut f\"ur Astrophysik (MPA Garching), 
Max-Planck-Institut f\"ur Extraterrestrische Physik (MPE), 
National Astronomical Observatories of China, New Mexico State University, 
New York University, University of Notre Dame, 
Observat\'ario Nacional / MCTI, The Ohio State University, 
Pennsylvania State University, Shanghai Astronomical Observatory, 
United Kingdom Participation Group,
Universidad Nacional Aut\'onoma de M\'exico, University of Arizona, 
University of Colorado Boulder, University of Oxford, University of Portsmouth, 
University of Utah, University of Virginia, University of Washington, University of Wisconsin, 
Vanderbilt University, and Yale University.

% I didn't use CRTS data in the end  ...
% from http://nesssi.cacr.caltech.edu/DataRelease/policy.html  
The CSS survey is funded by the National Aeronautics and Space Administration under Grant No. NNG05GF22G issued through the Science Mission Directorate Near-Earth Objects Observations Program.  The CRTS survey is supported by the U.S.~National Science Foundation under grants AST-0909182.


% I don't use ZTF DR1 directly, but I do show the error curve ...
% from  https://www.ztf.caltech.edu/page/dr1#1a
Based on observations obtained with the Samuel Oschin 48-inch Telescope at the Palomar Observatory as part of the Zwicky Transient Facility project. ZTF is supported by the National Science Foundation under Grant No. AST-1440341 and a collaboration including Caltech, IPAC, the Weizmann Institute for Science, the Oskar Klein Center at Stockholm University, the University of Maryland, the University of Washington, Deutsches Elektronen-Synchrotron and Humboldt University, Los Alamos National Laboratories, the TANGO Consortium of Taiwan, the University of Wisconsin at Milwaukee, and Lawrence Berkeley National Laboratories. Operations are conducted by COO, IPAC, and UW.





\appendix
\section{Measuring Quasar properties} 
\label{app:measureBHmass}
In this work we employ black hole masses, bolometric luminosities, and K-corrections from \cite{shen2011} catalog, based on single-epoch SDSS spectra. Here we explain the choices made in the difficult art of estimating each of these quasar physical properties.

It is non-trivial to measure the mass of black holes living in the centers of active galaxies, even provided a detailed spectrum. The most common  approach to estimate black hole masses in AGN is to assume that the broad-line region (BLR) is virialized:


\begin{equation}
M_{BH} = f \frac{ R\Delta V^{2} }{G} = f M_{vir}
\end{equation}


where $f$ is a constant of order unity, $R$ is the size of the BLR (estimated from  emission-line lag $\Delta t$ as $R = c \Delta t$), $\Delta V$ is virial velocity, $G$ gravitational constant \citep{shen2008}.  From reverberation mapping studies (eg. \citealt{shen2018}) we know that continuum luminosity $L$ is related to the size of the BLR region as $R \propto L^{\gamma}$ \citep{vestergaard2006}, with $\gamma$ very close to $1/2$ (eg. \citealt{bentz2009} finds from RM studies $\gamma = 0.519 \pm 0.06$). Thus we find  that $R \Delta V^{2} \propto L^{\gamma} \Delta V^{2} \equiv \mu$. The virial velocity $\Delta V$ is usually estimated from the width of the broad emission lines (or line dispersion).  In the absence of a quasar spectrum, there are alternative methods using a conversion of the broad-band photometry into monochromatic fluxes in the vicinity of reverberating lines (eg. \citealt{kozlowski2015}, used in \citealt{kozlowski2017b} to estimate black hole mass for 280 000 AGN).   
Depending on the redshift, different rest-frame calibrated emission lines shift into the observed passband: broad H$\alpha$ at  $6562\mbox{\AA}$, H$\beta$ at  $5100 \mbox{\AA}$, Mg\,{\sc ii} at $3000\mbox{\AA}$, and C\,{\sc iv} at $1350\mbox{\AA}$ (see Fig.7 in \citealt{shen2018}, and \citealt{vestergaard2002}). Some authors even consider separately C\,{\sc iv}-based and  Mg\,{\sc ii}-based  black hole mass estimates. We refer the reader to \citet{shen2008} who in detail describes various biases and inherent assumptions of virial black hole mass measurements. 



Another important quasar property - bolometric luminosity, is most often estimated from the absolute i-band magnitude, $M_{i}$ (see \citealt{shen2008}, Fig.2). $M_{i}$ is derived from the observed i-band magnitude, by correcting for Galactic extinction, and correcting for the fact that at different redshifts different portions of the spectral energy distribution are observed by the telescope filter bandpass. The latter, known as K-correction $K(z)$ \citep{oke1968},  is defined as $m_{intrinsic} = m_{observed} - K(z)$. In the early 2000`s the common approach was to K-correct to redshift 0, but as \citep{richards2006a} pointed out, since the distribution of quasars peaks at redshift 2, for most quasars correcting to the redshift of 0 required shifting the observed spectrum into the far infrared. Moreover, the procedure was to correct separately for the continuum and emission line contributions, assuming a particular spectral shape (eg. power law  $f_{\nu} \propto \nu^{\alpha}$, with $\alpha=-0.5$ - see \citealt{schneider2010, vandenberk2001, richards2006a}).  This introduces a larger error for K(z=0) than for K(z=2) if the assumed spectral shape $\alpha=-0.5$ is far from the real spectral index. In early 2010`s, after  \citealt{richards2006a, wisotzki2000, blanton2003},  the practice started shifting towards K-correcting to redshift 2,  and including custom qusasar spectral shapes, as reflected by the content of \cite{shen2011} quasar catalog. Thus in this study we use  the absolute i-band magnitude K-corrected to z=2: $M_{i}(z=2)$. % from \citep{shen2011}. 



These methods were used to create catalogs of quasar properties derived from spectra. Since quasars are variable at ${\sim}0.2$ mag level, the ideal is to use a single-epoch calibrated spectrum to estimate the continuum luminosity, and find virial black hole masses using relationships based on the monochromatic fluxes and broad line widths described above. A glance at the available quasar catalogs reveals that, given any SDSS data release, there is indeed first a catalog of basic quasar properties (redshift and photometry - eg. \citealt{schneider2007, schneider2010}), and more detailed catalogs containing black hole masses and bolometric luminosities  follow (eg. \citealt{shen2008, shen2011}). More recently, once SDSS DR12 Quasar Catalog \citep{paris2017} was released,  \citealt{kozlowski2017a} followed using SDSS photometry as a proxy for monochromatic luminosities. \citet{chen2018} added a detailed analysis of continuum luminosities in the  H$\alpha$, H$\beta$ regions for low-redshift quasars. Using the spectra from Chinese LAMOST survey \citet{dong2018} also sought to estimate virial black hole masses, and the results, while consistent with \citet{shen2011}, suffered from the necessity to peg the non-calibrated spectra to the SDSS photometry which was taken  a different epoch. Thus even though the SDSS DR12 Quasar Catalog of \cite{paris2018}  is the most recent, like \citet{paris2017} it lacks black hole masses and bolometric luminosities, and there is no recent work that re-analyzed the spectral data. Therefore we use black hole mass estimates and monochromatic luminosities from \citet{shen2011}, based directly on single-epoch spectra. 


% \section{Simulated light curves}

% \begin{figure*} % code2/Explore_simulation_results.ipynb 
% \plotone{figs/macleod2011_Fig18_Jeff1-190401.png}
% \caption{Comparison of retrieved parameters in relation to input parameters, shown as Fig.18 in \citet{macleod2011} }
% \label{fig:lc_simulatedresults2}
% \end{figure*} 



\section{CLQSO candidates}
\label{app:clqso_cands}

Based on the DRW model  parameters $\tau$, $\sigma$ fitted with \project{celerite} using the SDSS and PS1 data, we find that there are quasars for which there is a pronounced difference between $\tau$, $\sigma$  inferred from combined SDSS-PS1 data vs just SDSS. Specifically, Fig.~\ref{fig:sigma_tau_ratios} shows that there are objects where 
$f_{\sigma} = \log_{10}{\left( \sigma_{\mathrm{SDSS-PS1}} / \sigma_{\mathrm{SDSS}} \right)} > 0.4 $ and 
 $f_{\tau} = \log_{10}{\left( \tau_{\mathrm{SDSS-PS1}} / \tau_{\mathrm{SDSS}} \right)} > 1  $.  This corresponds to tenfold increase in $\tau$ and over twofold increase in $\sigma$ ($10^{0.4} = 2.51$). By visual inspection of objects simultaneously satisfying the threshold $f_{\sigma}> 0.4$ and $f_{\tau}> 1 $ we find that these undergo a significant ($>0.5$ mag) change in brightness between the SDSS (baseline 1998 - 2008) and PS1 DR2 observations (2009-2014 - see Fig.~\ref{fig:lc_extent}). Thus DRW fitting could also be a way of finding changing-look quasar (and AGN) candidates. Figs.~\ref{fig:clqso1}--~\ref{fig:clqso4} show the SDSS-PS1 r-band light curves of 38 CLQSO candidates, with median SDSS part brighter than 20.5 mag. The open circles indicate day-averaged epochs (see Sec.~\ref{sec:data}). Table~\ref{tab:clqso} contains the basic physical parameters for these quasars. 

\begin{figure*} % made with ../code2/Explore_SDSS_PS1_lightcurves.ipynb 
\plotone{figs/SDSS_PS1_outliers_page_0.png}
\caption{Outliers in the space of recovered DRW parameters between SDSS and SDSS-PS1, as well as median offsets. Page 1.}
\label{fig:clqso1}
\end{figure*}

\begin{figure*}
\plotone{figs/SDSS_PS1_outliers_page_1.png}
\caption{As Fig.~\ref{fig:clqso1}, page 2. }
\label{fig:clqso2}
\end{figure*}


\begin{figure*}
\plotone{figs/SDSS_PS1_outliers_page_2.png}
\caption{As Fig.~\ref{fig:clqso1}, page 3. }
\label{fig:clqso3}
\end{figure*}

\begin{figure*}
\plotone{figs/SDSS_PS1_outliers_page_3.png}
\caption{As Fig.~\ref{fig:clqso1}, page 4. }
\label{fig:clqso4}
\end{figure*}



\begin{deluxetable*}{c|CCCCCCC}
%\tablewidth{\pagewidth}
\tablecaption{CLQSO candidates : catalog information from \cite{shen2011}, concerning DR7 name (dbID) and SDSSJID location ($\alpha$, $\delta$, J2000), distance (spectrum-based redshift), and physical parameters (bolometric luminosity  $L_{Bol}$, black hole mass $M_{\mathrm{BH}}$, Eddington ratio $f_{Edd} = L_{Bol} / L_{Edd}$).  \label{tab:clqso} } 

\tablehead{\colhead{dbID} & \colhead{SDSSJID}  & \colhead{$\alpha$} & \colhead{$\delta$} & \colhead{redshift} & \colhead{$\log_{10}{(L_{Bol})}$} & \colhead{$\log_{10}{(M_{\mathrm{BH}})}$} & \colhead{$f_{Edd}$} }
\startdata
8442 & \object[SDSS J001731.70+004910.1]{001731.70+004910.1} & 4.382 & 0.819 & 2.43 & 46.61 & 9.09 & -0.58   \\
123909 & \object[SDSS J001626.54+003632.4]{001626.54+003632.4} & 4.111 & 0.609 & 3.24 & 46.57 & 9.47 & -1.0   \\
221006 & \object[SDSS J005142.20+002129.0]{005142.20+002129.0} & 12.926 & 0.358 & 1.55 & 45.95 & 8.24 & -0.39   \\
257776 & \object[SDSS J005513.15-005621.2]{005513.15-005621.2} & 13.805 & -0.939 & 3.61 & 47.13 & 9.58 & -0.54   \\
292959 & \object[SDSS J232030.97-004039.2]{232030.97-004039.2} & 350.129 & -0.678 & 1.72 & 46.69 & 9.39 & -0.8   \\
307136 & \object[SDSS J232221.81+010733.4]{232221.81+010733.4} & 350.591 & 1.126 & 0.76 & 45.17 & 8.06 & -0.99   \\
467617 & \object[SDSS J231032.17-011449.5]{231032.17-011449.5} & 347.634 & -1.247 & 1.82 & 46.03 & 7.86 & 0.06   \\
568312 & \object[SDSS J231953.07-010139.0]{231953.07-010139.0} & 349.971 & -1.028 & 1.15 & 45.6 & 8.29 & -0.79   \\
612585\tablenotemark{a} & \object[SDSS J010812.00-000516.5]{010812.00-000516.5} & 17.05 & -0.088 & 1.0 & 45.52 & 9.06 & -1.64  
 \\
751557\tablenotemark{b} & \object[SDSS J225240.37+010958.7]{225240.37+010958.7} & 343.168 & 1.166 & 0.53 & 45.32 & 8.88 & -1.66  \\
1003694\tablenotemark{c}  & \object[SDSS J012114.19-010310.8]{012114.19-010310.8} & 20.309 & -1.053 & 1.89 & 46.59 & 8.83 & -0.34   \\
1124333 & \object[SDSS J222918.25-004003.6]{222918.25-004003.6} & 337.326 & -0.668 & 1.16 & 45.81 & 8.35 & -0.64  \\
1299803\tablenotemark{c} & \object[SDSS J014303.23-004354.0]{014303.23-004354.0} & 25.763 & -0.732 & 0.53 & 45.78 & 8.68 & -1.0  \\
1378415 & \object[SDSS J221347.32+001928.4]{221347.32+001928.4} & 333.447 & 0.325 & 2.31 & 46.41 & 8.59 & -0.29   \\
1412379 & \object[SDSS J221831.58-004548.9]{221831.58-004548.9} & 334.632 & -0.764 & 1.23 & 46.15 & 9.48 & -1.43   \\
1446022 & \object[SDSS J220535.23+000756.3]{220535.23+000756.3} & 331.397 & 0.132 & 1.69 & 46.45 & 9.25 & -0.9   \\
1644710 & \object[SDSS J021259.00-000550.1]{021259.00-000550.1} & 33.246 & -0.097 & 0.81 & 45.67 & 8.38 & -0.81   \\
1730482 & \object[SDSS J021529.02-005314.9]{021529.02-005314.9} & 33.871 & -0.887 & 1.37 & 45.98 & 8.8 & -0.92   \\
1901056 & \object[SDSS J215055.51-001739.4]{215055.51-001739.4} & 327.731 & -0.294 & 1.54 & 46.26 & 8.6 & -0.44   \\
1976348 & \object[SDSS J215841.40-001507.7]{215841.40-001507.7} & 329.673 & -0.252 & 1.46 & 46.92 & 9.39 & -0.57   \\
2006852 & \object[SDSS J023917.86-001916.8]{023917.86-001916.8} & 39.824 & -0.321 & 1.41 & 46.07 & 8.73 & -0.76   \\
2061101 & \object[SDSS J022505.06+001733.2]{022505.06+001733.2} & 36.271 & 0.293 & 2.42 & 46.38 & 8.09 & 0.2   \\
2069362 & \object[SDSS J022644.03+003305.8]{022644.03+003305.8} & 36.683 & 0.552 & 2.38 & 46.54 & 9.32 & -0.88   \\
2104791 & \object[SDSS J022239.83+000022.5]{022239.83+000022.5} & 35.666 & 0.006 & 0.99 & 46.28 & 9.33 & -1.16   \\
2484608 & \object[SDSS J025654.42-011455.4]{025654.42-011455.4} & 44.227 & -1.249 & 0.54 & 45.57 & 8.48 & -1.01   \\
3052176 & \object[SDSS J030504.07+011324.5]{030504.07+011324.5} & 46.267 & 1.223 & 0.61 & 45.29 & 9.2 & -2.01   \\
3076365 & \object[SDSS J031759.16-004606.4]{031759.16-004606.4} & 49.497 & -0.768 & 0.65 & 45.36 & 8.68 & -1.43   \\
3219103 & \object[SDSS J205724.14-003018.7]{205724.14-003018.7} & 314.351 & -0.505 & 4.66 & 47.64 & 9.83 & -0.29   \\
3633437 & \object[SDSS J205105.02-005847.5]{205105.02-005847.5} & 312.771 & -0.98 & 0.54 & 45.34 & 8.47 & -1.23   \\
3781306 & \object[SDSS J032745.74+005217.2]{032745.74+005217.2} & 51.941 & 0.871 & 1.16 & 45.77 & 0.0 & -999.0   \\
3858587 & \object[SDSS J032825.19-003252.3]{032825.19-003252.3} & 52.105 & -0.548 & 0.77 & 45.61 & 8.68 & -1.17   \\
3916393 & \object[SDSS J235416.94-000222.5]{235416.94-000222.5} & 358.571 & -0.04 & 0.88 & 45.76 & 8.51 & -0.85   \\
3928645 & \object[SDSS J235344.01+005216.9]{235344.01+005216.9} & 358.433 & 0.871 & 0.65 & 45.37 & 8.71 & -1.44   \\
3946479 & \object[SDSS J235248.71-001518.4]{235248.71-001518.4} & 358.203 & -0.255 & 1.34 & 45.8 & 9.01 & -1.3   \\
3976336 & \object[SDSS J235213.27-004326.3]{235213.27-004326.3} & 358.055 & -0.724 & 0.9 & 45.64 & 8.85 & -1.3   \\
4069419 & \object[SDSS J003359.39+000230.0]{003359.39+000230.0} & 8.497 & 0.042 & 1.64 & 45.95 & 9.05 & -1.21   \\
4205621 & \object[SDSS J203932.41-001818.3]{203932.41-001818.3} & 309.885 & -0.305 & 1.58 & 46.21 & 8.66 & -0.55   \\
4913626 & \object[SDSS J034512.62+002245.7]{034512.62+002245.7} & 56.303 & 0.379 & 0.42 & 45.5 & 8.81 & -1.41   \\
 \enddata
 \tablenotetext{\tiny a}{S82X LaMassa2016, XMM Newton, Galex UV, UKIDSS, VHS, WISE}
 \tablenotetext{\tiny b}{M19, CLQSO candidate, Magellan follow-up}
 \tablenotetext{\tiny c}{S82X LaMassa2019, WISE}
\end{deluxetable*}





%%%%%%%%%%%%%%%%%%%%%%%%%%%%%%%%%%%%%%%%%%%%%%%%%%

%%%%%%%%%%%%%%%%%%%% REFERENCES %%%%%%%%%%%%%%%%%%

% The best way to enter references is to use BibTeX:

\bibliographystyle{aasjournal} 
\bibliography{references}

%%%%%%%%%%%%%%%%%%%%%%%%%%%%%%%%%%%%%%%%%%%%%%%%%%

\end{document}

% End of file `sample62.tex'.
