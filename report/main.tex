% mnras_template.tex
%
% LaTeX template for creating a MNRAS paper
%
% v3.0 released 14 May 2015
% (version numbers match those of mnras.cls)
%
% Copyright (C) Royal Astronomical Society 2015
% Authors:
% Keith T. Smith (Royal Astronomical Society)

% Change log
%
% v3.0 May 2015
%    Renamed to match the new package name
%    Version number matches mnras.cls
%    A few minor tweaks to wording
% v1.0 September 2013
%    Beta testing only - never publicly released
%    First version: a simple (ish) template for creating an MNRAS paper

%%%%%%%%%%%%%%%%%%%%%%%%%%%%%%%%%%%%%%%%%%%%%%%%%%
% Basic setup. Most papers should leave these options alone.
\documentclass[fleqn,usenatbib]{mnras}  % a4paper,

% MNRAS is set in Times font. If you don't have this installed (most LaTeX
% installations will be fine) or prefer the old Computer Modern fonts, comment
% out the following line
%\usepackage{newtxtext,newtxmath}
%\usepackage{lmodern}
% Depending on your LaTeX fonts installation, you might get better results with one of these:
\usepackage{mathptmx}
%\usepackage{txfonts}


% Use vector fonts, so it zooms properly in on-screen viewing software
% Don't change these lines unless you know what you are doing
\usepackage[T1]{fontenc}
\usepackage{ae,aecompl}
\usepackage{diagbox}

%%%%% AUTHORS - PLACE YOUR OWN PACKAGES HERE %%%%%

% Only include extra packages if you really need them. Common packages are:
\usepackage{graphicx}	% Including figure files
\usepackage{amsmath}	% Advanced maths commands
\usepackage{amssymb}	% Extra maths symbols
\usepackage{savesym}  % prevent symbol conflicts
%\generate{%
%  \file{breqn.sty}{\nopreamble\from{breqn.dtx}{breqn.sty}}%
%}
%\usepackage{breqn} % automatic breaking equation 
%\usepackage{fancyvrb}
%\VerbatimFootnotes
\usepackage{cprotect}  % to allow verb in caption 

%%%%%%%%%%%%%%%%%%%%%%%%%%%%%%%%%%%%%%%%%%%%%%%%%%

%%%%% AUTHORS - PLACE YOUR OWN COMMANDS HERE %%%%%

% Please keep new commands to a minimum, and use \newcommand not \def to avoid
% overwriting existing commands. Example:
%\newcommand{\pcm}{\,cm$^{-2}$}	% per cm-squared

%%%%%%%%%%%%%%%%%%%%%%%%%%%%%%%%%%%%%%%%%%%%%%%%%%

%%%%%%%%%%%%%%%%%%% TITLE PAGE %%%%%%%%%%%%%%%%%%%

% Title of the paper, and the short title which is used in the headers.
% Keep the title short and informative.
\title[DRW baseline]{ Improving Damped Random Walk parameters for SDSS Stripe82 Quasars with baseline extension with PanStarrs1 data. }

% The list of authors, and the short list which is used in the headers.
% If you need two or more lines of authors, add an extra line using \newauthor
\author[K. Suberlak et al.]{
Krzysztof Suberlak,$^{1}$\thanks{E-mail: suberlak@uw.edu}
\v{Z}eljko Ivezi\'c $^{1}$
Chelsea MacLeod $^{2}$
\\
% List of institutions
$^{1}$Department of Astronomy, University of Washington, Seattle, WA, United States\\
$^{2}$Harvard Smithsonian Center for Astrophysics, 60 Garden St, Cambridge, MA 02138, United States \\
}

% These dates will be filled out by the publisher
\date{Accepted XXX. Received YYY; in original form ZZZ}

% Enter the current year, for the copyright statements etc.
\pubyear{2018}

% Don't change these lines
\begin{document}
\label{firstpage}
\pagerange{\pageref{firstpage}--\pageref{lastpage}}
\maketitle

% Abstract of the paper
\begin{abstract}

 %Aim: Improve on DRW parameters reported in \cite{macleod2011} by an increase of the QSO light curve baseline.  We compare the tools used to fitting for $\tau$ and $SF_{\infty}$ to those of \cite{kozlowski2017a}.
\end{abstract}


%%%%%%%%%%%%%%%%% NEW NEW NEW BODY %%%%%%%%%%%%%%%%%%%%%%%%%%%%%%%%%

\section{Introduction}

Quasars are variable.  Their light curves have been successfully described using the Damped Random Walk model (Kelly+2009, Macleod+2010,  Kozlowski+2010,Zu+2011, Kasliwal+2015,   ) . The origin of variability is debated, with thermal origin being the favorite explanation (Czerny+1999,  Kelly+2013, ), connected to the inhomogeneity of the accretion disk (Dexter,Agol 2011 ), or even magnetically elevated disks (Dexter, Begelman 2018 ). 

The DRW fit parameters have been linked to the physical quasar properties. Indeed, MacLeod+2010, using the SDSS light curves for Stripe82  found correlations of the characteristic timescale and variability amplitude  with the black hole mass, and quasar luminosity. 

Quasar variability, and specifically modelling it as  a DRW, is also a reliable way to distinguish quasars from stars based on merely optical photometry (MacLeod+2011).  In that case the fit biases are less important than the fact that DRW timescale and amplitude for QSO are order of  magnitude different from stars (MacLeod+2011).  It is especially useful for quasars that could not be easily identified by color-color diagrams (Sesar+2007) - those in the intermediate redshift range, whose colors mimic the M dwarfs (Yang+2017).  

Accurate QSO population studies is important for measurement of QLF, and variability has been used before to increase the completeness of quasar selection  ( Richards+2008, Ross+2013, Palanque-Delabrouille+2013,2016 , AlSayyad+2016, McGreer+2013,2017 ). 


Because DRW is a stochastic process, fitting is more involved than with simpler source variability, such as RR Lyrae, or Eclipsing Binaries that result in well-defined light curve shape.  Indeed,  even with identical input parameters,  two different DRW light curves would have a very different appearance. It has been found (eg. MacLeod+2010,  Kozlowski+2010,2017) that regardless of method,  we can most reliably recover input parameters if we use the longest light curve baseline possible. A rule of thumb is that the light curve has to be at least  ten  times longer than the recovered timescale.  We confirm this observation with simulations of DRW light curves spanning a variety of ratios of input timescale to light curve lenght.  

The light curve baseline is the key in an unbiased recovery of light curve parameters. As it has been  8 years since MacLeod+2010 have published their research,   we can now benefit from additional data from other surveys that have observed the same quasars since. We show how combining the SDSS data with CRTS, PTF , PS1,  and simulated LSST data, decreases the bias in recovered parameters.  Thus with added data, extending the baseline by 50\% on average,  we revisit correlations studied by MacLeod+2010.  We confirm the general trends, and provide forecast for improvement with the advent of ZTF, LSST. Extended baseline is the advantage that is not afforded by studies only using single survey data (eg. Hernitschek+2016)



\section{}

%%%%%%%%%%%%%%%%% OLD OLD OLD %%%%%%%%%%%%%%%%%%

\section{Motivation}
\cite{macleod2011} successfully  derived  many QSO parameters for the DRW model based on fits to SDSS light curves in S82. Encouraged by conclusions of \cite{kozlowski2017a},  we expand baselines of quasar light curves  utilizing data from CRTS and PS1. We show improvement in the accuracy of parameter fit (Hernitschek+2016 sought to improve on parameters, but had insufficient baseline using solely PS1). 

\section{Methods}
 We first confirm the scaling relations by \cite{kozlowski2017a} by testing the retrieval of simulated light curve parameters with Celerite . In addition to reproducing his Fig.2  we also plot the fractional bias due to insufficient length of the  light curve baseline. We confirm that longer  baseline should significantly improve time scale constraints. Light curve error distribution and sampling are less important, i.e. it would improve the accuracy of fit more to have a larger baseline than denser sampling. 

Then we explore the combined SDSS-PTF-CRTS-PS1 Quasar dataset in the Stripe82 footprint (Fig.~\ref{fig:baselines}). 

\subsection{Data}
We start with the SDSS DR7 QSO \citep{schneider2007} near-simultaneous ugriz SDSS photometry \footnote{\url{http://www.astro.washington.edu/users/ivezic/macleod/qso_dr7/Southern.html }}. We also used the CRTS DR2 (Drak et al 2009 ) light curves obtained for these objects by  B.Sesar using a positional query. With the SDSS ra,dec, we obtained the PTF DR1 (Rau et al. 2009) r-band light curves. We use PanSTARRS DR2 (Chambers et al. 2011, Flewelling et al. 2018) grizy light curves  for DR7 QSO from C. MacLeod. We make an outer join of all catalogs,  flagging from which survey came which data point, as well as photometric filter. 


\subsection{Photometric offsets}

To effectively combine all photometric information from various telescopes and imaging filters we re-derived the necessary photometric offsets. We used as our sample blue SDSS S82 stars (-1<g-i<1), limited to this range in color because it closely matches the natural distribution for quasars (see Fig.~\ref{fig:quasar_colors})


\begin{figure}
\includegraphics[width=1.05\columnwidth]{figs/SDSS_S82_QSO_colors.png}
%\vskip -0.15in
\caption{Color - color diagrams for 6444 SDSS S82 quasars with PTF, PS1 and CRTS photometry. The convention is to always define a 'color' by subtracting the redder filter from the bluer filter. That way any color has negative values for intrinsically bluer objects - emitting more in the blue part of the spectrum (eg. QSO, RR Lyr), and positive values for redder objects (eg. M stars). Thus using the SDSS base of 'u g r i z' colors,  we form u-g, g-r,  r-i, i-z  colors, as well as g-i, which skips the r filter. Another convention is to plot the bluer color on x-axis (eg. u-g) vs redder color (eg. g-r) on y axis (see \citealt{ivezic2002,sesar2007}). Thus from the upper-left panel to bottom-right panel we cycle through color pairs showing that quasars occupy a distinct locus in each color combination. To calculate photometric offsets between SDSS and PS1, PTF, CRTS, we employ standard stars with colors based on the region occupied by quasars in the u-g vs g-i color space. }
\label{fig:quasar_colors}
\end{figure} 



We find photometric offsets as follows : 

\begin{equation}
m - s = f(x)
\end{equation}

where $m$ is the mean magnitude for a target survey (eg. PS1\{g,r,i,z,y\}, or PTF\{g,R\}),  $s$ is the synthetic SDSS magnitude in a given band, and $x$ is the median SDSS color (eg. $g-i$).   ( For instance, \ref{tonry2012} derived all offsets against $x$ = SDSS($g-r$), $m = PS1 \{g,r,i,z,y\}$, and $s$ = SDSS($r$)



The issue of interstellar extinction : due to dust present between us and the standard stars (or background quasars), the observed light will appear slightly redder because dust preferentially scatters blue light away. This depends on the location of the source on the  sky and is related to the dust inhomogeneities in the Milky Way. 

However, in deriving bandpass to bandpass transformation all that matters is the flux received at the Top of the Atmosphere (TOA), for which all photometry is corrected at the pipeline processing level (both for SDSS,  PTF, PS1, and CRTS). In other words, all that matters is that any SED at TOA will have slightly different magnitude (hence colors) depending on whether we observe it with SDSS(r),  PS1(r), or PTF(R).  Therefore, to derive photometric offsets (which are time-independent), we use Sloan colors not corrected for extincton (and likewise, PS1, PTF, etc.) .  

In that way we form a 'master bandpass', consisting of SDSS bands, and PS1, PTF, CRTS equivalents.  Since all SDSS bands are observed nearly simultaneously, we choose SDSS(r) as the 'master band', and we transform photometry from all 'nearby' bandpasses in other surveys (PTF gR,  PS1 gri, CRTS V) - PS1 u,z,y are too distant from SDSS r. To separate in color space the stellar locus we use SDSS(g-i) color because it has a larger wavelength baseline than SDSS(g-r) color.  For each 'master band' light curve we keep track which points originated from SDSS(r),  PTF(gR), PS1(gri), or CRTS(V). 



%Correction for insterstellar extinction is only needed for color selection, since otherwise objects would appear to be of the later stellar type (redder) than they really are. To correct for extinction we use the most up-to-date maps of stellar extinction Bayestar17 (Green+2018). These extinction maps are 3D probabilistic, and we assume a uniform distance of 4 kpc for the dust column for all stars.  

% # could be more accurate with parallax distances from GAIA DR2 if available ...





\subsection{Simulated light curves}
We made a controlled experiment of long (20 tau) , well-sampled (dt=5 days) light curves,  with 400 points each . We used different priors (Jeff1, Jeff2, p1, p2, flat), and found that  sigma, tau from MAP (maximum a posteriori estimate) with Jeff1 is most consistent with Chelsea's code. We further investigated the logL evaluated on  a grid of sigma,tau, and conclude that the non-Gaussian shape of  the log-likelihood cauese the MAP to be biased, and the expectation value of marginalized posterior distribution is less biased  ( at 1\% level) . We find that the expectation value based sigma, tau are less biased, and for a very coarse grid 25x25 elements, the value of input parameters is  still recovered at the 1\% level (we expect the overall distribution to be biased on the 1\% level, so this is sufficient accuracy). 



We first check whether there is an improvement of fit for simulated DRW sampled at observed cadence - we plot $\tau_{out}$ vs $\tau_{in}$  for    SDSS sampling,    PS1+SDSS sampling,  PS1+CRTS+SDSS sampling,   PS1+CRTS+PTF+SDSS sampling.  This helps establish, based on simulated data (where we know the truth), whether we should expect much improvement in fit accuracy when using real data. 

We then perform fits using observed points selecting photometry only from a subset of surveys : $\tau_{PS1}$, $\tau_{(PS1+SDSS)}$,  $\tau_{SDSS}$.  We also check whether we get a better fit  behavior using only bright quasars with   $\tau_{\langle mag\rangle<19}$.

Using the best combination of survey data,  we revisit \cite{macleod2011} correlations of retrieved characteristic quasar timescale $\tau$ and variability amplitude $\sigma$ with black hole mass, luminosity, etc.  

\begin{figure}
\includegraphics[width=1.05\columnwidth]{figs/epochs_S82_QSO_compare.png}
%\vskip -0.15in
\caption{Count  of raw photometric measurements for Quasars in Stripe 82 from four surveys . Note that both CRTS and OTF significantly increase the original baseline of SDSS measurements.}
\label{fig:baselines}
\end{figure} 


%%%%%%%%%%%%%%%%%%%%%%%%%%%%%%%%%%%%%%%%%%%%%%%%%%

%%%%%%%%%%%%%%%%%%%% REFERENCES %%%%%%%%%%%%%%%%%%

% The best way to enter references is to use BibTeX:

\bibliographystyle{mnras}
\bibliography{references} % if your bibtex file is called references.bib

%%%%%%%%%%%%%%%%%%%%%%%%%%%%%%%%%%%%%%%%%%%%%%%%%%


% Don't change these lines
\bsp	% typesetting comment
\label{lastpage}
\end{document}

% End of mnras_template.tex
