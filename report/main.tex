%%
%% Beginning of file 'sample62.tex'
%%
%% Modified 2018 January
%%
%% This is a sample manuscript marked up using the
%% AASTeX v6.2 LaTeX 2e macros.
%%
%% AASTeX is now based on Alexey Vikhlinin's emulateapj.cls 
%% (Copyright 2000-2015).  See the classfile for details.

%% AASTeX requires revtex4-1.cls (http://publish.aps.org/revtex4/) and
%% other external packages (latexsym, graphicx, amssymb, longtable, and epsf).
%% All of these external packages should already be present in the modern TeX 
%% distributions.  If not they can also be obtained at www.ctan.org.

%% The first piece of markup in an AASTeX v6.x document is the \documentclass
%% command. LaTeX will ignore any data that comes before this command. The 
%% documentclass can take an optional argument to modify the output style.
%% The command below calls the preprint style  which will produce a tightly 
%% typeset, one-column, single-spaced document.  It is the default and thus
%% does not need to be explicitly stated.
%%
%%
%% using aastex version 6.2
\documentclass[twocolumn]{aastex62}
\usepackage{natbib}
\usepackage[T1]{fontenc}

\let\oldAA\AA
\renewcommand{\AA}{\text{\normalfont\oldAA}}



%% The default is a single spaced, 10 point font, single spaced article.
%% There are 5 other style options available via an optional argument. They
%% can be envoked like this:
%%
%% \documentclass[argument]{aastex62}
%% 
%% where the layout options are:
%%
%%  twocolumn   : two text columns, 10 point font, single spaced article.
%%                This is the most compact and represent the final published
%%                derived PDF copy of the accepted manuscript from the publisher
%%  manuscript  : one text column, 12 point font, double spaced article.
%%  preprint    : one text column, 12 point font, single spaced article.  
%%  preprint2   : two text columns, 12 point font, single spaced article.
%%  modern      : a stylish, single text column, 12 point font, article with
%% 		  wider left and right margins. This uses the Daniel
%% 		  Foreman-Mackey and David Hogg design.
%%  RNAAS       : Preferred style for Research Notes which are by design 
%%                lacking an abstract and brief. DO NOT use \begin{abstract}
%%                and \end{abstract} with this style.
%%
%% Note that you can submit to the AAS Journals in any of these 6 styles.
%%
%% There are other optional arguments one can envoke to allow other stylistic
%% actions. The available options are:
%%
%%  astrosymb    : Loads Astrosymb font and define \astrocommands. 
%%  tighten      : Makes baselineskip slightly smaller, only works with 
%%                 the twocolumn substyle.
%%  times        : uses times font instead of the default
%%  linenumbers  : turn on lineno package.
%%  trackchanges : required to see the revision mark up and print its output
%%  longauthor   : Do not use the more compressed footnote style (default) for 
%%                 the author/collaboration/affiliations. Instead print all
%%                 affiliation information after each name. Creates a much
%%                 long author list but may be desirable for short author papers
%%
%% these can be used in any combination, e.g.
%%
%% \documentclass[twocolumn,linenumbers,trackchanges]{aastex62}
%%
%% AASTeX v6.* now includes \hyperref support. While we have built in specific
%% defaults into the classfile you can manually override them with the
%% \hypersetup command. For example,
%%
%%\hypersetup{linkcolor=red,citecolor=green,filecolor=cyan,urlcolor=magenta}
%%
%% will change the color of the internal links to red, the links to the
%% bibliography to green, the file links to cyan, and the external links to
%% magenta. Additional information on \hyperref options can be found here:
%% https://www.tug.org/applications/hyperref/manual.html#x1-40003
%%
%% If you want to create your own macros, you can do so
%% using \newcommand. Your macros should appear before
%% the \begin{document} command.
%%
\newcommand{\vdag}{(v)^\dagger}
\newcommand\aastex{AAS\TeX}
\newcommand\latex{La\TeX}

%% Reintroduced the \received and \accepted commands from AASTeX v5.2
\received{January 1, 2019}
\revised{January 17, 2019}
\accepted{February 1, 2019}%\today}
%% Command to document which AAS Journal the manuscript was submitted to.
%% Adds "Submitted to " the arguement.
\submitjournal{ApJ}

%% Mark up commands to limit the number of authors on the front page.
%% Note that in AASTeX v6.2 a \collaboration call (see below) counts as
%% an author in this case.
%
%\AuthorCollaborationLimit=3
%
%% Will only show Schwarz, Muench and "the AAS Journals Data Scientist 
%% collaboration" on the front page of this example manuscript.
%%
%% Note that all of the author will be shown in the published article.
%% This feature is meant to be used prior to acceptance to make the
%% front end of a long author article more manageable. Please do not use
%% this functionality for manuscripts with less than 20 authors. Conversely,
%% please do use this when the number of authors exceeds 40.
%%
%% Use \allauthors at the manuscript end to show the full author list.
%% This command should only be used with \AuthorCollaborationLimit is used.

%% The following command can be used to set the latex table counters.  It
%% is needed in this document because it uses a mix of latex tabular and
%% AASTeX deluxetables.  In general it should not be needed.
%\setcounter{table}{1}

%%%%%%%%%%%%%%%%%%%%%%%%%%%%%%%%%%%%%%%%%%%%%%%%%%%%%%%%%%%%%%%%%%%%%%%%%%%%%%%%
%%
%% The following section outlines numerous optional output that
%% can be displayed in the front matter or as running meta-data.
%%
%% If you wish, you may supply running head information, although
%% this information may be modified by the editorial offices.
\shorttitle{Improved DRW for S82 QSO}
\shortauthors{Suberlak et al.}
%%
%% You can add a light gray and diagonal water-mark to the first page 
%% with this command:
% \watermark{text}
%% where "text", e.g. DRAFT, is the text to appear.  If the text is 
%% long you can control the water-mark size with:
%  \setwatermarkfontsize{dimension}
%% where dimension is any recognized LaTeX dimension, e.g. pt, in, etc.
%%
%%%%%%%%%%%%%%%%%%%%%%%%%%%%%%%%%%%%%%%%%%%%%%%%%%%%%%%%%%%%%%%%%%%%%%%%%%%%%%%%

%% This is the end of the preamble.  Indicate the beginning of the
%% manuscript itself with \begin{document}.

% from Foreman2017  preamble : 
\newcommand{\project}[1]{\textsf{#1}}


\begin{document}

\title{Improving Damped Random Walk parameters for SDSS Stripe82 Quasars with Pan-STARRS1. }


%% LaTeX will automatically break titles if they run longer than
%% one line. However, you may use \\ to force a line break if
%% you desire. In v6.2 you can include a footnote in the title.

%% A significant change from earlier AASTEX versions is in the structure for 
%% calling author and affilations. The change was necessary to implement 
%% autoindexing of affilations which prior was a manual process that could 
%% easily be tedious in large author manuscripts.
%%
%% The \author command is the same as before except it now takes an optional
%% arguement which is the 16 digit ORCID. The syntax is:
%% \author[xxxx-xxxx-xxxx-xxxx]{Author Name}
%%
%% This will hyperlink the author name to the author's ORCID page. Note that
%% during compilation, LaTeX will do some limited checking of the format of
%% the ID to make sure it is valid.
%%
%% Use \affiliation for affiliation information. The old \affil is now aliased
%% to \affiliation. AASTeX v6.2 will automatically index these in the header.
%% When a duplicate is found its index will be the same as its previous entry.
%%
%% Note that \altaffilmark and \altaffiltext have been removed and thus 
%% can not be used to document secondary affiliations. If they are used latex
%% will issue a specific error message and quit. Please use multiple 
%% \affiliation calls for to document more than one affiliation.
%%
%% The new \altaffiliation can be used to indicate some secondary information
%% such as fellowships. This command produces a non-numeric footnote that is
%% set away from the numeric \affiliation footnotes.  NOTE that if an
%% \altaffiliation command is used it must come BEFORE the \affiliation call,
%% right after the \author command, in order to place the footnotes in
%% the proper location.
%%
%% Use \email to set provide email addresses. Each \email will appear on its
%% own line so you can put multiple email address in one \email call. A new
%% \correspondingauthor command is available in V6.2 to identify the
%% corresponding author of the manuscript. It is the author's responsibility
%% to make sure this name is also in the author list.
%%
%% While authors can be grouped inside the same \author and \affiliation
%% commands it is better to have a single author for each. This allows for
%% one to exploit all the new benefits and should make book-keeping easier.
%%
%% If done correctly the peer review system will be able to
%% automatically put the author and affiliation information from the manuscript
%% and save the corresponding author the trouble of entering it by hand.



\correspondingauthor{Krzysztof Suberlak}
\email{suberlak@uw.edu}

\author[0000-0002-9589-1306]{Krzysztof L. Suberlak}
\affiliation{Department of Astronomy \\ University of Washington \\ Seattle, WA 98195, USA}


\author{\v{Z}eljko Ivezi\'c}
\affiliation{Department of Astronomy \\ University of Washington \\ Seattle, WA 98195, USA}



\author{Chelsea MacLeod}
\affiliation{Harvard Smithsonian Center for Astrophysics \\ 60 Garden St, Cambridge, MA 02138, USA}




% Abstract of the paper
\begin{abstract}

We use the Panoramic Survey Telescope and Rapid Response System 1 Survey (Pan-STARRS1, PS1) data to extend  the Sloan Digital Sky Survey (SDSS) Stripe 82 quasar light curves. Combining PS1 gri and SDSS r light curves provides 15 years baseline for 9248 quasars, improving on previous studies using SDSS data only.  We fit the light curves with Damped Random Walk (DRW) model, and correlate the DRW parameters -  asymptotic variability amplitude $SF_{\infty}$, and characteristic timescale $\tau$, with quasar physical properties - black hole mass, bolometric luminosity, and redshift. We find that compared to other studies the characteristic timescale $\tau$ is more strongly dependent on quasar luminosity, and has a weaker dependence on the black hole mass. The variability amplitude is less strongly dependent on the quasar luminosity. We make predictions on fidelity of DRW retrieval once ZTF and LSST data become available. 
\end{abstract}


%%%%%%%%%%%%%%%%% BODY %%%%%%%%%%%%%%%%%%%%%%%%%%%%%%%%%

\section{Introduction}

Quasars are variable at the rms level of 0.2 mag. They are distant active galactic nuclei, harboring a supermassive black hole surrounded by a hot accretion disk. Although it is agreed upon that the thermal emission from the accretion disk is the source of the majority of optical light, the detailed origin of variability has been debated for the past 50 years (\citealt{sun2018} and references therein). Some favor a thermal origin of variability \citep{kelly2013}, related to the propagation of inhomogeneities ("hot spots") in the disk \citep{dexter2011, cai2016}, others suggested magnetically elevated disks \citep{dexter2019}, or X-ray reprocessing  \citep{kubota2018}.  Indeed, it may well be that the answer involves combination of these -  as \cite{sanchez2018} suggests, perhaps short-term variability (hours-days) is linked to the changes in X-ray flux, while long-term variability (months-years) is more intrinsic to the disk \citep{edelson2015,lira2015}. Nevertheless, quasar light curves have been successfully described using the Damped Random Walk (DRW) model \citep{kelly2009, macleod2010, kozlowski2010, zu2011, kasliwal2015a}, and the DRW parameters have been linked to the physical quasar properties (\citealt{macleod2010} , hereafter M10). 

Variability is also a classification tool, allowing to distinguish quasars from other variable sources that do not exhibit a stochastic variability pattern \citep{macleod2011}. This property is especially useful for selecting quasars in the intermediate redshift range, which overlaps the stellar locus in color-color diagrams \citep{sesar2007, yang2017}). Variability has also been used to increase the completeness in measurements of Quasar Luminosity Function (see \citealt{ ross2013, palanque2013, alsayyad2016, mcgreer2013, mcgreer2018}). 

Due to its stochastic nature, the DRW process requires the light curve to be several times longer than the characteristic timescale for an unbiased parameter retrieval (\citealt{kozlowski2010, kozlowski2017a}, hereafter K17). For this reason, some studies have restricted the probed redshift range, limiting the quasar sample to where one would expect only shorter timescales based on previous studies \citep{sun2018, guo2017}, some have elected not to study timescales at all \citep{sun2018,sanchez2018}, while others have suffered from short-baseline biases by employing single-survey data \citep{hernitschek2016}. 

In this study by extending available quasar light curves we are able to better recover a wider range of DRW timescales, and probe a wider range of redshifts and black hole masses. Almost a decade ago M10 published their research using SDSS Stripe 82 data, and new datasets (PS1,PTF,CRTS) have become available since, that can extend the quasar light curves by almost 50\% .  Indeed, \citet{li2018} combined SDSS and Dark Energy Camera Legacy Survey (DECaLS) data, to provide a 15 year baseline, but by focusing on a large area to encompass as many quasars as possible (119,305 up to z=4.89) suffered from poor sampling which lends itself better to an ensemble structure function approach rather than direct light curve modeling. On the other hand, \citet{sanchez2018} who employed optical data from QUEST-La Silla AGN variability survey, also used the Structure Function parametrization (amplitude of variability and the excess variance), because the light curve length (less than 5 years for 2435 quasars)  excluded the possibility of unbiased retrieval of characteristic timescale. A different parameter space of cadence and baseline was explored by Kepler-based studies, with extremely well-sampled light curves (30 min cadence) of a small number of quasars, with baselines of up to 4 years \citep{mushotzky2011, edelson2014, kasliwal2015a, kasliwal2017, smith2018, aranzana2018}. Finally other studies suffered from restricted quasar samples to avoid aforementioned biases \citep{guo2017, sun2018}, or small number of objects satisfying their cuts \citep{kelly2009,kelly2013,simm2016}.  There were also those who studied quasar variability with the ensemble structure function approach, which works well if only a small number of epochs is available per object, but it lacks the appeal of a direct object-by-object modeling \citep{bauer2009, macleod2012, caplar2017, kozlowski2016c}


 Unlike previous studies, in this work, by combining SDSS and PS1 data we afford both an extended baseline (15 years), and a large number (9000) of quasars to which we fit the DRW model. First we confirm in Section~\ref{sec:methods} that extending the quasar baseline is the most important improvement in providing unbiased estimates of the DRW model parameters (K17). In Section~\ref{sec:data} we describe the datasets employed, and their combination onto a common photometric system. In Section~\ref{sec:simulation} we test the improvement in DRW parameters with simulation. Finally in Section~\ref{sec:results} we describe the main results analyzing correlations between physical parameters and variability, and in Section~\ref{sec:conclusions} we summarize the main results. In this work we adopt a $\Lambda $CDM cosmology with $h_{0} = 0.7$ and $\Omega_{m} = 0.3 $. 


%
%
\section{Methods}\label{sec:methods}
\subsection{DRW as a Gaussian Process}
Damped Random Walk (Ornstein-Uhlenbeck process) can be modeled as a member of a class of Gaussian Processes (GP). Each GP is described by a mean and a kernel - a covariance function that contains a measure of correlation between two points: $x_{n}$ and $x_{m}$, separated by $\Delta t_{nm}$ (autocorrelation). For the  DRW process, the covariance  between two observations spaced by  $\Delta t_{nm}$ is: 

\begin{eqnarray}
k(\Delta t_{nm}) &=& a \exp{(-\Delta t_{tm} / \tau)} \\
                 &=& \sigma^{2}\exp{(-\Delta t_{tm} / \tau)}  \nonumber \\
                 &=& \sigma^{2} ACF(\Delta t_{tm})\nonumber
\end{eqnarray} 

Here $a$ or $\sigma^{2}$ is an amplitude of correlation decay as a function of $\Delta t_{tm}$,  while $\tau$ is the characteristic timescale over which correlation drops by $1/e$. For a DRW,  the correlation function $k(\Delta t_{nm})$ is also related to the autocorrelation function $ACF$. 

Not explicitly used in this paper, but of direct relevance to the DRW modeling, is the structure function (SF). SF can be found from the data as the root-mean-squared of  magnitude differences $\Delta m$  calculated as a function of temporal separation $\Delta t$ (we drop subscripts $n$,$m$ for brevity). SF is directly related to a DRW kernel $k(\Delta t)$:

\begin{equation}
SF(\Delta t) = SF_{\infty} (1-\exp{(-|\Delta t|/\tau)})^{1/2}
\end{equation}

For quasars SF follows approximately a power law: $SF \propto \Delta t^{\beta}$,  and it levels out for large time lags $\Delta t$ to a constant value of $SF_{\infty}$.   Note that $SF_\infty = \sqrt{2} \sigma$  in the above (also see \citet{macleod2012, bauer2009, graham2015a} for an overview).


The likelihood for the particular value of DRW parameters given the data is evaluated with   \project{celerite} \citep{foreman2017} - a fast GP solver that scales linearly with the number of data points $\mathcal{O}(N)$ thanks to an optimization that exploits the structure of covariance matrix for kernels that are a mixture of exponentials, such as a DRW kernel \citep{foreman2018,ambikasaran2015}. The method employed is similar to that used by \cite{rybicki1992, kozlowski2010},  M10 - like in previous work, we use a  uniform prior in log space on DRW fit parameters. The main difference in our approach is that rather than adopting the Maximum A-Posteriori (MAP) as the 'best-fit' value for the DRW parameters (as done in \citealt{kozlowski2010}, K17, \citealt{kozlowski2016b}, M10, \citealt{macleod2011}),  we find the expectation value of the marginalized log posterior. This is advantageous because of a non-Gaussian shape of the log posterior - it if were described by a 2D Gaussian,  then the expectation value would coincide with the maximum of the log posterior.

\subsection{The impact of light curve baseline}\label{sec:baseline}

K17 reports that one cannot trust any results of DRW fitting unless the light curve length is at least ten times longer than the characteristic timescale. We confirm these generic trends, although we find that the stringent requirements of K17 can be somewhat relaxed. Following K17 setup,  we  model 10 000  DRW light curves with fixed length (baseline) $t_{exp}=8$ years, asymptotic variability amplitude of  $\mathrm{SF}_{\infty} = 0.2$ mag, SDSS or OGLE-like cadence, sampling over a range of input timescales. With fixed baseline, this spans the parameter space of $\rho = \tau / t_{exp}$, $\rho \in   \{ 0.01 : 15\}$. We simulate 100 light curves at each $\rho$. 

\begin{figure*}  % code/PLOT_Fig2_George_Celerite.ipynb
\plottwo{figs/celerite_SDSS_Jeff1_expectation-190208_results_celerite_R.png}{figs/celerite_OGLE_Jeff1_expectation-190208_results_celerite_R.png}
\caption{Probing the parameter space of $\rho = \tau / t_{exp}$, with a simulation of  10 000 light curves : 100 light curves per each of 100 $\rho$ values spaced uniformly in logarithmic space between $\rho \in   \{ 0.01 : 15\}$ . With a baseline $t_{exp}$ set to 8 years,  we sample a range of 100 input timescales. Left panel shows the SDSS-like cadence with N=60 epochs, and the right panel the OGLE-like cadence with N=445 epochs. The dotted horizontal and solid vertical lines represent $\rho = 0.1$, i.e. the baseline is ten times longer than considered timescale. The diagonal line is $y=x$, i.e. the line that would be followed if the recovered  $\rho$ ($\tau$) was exactly the same as the input $\rho$ ($\tau$). Given a quasar light curve, which has one true underlying DRW timescale,  as we extend the baseline, we move from the bottom-left (unconstrained) to the top-right (well-constrained) part of the parameter space. } 
\label{fig:rho_space}
\end{figure*}


The true underlying DRW signal $s(t)$ is found by iterating over the array of time steps $t$.  At each step, we draw a point from a Gaussian distribution, for which the mean and standard deviation are re-calculated at each timestep (see \citealt{kelly2009} (eqs. A4 and A5) as well as in \cite{macleod2010} (Sec. 2.2 ), and K17). Starting at $t_{0}$, the signal is equal to the mean magnitude, $s_{0} = \langle m \rangle$. After a timestep $\Delta t_{i} = t_{i+1} - t_{i}$, the signal $s_{i+1}$ is drawn from  $\mathcal{N}(loc, stdev)$, with : 

\begin{equation}
loc = s_{i} e ^ { - r  }  + \langle m \rangle \left( 1 - e ^{ - r }\right)
\end{equation}

and 

\begin{equation}
stdev^{2} =  0.5  \, \mathrm{SF}_{\infty}^{2} \left( 1 - e ^{  - 2 r  }  \right)
\end{equation}

where  $r = \Delta t_{i} / \tau$, and $\tau$ is the damping timescale.


To simulate observational conditions  we add to the true underlying signal  $s(t)$ a noise offset, $n(t)$.  Like K17,  we assume $n(t)$ to be drawn from a Gaussian distribution $\mathcal{N}(0,\sigma(t))$ with a width $\sigma(t)$, corresponding to the  photometric uncertainty at the given epoch: 

\begin{equation}
y(t) = s(t) + n(t) 
\end{equation}



We adopt SDSS S82-like cadence with N=60 epochs, or OGLE-III like cadence with N=445 epochs.  The errors are set by the adopted mean magnitudes, $r=17$ and $I=18$ :

\begin{eqnarray}
\sigma_{SDSS}^{2} &=& 0.013^{2} + \exp{(2 (r-23.36))} \\
\sigma_{OGLE}^{2} &=& 0.004^{2} + \exp{(1.63 (I - 22.55))}
\end{eqnarray}



Fig.~\ref{fig:rho_space} shows the recovered $\rho_{out}$ as a function of input $\rho_{in}$. This confirms the findings of K17: the recovered $\rho$ becomes meaningless ('unconstrained') if the available DRW light curve baseline is not at least several times longer than the input timescale ($\log_{10}{(\rho)} \lessapprox -0.5$, i.e. $\rho \lessapprox 3 $). Consequently, by extending the baseline we can move from the biased region (bottom left) to the unbiased regime (top right). This is the basis for this study,  in which we extend the baselines of quasar light curves from SDSS-only (10 years) to combined SDSS-PS1 (15 years). 

\subsection{Departure from DRW?}

The power spectral density (PSD) of the DRW process is:

\begin{equation}
P(f) = \frac{4\sigma^{2}\tau}{1+(2 \pi \tau f)^{2}}
\end{equation}
(with $\sigma = \mathrm{SF}_{\infty} / \sqrt{2}$, $\tau$ the characteristic timescale, $f$ the frequency), so that $P(f) \propto f^{-2}$  at high frequencies $f > (2\pi \tau)^{-1}$, and levels to a constant value at lower frequencies\citep{kelly2014}. This can be parametrized as $\log{(P(f))} \propto \alpha_h \log{(f)}$ at high frequencies (short timescales), and  $\log{(P(f))} \propto \alpha_l \log{(f)}$  at low frequencies (long timescales), so that for a pure DRW process $\alpha_{h}=-2$ and $\alpha_{l} = 0$.  

Various answers exist about the exact value of  $\alpha_{l}$ and $\alpha_{h}$ for accurate description of quasar variability,  or more broadly speaking -  AGN variability. 

One set of answers come from studying optical light curves of wide-field, ground-based photometric surveys, such as OGLE, SDSS, or PS1. In general, these studies used  light curves with rather sparse non-uniform sampling, cadence of a few epochs per week,  and overall length (baseline) of the order of several years.  
For instance, \citet{zu2013}  using 223 $I$-band  OGLE quasar  light curves (baseline of  ${\sim}7$ years, ${\sim}570$ epochs) considered whether different covariance functions (powered-exponential, Mat\`ern, Kepler-exponential, Pareto exponential - see Eqs.5-9 in \citealt{zu2013}) may present a better fit, and found that while there may be small deviations, they are not significant enough to depart from a robust DRW description.  Similarly, \citet{sun2018} found that the data quality of SDSS-like light curves was insufficient to distinguish between CAR(1) and more complex models.   For the longer timescales, \cite{guo2017} with a low-redshift subset of SDSS S82 quasars  claimed  that the low-frequency slope  $\alpha_{l}$ should not be steeper than $-1.3$. For the same SDSS S82 quasars,  K17 (as well as \citealt{kozlowski2016b} and \citealt{caplar2017}) found PSD slopes $\alpha_{h} \leqslant -2$, which meant that models assuming a DRW $\alpha_{h}=2$ would result in decorrelation timescales biased low. He stated that it is currently not possible to distinguish between close-to-DRW and DRW processes for the S82 quasars because good sampling at both low frequencies (long timescales, white noise part), and high frequencies (short timescales, red / pink noise part) is not available. However, \cite{simm2016} used a limited PS1 sample of 90 X-ray selected AGN, and with PS1 data found that their data can be described by a broken power law with low-frequency slope  of $-1$ and high frequency slope from $-2$ to $-4$, with a break at timescale between $~200$ to $~300$ days.   

A second set of answers comes from studying data with excellent cadence but short baselines from a space based Kepler mission \citep{borucki2010}. Using Kepler light curves   with a half-hour cadence enabled an unprecedented view of the high-frequency part of the AGN spectrum, but the results are inconclusive. \cite{mushotzky2011}  analyzed the four AGN light curves from 2010-2011 (three separate quarters), and found evidence of steeper slopes ($\alpha_{h} {\sim} -2.6 $ to $ -3.1$).  Later,  \cite{edelson2014} combined $3.4$ yrs worth of data ($13$ quarters)  for an AGN Zw 229-15, and found the slope increasing from $\alpha_{h} {\sim} -2 $ to $ -4 $ at frequency corresponding to $\Delta t = 5$ days. \cite{smith2018} used 4 years of Kepler data for 21 AGN,  and found slopes generally steeper than the DRW, between $\alpha_{h} {\sim} -2$ and $-3.4$ . They also concluded that, in accordance with \cite{caplar2017}, perhaps AGN are described by a combination of DRW behavior, and a changing PSD, tied to an accretion duty cycle. With the original  Kepler  mission ending in   2013 with the failure of the second reaction wheel,  it transitioned to K2, pointing at various fields around the ecliptic, which limits further the available baseline for AGN study.  \citet{aranzana2018} conducted the most extensive study of K2 AGN light curves to date, including 252 well-sampled AGN. Using data spanning 80 days they find a range of high-frequency slopes from $\alpha_{h} {\sim} -1$ to $-3.2$, with the median of $\alpha_{h} {\sim} -2.2$,  consistent with the DRW model.

% may also mention here  Caplar2017 and his PTF studies ? 


% perhaps skip, about the impact of PSD shape on the RM studies  .... 
The exact shape of quasar PSD would affect other areas of study. One example is reverberation mapping (RM), which was used to provide  accurate black hole mass estimates. RM is based on measuring the lags between light curves observed at different wavelengths. The two most widely used approaches to measuring interband lags are interpolated cross-correlation function (ICCF, \citealt{gaskell1987, peterson2004})  and  light curve modeling via Markov chain Monte Carlo approach (JAVELIN, \citealt{zu2011}). The PSD of the DRW comes explicitly in the latter,  which assumes that the higher energies drive lower energies (whether due to x-ray reprocessing or other mechanism), and that the driving light curve is well-modeled by a DRW (with PSD equal to  $-2$ or flatter).  In this scenario other light curves (at longer wavelength) are related to it via transfer function \cite{edelson2019}. Using a wrong PSD means at best that errors are underestimated. Therefore a convincing detection of a departure of quasar PSD from that of the DRW  would have  a direct impact on RM studies that use  JAVELIN or other light curve modeling tools. 


In the end, we follow the direction laid out in M10, Sec.4.4, which shows that while within S82 sampling assuming $\alpha_{h}= 2 $ it is impossible to reliably distinguish between a $\alpha_{l}=0$ and $-1$. A future study with appropriate cadence would be necessary to model each quasar individually with different $\alpha_{l}$ and $\alpha_{h}$. We elect to use the DRW description ($\alpha_{l}=0$, $\alpha_{h}=-2$) to allow a better comparison of our results with M10. 

%
%
%
%
%

\section{Data}\label{sec:data}
\subsection{Surveys}
We focus on the data pertaining to a 290 deg$^{2}$ region of southern sky, repeatedly observed by the Sloan Digital Sky Survey (SDSS) between 1998 and 2008. Originally aimed at supernova discovery, objects in this area, known as Stripe82 (S82), were  re-observed on average 60 times (see \citealt{macleod2012}, Sec. 2.2 for overview, and \citealt{annis2014} for details). Availability of well-calibrated \citep{ivezic2007}, long-baseline light curves spurred variability research (see \citealt{sesar2007}). The catalog prepared by \citet{schneider2008} as part of DR9  contains 9258 spectroscopically confirmed quasars.  


\begin{figure*} % code2/Stats_simulation.ipynb 
\plotone{figs/lightcurves_extent.png}
\caption{An illustration of survey baseline, sky area covered, and depth. The width of each rectangle corresponds to the extent of light curves available (or simulated) for Stripe 82 quasars for each survey. For SDSS this means DR7; for CRTS DR2, PS1 DR2, PTF DR2, ZTF year 2018, and LSST the full 10-year survey. The lower edge of each rectangle corresponds to the $5\sigma$ limiting magnitude (SDSS r, PS1 r, PTF R, ZTF r, LSST r, CRTS V). The vertical extent corresponds to the total survey area (for SDSS, up to and including DR15).  Note how PS1 and PTF extend the baseline of SDSS by approximately $50\%$, and how inclusion of LSST triples the SDSS baseline. For reference, the area covered by LSST is $25 000$ sq.deg., which corresponds to  $60\%$ of the sky. The whole sky has an area of $4\pi$ steradians (41253 sq.deg.).}
\label{fig:lcExtent}
\end{figure*} 


We extend SDSS  light curves with PanSTARRS (PS1) \citep{chambers2011,flewelling2018}, Catalina Real-Time Transient Survey (CRTS) \citep{drake2009}, and Palomar Transient Factory (PTF) \citep{rau2009} data. Of 9258  SDSS quasars, within $0.5 ''$ there are 9248 PS1 matches, 6455 PTF matches, and 7737 CRTS matches. Of these, 6444 quasars have coverage in all surveys (SDSS-PS1-PTF-CRTS).  Fig~\ref{fig:lcExtent}  depicts the  baseline coverage of various surveys.  Each survey uses a unique set of bandpasses and cadences: SDSS light curves contain near-simultaneous $\{u,g,r,i,z\}_{\mathrm{SDSS}}$, and PS1, PTF observations in different filters are  non-simultaneous.%: $\{g,r,i,z,y\}_{\mathrm{PS1}}$,  $\{g,R\}_{\mathrm{PTF}}$, $V_{\mathrm{CRTS}}$.  

\subsection{Photometric offsets}

We combine quasar photometry into a single 'master' bandpass.  We choose  $r_{\mathrm{SDSS}}$ as the target bandpass since it has the best photometry, and we calculate color terms that afford transformation from other photometric systems to SDSS. We translate to$r_{\mathrm{SDSS}}$ only photometry from nearby filters: $\{g,R\}_{\mathrm{PTF}}$, $\{g,r,i\}_{\mathrm{PS1}}$, $V_{\mathrm{CRTS}}$. 

Color terms are derived using the SDSS standard stars catalog~\citep{ivezic2007}.  We focus on a 10\% subset of randomly chosen stars from the catalog, and find CRTS (B.Sesar, priv.comm.), PS1 (from MAST \url{http://panstarrs.stsci.edu}) and PTF (IRSA PTF Object Catalog \url{https://irsa.ipac.caltech.edu/}) matches. 

We consider the difference  between the target (SDSS) and source (eg.PS1) photometry as a function of the mean SDSS $(g-i)$ color : 


\begin{equation}
m_{\mathrm{PS1}} - m_{\mathrm{SDSS}} = f(g-i)
\end{equation}

While \cite{tonry2012} chose to spread the stellar locus with $(g-r)$ color, we prefer to use the $(g-i)$ color since it provides a larger wavelength baseline. Some authors (eg. \citet{li2018}) allow the transformation to be a higher-order polynomial, but since we are limited to the narrow region of $(g-i)$ space occupied by quasars, before the stellar locus bend (Fig.~\ref{fig:quasarColors}), we find that the linear fit is sufficient (see Fig.~\ref{fig:offsetsPS1}). We summarize the resulting offsets to SDSS in Table~\ref{tab:offsets}.


% QSO COLORS 
\begin{figure*} % code/AA_quasar_colors.ipynb 
\plotone{figs/SDSS_S82_CMD_qso_stars_2.png}
\caption{Regions of color-color (upper left, upper right, bottom left), and color-magnitude (bottom right)  space occupied by SDSS S82 quasars (color) and stars (contours). We use quasar median photometry from \citet{schneider2010}, and standard stars catalog of \citet{ivezic2007}, showing a random subset of 10 000 stars. We find the SDSS-PS1 color terms using the region of SDSS color space that best represents quasars, that are generally bluer than the stellar locus:  $-0.5<(g-i)<1$. Quasars also overlap other variable sources (eg. RR Lyrae), not shown here \citep{sesar2007}. }
\label{fig:quasarColors}
\end{figure*} 


% PS1 
\begin{figure*}% code/AC_SDSS_PS1_offsets.ipynb 
\plotone{figs/Offsets_PS1-SDSSr_SDSSgi_ext-NO.png}
\caption{The SDSS-PS1 offsets, derived from the SDSS standard stars catalog \citep{ivezic2007}. From randomly chosen subset of 50 000 stars, we selected only 23 000 that are sufficiently bright ($r_{\mathrm{SDSS}} < 19$) to minimize scatter due to photometric uncertainty. On each panel we plot 20 000 SDSS stars that have PS1 match within 0.1 arcsec. Vertical dashed lines mark the region in the SDSS color space occupied by quasars (see Fig.~\ref{fig:quasarColors}), used to fit the stellar locus with a first order polynomial, marked by the solid red line.}
\label{fig:offsetsPS1}
\end{figure*} 



\begin{deluxetable}{c|cc}
%\tablewidth{\textwidth}
\tablecaption{Color terms (offsets) between CRTS, PTF, PS1 passbands used in combined light curves, and  SDSS, using the mean $(g-i)_{\mathrm{SDSS}}$ color to spread the stellar locus. Thus the $r_{\mathrm{SDSS}}$ synthetic magnitude ($r_{s}$ for short) can be found as $r_{s} = x-B_{0}-B_{1}(g-i)_{\mathrm{SDSS}}$. This linear trend is illustrated on Fig.\ref{fig:offsetsPS1}, where we plot $(x-r_{\mathrm{SDSS}})$ as a function of $(g-i)_{\mathrm{SDSS}}$ for $x=g_{\mathrm{P1}}, r_{\mathrm{P1}},i_{\mathrm{P1}}$.\label{tab:offsets}}  

\tablehead{\colhead{Band (x)} & \colhead{$B_{0}$} & \colhead{$B_{1}$}}
\startdata
CRTS V & -0.0464  & -0.0128 \\
PTF g &  -0.0294  &  0.6404 \\
PTF R &  0.0058   & -0.1019 \\
PS1 g &  0.0194   &  0.6207 \\
PS1 r &  0.0057   & -0.0014 \\
PS1 i &  0.0247   & -0.2765 \\
\enddata

\tablecomments{To derive the color terms we used SDSS S82 standard stars catalog \citep{ivezic2007}. We randomly selected 10\% of that catalog, for which 48250 have CRTS light curves. We  obtained PS1 photometry from MAST, and PTF from IRSA PTF Object Catalog. We imposed quality cuts requiring that the stars are bright: $r < 19$.}
\end{deluxetable}




%NOTE ABOUT EXTINCTION:  due to dust present between us and the standard stars (or background quasars), the observed light will appear slightly redder because dust preferentially scatters blue light away. This depends on the location of the source on the  sky and is related to the dust inhomogeneities in the Milky Way. 

%However, in deriving bandpass to bandpass transformation all that matters is the flux received at the Top of the Atmosphere (TOA), for which all photometry is corrected at the pipeline processing level (both for SDSS,  PTF, PS1, and CRTS). In other words, all that matters is that any SED at TOA will have slightly different magnitude (hence colors) depending on whether we observe it with SDSS(r),  PS1(r), or PTF(R).  Therefore, to derive photometric offsets (which are time-independent), we use Sloan colors not corrected for extincton (and likewise, PS1, PTF, etc.) .  

%Correction for insterstellar extinction is only needed for color selection, since otherwise objects would appear to be of the later stellar type (redder) than they really are. To correct for extinction we use the most up-to-date maps of stellar extinction Bayestar17 (Green+2018). These extinction maps are 3D probabilistic, and we assume a uniform distance of 4 kpc for the dust column for all stars.  

%
%
%
%
%

\section{Simulations : lessons learned}\label{sec:simulation}

We simulate the theoretical improvement of the DRW parameter retrieval in extended light curves. We generate long and well-sampled light curves, all with input $\tau = 575 $ days, $SF_{\infty} = 0.2$ mag (the median of S82 distribution in M10). Then we subsample at SDSS, SDSS-PS1, and predicted ZTF and LSST cadence (see Fig.~\ref{fig:simLC}). For the LSST 10-year segment (final LSST DR10 in 2031) we assumed 50 epochs per year, randomly distributed throughout the year, with the following error model:

\begin{eqnarray}
\label{eq:errorModel}
\sigma_{LSST}(m)^{2} &=& \sigma_{sys}^{2} + \sigma_{rand}^{2} \,\, \mathrm{(mag)}^{2} \\
\sigma_{rand}^{2} &=& (0.04-\gamma)x + \gamma x^{2} \nonumber \\
x &=& 10^{0.4(m-m_{5})} \nonumber
\end{eqnarray}

with  $\sigma_{sys} = 0.005$, $\gamma=0.039$, $m_{5} = 24.7$ (see \citet{ivezic2019}, Sec.3.2).
For the ZTF 1-year segment (Spring 2019 ZTF DR1 includes data from 2018) we assumed 120 observations (every three nights) in $g_{\mathrm{ZTF}}$ and $r_{\mathrm{ZTF}}$, deriving the magnitude-dependent error model by plotting for ZTF standard stars on Fig.~\ref{fig:ztf_errors} best mag rms as a function of best median magnitude. We find that the LSST error model (Eq.~\ref{eq:errorModel}) with $\gamma = 0.05$, $\sigma_{sys} = 0.05 $, and $m_{5} = 20.8$ adequately describes the ZTF photometric uncertainty. 

\begin{figure}%[ht!] % code2/Subsample_master_LC.ipynb
\plotone{figs/Simulated_DRW-0008_sampled-3537034_fit.png}
\caption{Simulated underlying DRW process shown here in black ($\tau=575$d, SF$_{\infty} = 0.2$ mag, 4 points per day) subsampled at the real cadence of SDSS (red), PS1 (green) segments, and simulated LSST (blue) epochs, using time stamps from the combined SDSS-PS1 light curve for quasar dbID=3537034. The orange 'error snake' is an envelope marking the standard deviation of the fit to the data using a Gaussian process  with DRW kernel (Sec.~\ref{sec:simulation}).}
\label{fig:simLC}
\end{figure} 

\begin{figure}%[ht!]  % code2/Simulate_ZTF_cadence_error.ipynb
\plotone{figs/ZTF_error_curve.png}
\caption{The best mag rms plotted as a function of magnitude for ZTF non-variable stars with over 100 observations. We overplot the adopted error model, with $\gamma = 0.05$, $\sigma_{sys} = 0.05 $, and $m_{5} = 20.8$ (after \citealt{ivezic2019}). Properties of ZTF photometric uncertainties are largely similar to the PTF uncertainties.}
\label{fig:ztf_errors}
\end{figure} 


To mirror observational conditions we add to the true underlying DRW signal a Gaussian noise, with variance defined by photometric uncertainties for corresponding surveys. Given the noise properties of each survey (Fig.~\ref{fig:combinedLCerrors}), we found that relatively large uncertainties of CRTS and PTF segments introduced less improvement in recovery of DRW parameters, given that similar baseline is already covered by the PS1 data (Fig.~\ref{fig:lcExtent}).  We further found that inclusion of ZTF data for 2018 would not significantly change our results. Inclusion of PS1 data with its excellent photometry (as compared to ZTF or PTF) is the best improvement over existing SDSS results.  In the future (after more data has been assembled and re-calibrated) ZTF will help, but not as dramatically as LSST (see Figs.~\ref{fig:simLCresults1} and ~\ref{fig:simLCresults2}).  For this reason we found that using only SDSS-PS1 portion is the best tradeoff between adding more baseline vs introducing more uncertainty with noisy data.

\begin{figure}%[ht!]  %code2/Subsample_master_LC.ipynb 
\plotone{figs/sim_subsampled2_stats_error_lin.png}
\caption{Distribution of median photometric uncertainties ('errors') in combined r-band real light curves. This shows that the PTF and ZTF segments have much larger errors than SDSS, PS1. This is the reason for using only SDSS-PS1 part of the combined light curve.}
\label{fig:combinedLCerrors}
\end{figure} 

%As we found earlier, extending the light curve baseline decreases the bias in the retrieved DRW parameters (Fig.~\ref{fig:rho_space}). For light curves that simulate the extension of SDSS with PS1, ZTF, LSST segments, Figs.~\ref{fig:simLCresults1} and show that this is indeed the case: the longer the light curves, that more the retrieved $\tau$ and $SF_{\infty}$ are centered on the input values. 

\begin{figure*}  % code2/Explore_simulation_results.ipynb 
\plotone{figs/Simulated_Jeff1-EXP-190401.png}
\caption{The ratio of DRW parameters fitted with \project{celerite}: $\tau$ and $\sigma$, to the input $\tau_{in} = 575 $d, $\sigma_{in} = 0.2 / \sqrt{2} {\sim} 0.14$  ($SF_{\infty}=0.2$ mag).  Each line corresponds to different segment of the combined SDSS-PS1-ZTF-LSST light curve. Extending the baseline by adding more data to each simulated quasar light curve allows to recover better the input parameters. The improvement with first year of ZTF data is not as large as with LSST, due to significant baseline increase compared to SDSS-PS1 (see Fig.~\ref{fig:lcExtent}). For each of the 9258 simulated light curves we employ real SDSS-PS1 cadence and photometric uncertainties (adding a Gaussian offset to the ideal underlying DRW signal), and simulated ZTF and LSST cadence and uncertainties based on the appropriate error model (see Sec.~\ref{sec:simulation})}
\label{fig:simLCresults1}
\end{figure*} 

\begin{figure*} % code2/Explore_simulation_results.ipynb 
\plotone{figs/macleod2011_Fig18_Jeff1-190401.png}
\caption{Comparison of retrieved parameters in relation to input parameters, shown as Fig.18 in \citet{macleod2011} }
\label{fig:simLCresults2}
\end{figure*} 


%
%
%
%
%
%
%
%

\section{Results: variability parameters for S82 Quasars}\label{sec:results}

We extend Stripe82 quasar light curves by combining the SDSS r-band data with  the PS1 $g_{\mathrm{PS1}}$, $r_{\mathrm{PS1}}$, $i_{\mathrm{PS1}}$ transformed into SDSS r-band. We fit the light curves with \project{celerite} DRW model for each quasar using either the SDSS or both SDSS and PS1 components. This yields two sets of DRW parameters per quasar: $(\tau_{SDSS}, \mathrm{SF}_{\infty, SDSS})$, and $(\tau_{SDSS-PS1}, \mathrm{SF}_{\infty, SDSS-PS1})$. Because the variability is inherent to the quasar, for the remaining analysis we shift all timescales to quasar rest frame, and implicitly assume that the DRW timescales are considered in rest frame: $\tau_{\mathrm{RF}} = \tau_{\mathrm{OBS}} / (1+z)$.


\begin{figure} % code2/Compare_Celerite_Chelsea_real_fits.ipynb 
\plotone{figs/MacLeod2010_Fig3_restframe_NEW_.png}
\caption{Comparison of distributions of the rest-frame variability timescale $\tau_{RF}$ against the  asymptotic variability amplitude $SF_{\infty}$, for M10 SDSS r-band,  and \project{celerite} fits using  SDSS or SDSS-PS1 segments of combined S82 quasar light curves. The M10 (red,solid lines) and this work, using only SDSS (dashed, blue), overlap, as we recover the same underlying distributions. }
\label{fig:tauRF_SFinf}
\end{figure} 


In this section we first correct fitted $\tau$, $SF_{\infty}$ for wavelength dependence. Then we show the consistency with M10 results,  and consider the trends between DRW parameters and physical quasar properties: black hole mass $M_{\mathrm{BH}}$, absolute i-band  magnitude $M_{i}$, or redshift $z$.  


\subsection{Comparison to M10}
As this work is based on extending the SDSS light curves (studied by M10) with the PS1 data, we directly compare the results to M10. The DRW parameters recovered with \project{celerite} are  broadly consistent - Fig.~\ref{fig:tauRF_SFinf} shows the rest-frame  $\tau$, and $SF_{\infty}$ distributions for our results for the SDSS segment (blue dashed line),  SDSS-PS1 combined light curves (green dot-dashed line), and  M10 SDSS for r-band only (red solid line). When using the same data as M10 (SDSS), our results agree. A detailed object-by-object comparison on Fig.~\ref{fig:celeriteCompare} reveals a slight offset between log-ratios of DRW parameters, which can be attibuted to software differences.


\begin{figure} % code2/Compare_Celerite_Chelsea_real_fits.ipynb  
\plotone{figs/Compare_Chelsea_r-band_Celerite_SDSS_EXP_NEW_single_.png}
\caption{Comparison of \project{celerite} fits using only the  SDSS r-band segments of S82 quasars ($\sigma_{fit}, \tau_{fit}$), against results of M10 for SDSS r-band ($\sigma_{M10}, \tau_{M10}$), object-by-object. The small offset can be attributed to software differences. See Fig.~\ref{fig:tauRF_SFinf} for a comparison of rest-frame $\tau$ and $SF_{\infty}$ distributions. This is similar to Fig.3 in M10, except we plot only the r-band SDSS results.}
\label{fig:celeriteCompare}
\end{figure} 


\subsection{Rest-frame Wavelength Correction}
Since quasars are located at non-negligible redshifts, the observed wavelengths (eg. through the SDSS bandpass) probe a region of shorter wavelengths in emitted spectrum, depending on the redshift (see \citealt{shen2018}, Fig.7):  $\lambda_{obs} = \lambda_{RF}  (1+z)$. Thus to compare variability of the same parts of the rest-frame spectrum, we need to estimate the rest-frame emission wavelength $\lambda_{RF}$, and correct for the dependence of $\tau, SF_{\infty}$ on $\lambda_{RF}$. Since in this study we only used r-band light curves,  we plot our SDSS and SDSS-PS1 results against M10 for SDSS ugriz, assuming for $\lambda_{obs}$ the center wavelength of each SDSS bandpass ($3520$, $4800$, $6250$, $7690$, $9110$ $\mbox{\AA}$ for $ugriz$, respectively), $\lambda_{RF} = \lambda_{obs} / (1+z)$ (using quasar redshift from \citealt{schneider2010}). Fig.~\ref{fig:wavelength_dependence} shows $\tau, SF_{\infty}$  as a function of  $\lambda_{RF}$. A power law dependency:
\begin{equation}
f \propto \left( \frac{\lambda_{RF}}{4000 \mbox{\AA}} \right)^{B}
\end{equation}
describes the relationship well (solid red line on each panel). Therefore we use the same coefficients as M10: $B=-0.479$ and $0.17$ for $SF_{\infty}$ and $\tau$, respectively.

  
\begin{figure} % code2/Compare_Celerite_Chelsea_real_fits.ipynb 
\plotone{figs/macleod2010_Fig13_kdeplot_Chelsea_cut_NEW_.png}
\caption{Rest-frame timescale $\tau$ (top panel), and asymptotic structure function $SF_{\infty}$ (bottom panel), as a function of rest-frame wavelength $\lambda_{RF}$. The background contours show M10 SDSS $ugriz$ data, and the foreground contours  denote our results using  SDSS (red) and SDSS-PS1 (orange) segments. The red line indicates the best-fit power law to M10 data, with $B=0.17$ an $-0.479$ for $\tau_{RF}$, and $SF_{\infty}$, respectively.}
\label{fig:wavelength_dependence}
\end{figure} 

\subsection{Quasar Properties: Black Hole Mass, Absolute Luminosity}

It is difficult to measure the mass of black holes living in the centers of active galaxies, even provided a detailed spectrum.  The most common  approach to estimate black hole masses in AGN is to assume that the broad-line region (BLR) is virialized:


\begin{equation}
M_{BH} = f \frac{ R\Delta V^{2} }{G} = f M_{vir}
\end{equation}


where $f$ is a constant of order unity, $R$ is the size of the BLR (estimated from  emission-line lag $\Delta t$ as $R = c \Delta t$), $\Delta V$ is virial velocity, $G$ gravitational constant \citep{shen2008}.  From reverberation mapping studies (eg. \citealt{shen2018}) we know that continuum luminosity $L$ is related to the size of the BLR region as $R \propto L^{\gamma}$ \citep{vestergaard2006}, with $\gamma$ very close to $1/2$ (eg. \citealt{bentz2009} finds from RM studies $\gamma = 0.519 \pm 0.06$). Thus we find  that $R \Delta V^{2} \propto L^{\gamma} \Delta V^{2} \equiv \mu$. The virial velocity $\Delta V$ is usually estimated from the width of the broad emission lines (or line dispersion).  In the absence of a quasar spectrum, there are alternative methods using a conversion of the broad-band photometry into monochromatic fluxes in the vicinity of reverberating lines (eg. \citealt{kozlowski2015}, used in \citealt{kozlowski2017b} to estimate black hole mass for 280 000 AGN).   
Depending on the redshift, different rest-frame calibrated emission lines shift into the observed passband: broad H$\alpha$ at  $6562\mbox{\AA}$, H$\beta$ at  $5100 \mbox{\AA}$, Mg\,{\sc ii} at $3000\mbox{\AA}$, and C\,{\sc iv} at $1350\mbox{\AA}$ (see Fig.7 in \citealt{shen2018}, and \citealt{vestergaard2002}). Some authors even consider separately C\,{\sc iv}-based and  Mg\,{\sc ii}-based  black hole mass estimates. We refer the reader to \citet{shen2008} who in detail describes various biases and inherent assumptions of virial black hole mass measurements. 



Another important quasar property - bolometric luminosity, is most often estimated from the absolute i-band magnitude, $M_{i}$ (see \citealt{shen2008}, Fig.2). $M_{i}$ is derived from the observed i-band magnitude, by correcting for Galactic extinction, and correcting for the fact that at different redshifts different portions of the spectral energy distribution are observed by the telescope filter bandpass. The latter, known as K-correction $K(z)$ \citep{oke1968},  is defined as $m_{intrinsic} = m_{observed} - K(z)$. In the early 2000`s the common approach was to K-correct to redshift 0, but as \citep{richards2006a} pointed out, since the distribution of quasars peaks at redshift 2, for most quasars correcting to the redshift of 0 required shifting the observed spectrum into the far infrared. Moreover, the procedure was to correct separately for the continuum and emission line contributions, assuming a particular spectral shape (eg. power law  $f_{\nu} \propto \nu^{\alpha}$, with $\alpha=-0.5$ - see \citealt{schneider2010, vandenberk2001, richards2006a}).  This introduces a larger error for K(z=0) than for K(z=2) if the assumed spectral shape $\alpha=-0.5$ is far from the real spectral index. In early 2010`s, after  \citealt{richards2006a, wisotzki2000, blanton2003},  the practice started shifting towards K-correcting to redshift 2,  and including custom qusasar spectral shapes, as reflected by the content of \cite{shen2011} quasar catalog. Thus in this study we use  the absolute i-band magnitude K-corrected to z=2: $M_{i}(z=2)$. % from \citep{shen2011}. 



These methods were used to create catalogs of quasar properties derived from spectra. Since quasars are variable at ${\sim}0.2$ mag level, the ideal is to use a single-epoch calibrated spectrum to estimate the continuum luminosity, and find virial black hole masses using relationships based on the monochromatic fluxes and broad line widths described above. A glance at the available quasar catalogs reveals that, given any SDSS data release, there is indeed first a catalog of basic quasar properties (redshift and photometry - eg. \citealt{schneider2007, schneider2010}), and more detailed catalogs containing black hole masses and bolometric luminosities  follow (eg. \citealt{shen2008, shen2011}). More recently, once SDSS DR12 Quasar Catalog \citep{paris2017} was released,  \citealt{kozlowski2017a} followed using SDSS photometry as a proxy for monochromatic luminosities. \citet{chen2018} added a detailed analysis of continuum luminosities in the  H$\alpha$, H$\beta$ regions for low-redshift quasars. Using the spectra from Chinese LAMOST survey \citet{dong2018} also sought to estimate virial black hole masses, and the results, while consistent with \citet{shen2011}, suffered from the necessity to peg the non-calibrated spectra to the SDSS photometry which was taken  a different epoch. Thus even though the SDSS DR12 Quasar Catalog of \cite{paris2018}  is the most recent, like \citet{paris2017} it lacks black hole masses and bolometric luminosities, and there is no recent work that re-analyzed the spectral data. Therefore we use black hole mass estimates and monochromatic luminosities from \citet{shen2011}, based directly on single-epoch spectra. 


\subsection{Trends in Quasar Properties}
%\citep{sanchez2018}.

\begin{figure*}
\plotone{figs/macleod2010_Fig12_Shen2011.png}
\caption{Distribution of quasars as a function of  redshift, observed i-band magnitude, absolute i-band magnitude (K-corrected to z=2), and virial black hole mass. All quantities from \citep{shen2011}. }
\label{fig:quasar_properties}
\end{figure*} 

First, we consider quasar physical properties, which were updated between \citealt{shen2008} (M10) and  \citealt{shen2011} (used here).  Fig.~\ref{fig:quasar_properties} shows the distribution of quasars as a function of redshift $z$, absolute magnitude $M_{i}$, and black hole mass $M_{BH}$, and illustrates the fact that some trends are due to selection effects. For instance, the trend of increasing redshift with $M_{i}$ on the upper left and bottom left panels, is partially due to the fact that quasars have to be brighter to be included in the survey at increasing distances (luminosity-redshift degeneracy: see Sec.5, Fig.12 in M10, and \citealt{dong2018}). Higher redshift quasars are also more active  and have higher black hole masses due to cosmological downsizing (see \citealt{babic2007,labita2009, mclure2004}).



\subsection{Trends with Luminosity, Black Hole Mass, and Redshift}

Fig.~\ref{fig:sf_tau_bh_mass_luminosity} shows $\tau$ and $SF_{\infty}$ as a function of $M_{BH}$, $M_{i}$, and $z$. Correlations between variability parameters and the physical properties of quasars are of high utility. In the era of large synoptic surveys (ZTF, LSST), only a few percent of AGN with optical time-series will be followed up with spectroscopy \citep{ivezic2019}. A relationship where $(\tau$,$SF_{\infty}) = f(M_{i}, M_{BH})$ would allow to infer black hole masses  and luminosities for millions of quasars.

\begin{figure*}
\plotone{figs/macleod2010_Fig14_Shen2011_sdss-ps1.png}
\caption{Long-term variability ($SF_{\infty}$), and characteristic timescale ($\tau$), as a function of absolute i-band magnitude (K-corrected to redshift 2, proxy for bolometric luminosity), virial black hole mass, and redshift. }
\label{fig:sf_tau_bh_mass_luminosity}
\end{figure*} 



Some trends are dominant. For instance, brighter quasars have lower variability amplitude (upper-left panel), and this trend is largely independent of black hole mass (bottom-left panel). 

We investigate these correlations in more detail by fitting to $\tau$ or $SF_{\infty}$ (called $f$ below) a power law, using a Bayesian linear regression method \citep{kelly2007b} that incorporates measurement uncertainties in all latent variables  : 


\begin{eqnarray}
\label{eq:powlawmodel}
\log_{10}{f} = &A& + B \log_{10}\left( \lambda_{RF} / 4000 \mbox{\AA} \right) + C (M_{i} + 23) \nonumber \\
&+& D \log_{10}{\left( M_{BH} / 10^{9} M_{\odot}  \right)} 
\end{eqnarray} 

M10 fitted  this model independently to each of the five SDSS bands, reporting the band-averaged coefficients - Fig.~\ref{fig:MacLeodShen2011} shows the example of posterior samples for $f=SF_{\infty}$, illustrating that each band yields slightly different results.


\begin{figure*}
\plotone{figs/Chelsea_ugriz_Shen2011_SF.png}
\caption{Table 1 in M10 reported the band-averaged values for fit coefficients A,C,D for Eq.~\ref{eq:powlawmodel}.  Shown here are samples from posterior MCMC draws using M10 results for  $f=SF_{\infty}$, against \citealt{shen2011} $M_{i}$ and $M_{BH}$. Because the mean of posterior for SDSS r-band results only  (bold, solid lines) are different from the mean of band-averaged values (dashed lines),  we compare our SDSS-PS1 combined r-band results against M10 SDSS r-band only.}
\label{fig:MacLeodShen2011}
\end{figure*} 

Since we use combined SDSS-PS1 r-band data, we only compare our results to M10 SDSS r-band.  Figs.~\ref{fig:MacLeodCeleriteShen2011tau} and ~\ref{fig:MacLeodCeleriteShen2011sf} show the MCMC posterior draws from fitting Eq.~\ref{eq:powlawmodel}  to M10, as well as our SDSS, SDSS-PS1 results (S19, this work). Mean and standard deviation of each distribution are summarized in Table~\ref{tab:coefficients}.  For $f{=}\tau$ above, Fig.~\ref{fig:MacLeodCeleriteShen2011tau} shows that the new data supports a stronger dependence of variability amplitude on quasar luminosity by $0.04$ dex ($C{\sim}0.06$ rather than ${\sim}0.02$), but a smaller dependence on the black hole mass ($D{\sim}0.09$ rather than ${\sim}0.16$). For $f{=}SF_{\infty}$ above, Fig.~\ref{fig:MacLeodCeleriteShen2011sf} shows that there is a weaker trend with both luminosity ($C{\sim}0.105$ rather than ${\sim}0.117$) and black hole mass ($D{\sim}0.11$ rather than ${\sim}0.12$), although the latter is a very small change. 

% https://github.com/AASJournals/AASTeX60/issues/61 
\begin{deluxetable*}{cc|CCCC}
%\tablewidth{\pagewidth}
\tablecaption{Comparison of best-fit coefficients for Eq.~\ref{eq:powlawmodel} using M10 results, and this work (S18). B is set to $0.17$ or $0.479$ from fitting a power law between $\lambda_{RF}$ and $\tau$, $SF_{\infty}$ (see Fig.~\ref{fig:wavelength_dependence}). For $f=\tau$, both C and D are almost the same between M10 and this work using SDSS only (rows 1 and 2). When using SDSS-PS1,  C increases, and D is smaller than before (row 3). For $f = SF_{\infty}$, C and D are also the same between M10 and S19, SDSS (rows 4, 5), but when adding PS1 data, C and D become smaller (row 6). \label{tab:coefficients} } 

\tablehead{\colhead{$f$} & \colhead{Source} & \colhead{$A$(offset)} & \colhead{$B(\lambda_{RF})$} & \colhead{$C (M_{i})$} & \colhead{$D (M_{\mathrm{BH}})$} }
\startdata
$\tau$ & M10, SDSS & $2.432\pm0.026$ & $0.17\pm0.02$ & $0.011\pm0.009$ & $0.163\pm0.026$ \\
 & S19, SDSS & $2.603\pm0.021$ & $0.17\pm0.02$ & $0.022\pm0.007$ & $0.164\pm0.021$ \\
 \tableline
 & S19, SDSS-PS1 & $2.232\pm0.029$ & $0.17\pm0.02$ & $0.064\pm0.01$ & $0.094\pm0.029$ \\
 \tableline
SF$_{\infty}$ & M10, SDSS & $-0.489\pm0.011$ & $-0.479\pm0.005$ & $0.117\pm0.004$ & $0.12\pm0.011$ \\
 & S19, SDSS & $-0.517\pm0.01$ & $-0.479\pm0.005$ & $0.117\pm0.003$ & $0.12\pm0.01$ \\
  \tableline
  & S19, SDSS-PS1 & $-0.467\pm0.009$ & $-0.479\pm0.005$ & $0.105\pm0.003$ & $0.111\pm0.009$ \\
 \enddata
\end{deluxetable*}




\begin{figure*}
\plotone{figs/posterior_IDL_Chelsea_Celerite_CD_MI_Z2_tau.png}
\caption{Distribution of posterior draws from MCMC  Eq.~\ref{eq:powlawmodel}, for $f=\tau$ with SDSS only (dashed green line), or SDSS-PS1 (dot-dashed yellow line) combined quasar light curves, against M10 SDSS r-band.  The results from SDSS-only portion are consistent with M10 for the single band. Inclusion of the PS1 portion decreases the timescale dependence on black hole mass, but increases the luminosity dependence. This can be understood as a rotation of the plane in ($\tau$, $M_{i}$, $M_{BH}$) coordinates.  }
\label{fig:MacLeodCeleriteShen2011tau}
\end{figure*} 



\begin{figure*}
\plotone{figs/posterior_IDL_Chelsea_Celerite_CD_MI_Z2_SF.png}
\caption{Same as Fig.~\ref{fig:MacLeodCeleriteShen2011tau},  but fitting quasar absolute magnitude, and black hole mass  in Eq.~\ref{eq:powlawmodel} as a function of the asymptotic amplitude $f = SF_{\infty}$. New data from PS1 supports a weaker dependence of variability amplitude with luminosity and black hole mass. }
\label{fig:MacLeodCeleriteShen2011sf}
\end{figure*} 



\subsection{Comparison to other studies: Eddington ratio}

Dependence of both $\tau$ and $SF_{\infty}$ on black hole mass and quasar luminosity can be understood as a function of accretion rate, encoded in the Eddington ratio: $f_{Edd} {=} L_{Bol}/L_{\mathrm{Edd}}$, where $L_{\mathrm{Edd}} {=} 1.26 {\times} 10^{38} (M_{\mathrm{BH}} / M_{\odot})$ erg/s \citep{shen2011}. Fig.\ref{fig:EddingtonRatio} shows the dependence of $f_{Edd}$ on $M_{i}$, $M_{BH}$ and $\mathrm{SF}_{\infty}$, using SDSS-PS1 data. We find a much steeper power-law slope of $-0.9$, against $-0.23$ reported in M10 from SDSS data. 


\begin{figure*}
\plotone{figs/Eddington_ratio_Shen2011_SDSS-PS1.png}
\caption{Left : the Eddigton ratio $L/L_{Edd}$ (from \citealt{shen2011}) plotted as a function of $M_{BH}$  vs $M_{i}$. Right:  the asymptotic variability amplitude $SF_{\infty}$,  corrected for the wavelength dependence to $4000 \AA$, as a function of the Eddington ratio.  The slope of -0.9 (solid line) is much steeper than -0.23 found by M10 with the SDSS only data. }
\label{fig:EddingtonRatio}
\end{figure*} 

Additional PS1 data extending SDSS quasar light curves suggests that characteristic variability timescale is more strongly dependent on luminosity. This is consistent with \cite{sun2018}, who concluded with Structure Function analysis of their luminosity-matched quasar sample, that $\tau$ depends mostly on the bolometric luminosity. 


Anticorrelation of  variability amplitude with the Eddington ratio  has a variety of possible theoretical explanations. In the thin disk theory \cite{shakura1973, frank2002, netzer2013}, radius of the emission region at given wavelength increases with the Eddington ratio, and is inversely proportional to temperature\cite{rakshit2017}. Thus a hotter disk means that the emission observed in a given bandpass is emitted from a larger radius. From causality, a smaller region can be more variable than a larger one. Therefore, a  hotter disk would be less variable at a given wavelength than a colder one, and  the variability amplitude as studied in a particular bandpass (here, SDSS r-band) would be anticorrelated with Eddington ratio \citep{fausnaugh2016,edelson2015}. 

On the other hand, in the strongly inhomogeneous disk model \citet{dexter2011} independent temperature fluctuations in $N$ zones drive the variability. In that framework the inverse trend of variability amplitude against $L/L_{Edd}$  and $L_{Bol}$  can be understood qualitatively if more luminous quasars also have higher mass accretion rate, and thus greater number of disk inhomogeneities, resulting in smaller flux variability \citep{simm2016}. The inhomogeneous disk model was consistent with mean SDSS spectral analysis in \citet{ruan2014}, but was not a preferreed explanation for \citet{kokubo2015}. 

Both \citealt{rumbaugh2018} (with Dark Energy Survey structure function study) and  \citet{sun2018}  (with a low-z subsample of S82 SDSS quasars) confirm the anti-correlation between quasar variability and luminosity. However, \citealt{graham2019} do not find support for this trend with the sample extremely variable quasars (EVQs) in the CRTS dataset,  but when selecting for lower luminosity sources ($M_{V} < -23$), the anti-correlation is recovered. This is in accord with an interpretation that a dwindling fuel supply may correspond to higher variability. Furthemore, \citet{sanchez2018} combined the SDSS spectra with 5 year light curves of 2345 quasars obtained with Quasar Equatorial Survey Team (QUEST)-La Silla AGN Variability Survey, and  using the Bayesian parametrization of Structure Function \citep{schmidt2010},  SF$(\tau) = A(\tau/1 \mathrm{yr})^{\gamma} $, they also found that the amplitude of variability $A$ is anti-correlated with rest-frame emission wavelength,  and Eddington ratio (also see \citealt{simm2016}, \citealt{rakshit2017}).

Indeed, $f_{Edd}$ is a proxy for the strength of accretion, which together with orientation may be the key to explaining quasar main sequence (QMS) \citep{shen2014, marziani2018}. The QMS, defined by so-called Eigenvector-1, is the the anti-correlation between the broad line Fe{\sc ii} emission, and the strength of the narrow O{\sc iii} ($5007 \mbox{\AA}$) line \citep{wang1996}. An analysis of quasar clustering \citep{shen2014}, later confirmed by \citet{sun2015} with measurements of  black hole mass from the quasar host galaxy stellar dispersion \citep{ferrarese2000, kormendy2013}, showed that the entire diversity of quasars in  QMS can be explained by the variation in accretion (affecting $R_{\mathrm{Fe  II}}$ - the ratio of the  Fe{\sc ii} EQ  between $4435-4685$ $\mbox{\AA}$ and H$\beta$), or orientation effects (affecting the FWHM of the H$\beta$). However, \citet{panda2019} found that these are insufficient, and variations in metallicity, as well as a range of cloud densities, and turbulences are required. \cite{jiang2016} also found that metallicity, and in particular the iron opacity bump, may have a strong influence on the stability of an accretion disk, and thus linking metallicity to AGN variability. This is also consistent with findings of \cite{sun2018}: quasars with high  Fe{\sc ii} strength have higher metallicity, and have more stable disks. 



\subsection{Variability Timescales}

In the era of changing-look active galaxies ( including initially distinct classes of Changing-Look Quasars \citep{lamassa2015, macleod2019}, Changing-Look AGNs \citep{marchese2012, bianchi2009,risaliti2009}, Changing-Look LINERS \citep{frederick2019} to name a few) there is a revived interest in possibly linking the behavior of stellar-sized accreting systems (eg. Black Hole Binaries),  to that of galactic scale (eg. AGN, QSO, LINERS)\citep{noda2018, ruan2019}. 

It appears that there are several timescales at play, and possibly several interlinked mechanisms that drive the variability. 

There is a hierarchy of relevant timescales in the standard optically thick, geometrically thin, $\alpha$-disk model : dynamical, thermal, front, viscous, with   $t_{dyn} < t_{th} < t_{front}  < t_{visc} $ \citep{netzer2013, frank2002}.

The dynamical, or gas orbital, timescale is simply  an inverse of the Keplerian orbital angular frequency $ \Omega$  at radius R  : 

\begin{equation}
t_{dyn} {\sim}  1 / \Omega = \left( \frac{GM}{R^{3}}\right)^{-1/2}
\end{equation}


The main parameter  describing the accretion disk is $\alpha$ - the ratio of the (vertically averaged) total stress to thermal (vertically averaged) pressure \citep{lasota2016} : 

\begin{equation}
\alpha= \frac{\langle \tau_{r\varphi}  \rangle_{z} }{\langle P \rangle _{z}} 
\end{equation}


After \cite{lasota2016},  the hydrodynamical stress tensor (corresponding to  kinematic viscocity $\nu$) is:

\begin{equation}
\tau_{r\varphi } = \rho \nu \frac{\partial v_{\varphi}}{\partial R} = \rho \nu \frac{d \Omega}{d \ln{R}} = \frac{3 \rho \nu \Omega}{2}  
\end{equation}

so  with  $c_{s}$ -  local sound speed at radius $R$ (isothermal sound speed is $c_{s} = \sqrt{P/\rho}$),

\begin{equation}
\alpha  =   \frac{3 \rho \nu \Omega}{2 P} =  \frac{3 \Omega \nu}{2 c_{s}^{2}}
\end{equation}



This means that smaller $\alpha$ corresponds to less viscous disks. 


The thermal timescale, related to the time needed for re-adjustment to the thermal equilibrium (derived in detail in \cite{frank2002}), is the ratio of heat content per unit disk area to dissipation rate per unit disk area: $(dE / A) / (dE/dt /  A) = dt $.  The heat content per unit volume is ${\sim} \rho k T / \mu m_{p} {\sim} \rho c_{s}^{2}$, and heat content per unit area is  ${\sim} \rho c_{s}^{2} / h {\sim} \Sigma c_{s}^{2}$. Meanwhile, the dissipation rate per unit area, $D(R)$, is 

\begin{equation}
D(R) = \frac{9}{8} \nu \Sigma R^{-3} G M
\end{equation}

(eq. 4.30 in \citealt{frank2002}), so :

\begin{equation}
t_{th} {\sim} \frac{c_{s}^{2}R^{3}}{G M \nu } = \frac{c_{s}^{2}}{\nu \Omega} = \frac{t_{dyn}}{\alpha}
\end{equation}

Thus if the disk were inviscid ($\nu \rightarrow 0$), then $t_{th}\rightarrow\infty$ i.e. there is no contact with adjacent disk elements. 

The cooling and heating fronts propagate through the disk at  $\alpha c_{s} $ \citep{hameury2009}  - in that description  with no viscosity there is no communication between neighboring disk annuli, and thus no front propagation \citep{balbus1998, balbus2003}. Following \cite{stern2018}, if we define as $h/R$  the the disk aspect ratio, with the disk height $h = c_{s} / \Omega$, the characteristic time for front propagation is:

\begin{equation}
t_{front} {\sim} (h/R) ^ {-1} t_{th}
\end{equation}


The viscous timescale is the characteristic time it would take for a parcel of material to undergo a radial transport due to the viscous torques from the radius $R$ to the black hole \citep{czerny2006}. Note that while viscosity has probably magnetic origin \citep{eardley1975, grzedzielski2017}, in this simplistic order of magnitude estimate we use a hydrodynamical description of accretion flow.  With $\nu = \eta / \rho$ (kinematic viscosity being the ratio of dynamical viscosity to density), \cite{frank2002} shows (Chap.5.2) that 

\begin{equation}
t_{visc} {\sim} R^{2} / \nu {\sim}  R / v_{R} = (h/R)^{-2} t_{th}
\end{equation}

 We can parametrize each timescale for a black hole mass $M_{BH} = 10^{8} M_{\odot}$, at $R {\sim} 150 r_{g}$, with the gravitational radius $r_{g} = GM_{BH} / c^{2} {\sim} 4 \mathrm{au}$, using Eqs.5-8 in \cite{stern2018} : 


 \begin{equation}
 t_{dyn} {\sim} 10  \mathrm{days} \left(\frac{M_{\mathrm{BH}}}{10^{8} M_{\odot}} \right) 
 \left( \frac{R}{150 r_{g}}\right) ^{3/2} 
 \end{equation}

 \begin{equation}
 t_{th}   {\sim} 1 \,\mathrm{year} \left( \frac{\alpha}{0.03}\right)^{-1}  
 \left( \frac{M_{\mathrm{BH}}}{10^{8} M_{\odot}}\right) \left( \frac{R}{150 r_{g}}\right)^{3/2} 
 \end{equation}

  \begin{eqnarray}
  t_{front} {\sim} 20 \,\mathrm{years} \left( \frac{h/R}{0.05}\right)^{-1}   \left( \frac{\alpha}{0.03}\right)^{-1}  \nonumber  \\ 
  \left( \frac{M_{\mathrm{BH}}}{10^{8} M_{\odot}}\right)     \left( \frac{R}{150 r_{g}}\right) ^{3/2} 
 \end{eqnarray}

  \begin{eqnarray}
  t_{visc}  {\sim} 400 \, \mathrm{years} \left( \frac{h/R}{0.05}\right)^{-2}   \left( \frac{\alpha}{0.03}\right)^{-1} \nonumber  \\  
  \left(\frac{M_{\mathrm{BH}}}{10^{8} M_{\odot}} \right)     \left( \frac{R}{150 r_{g}}\right) ^{3/2}  
 \end{eqnarray}


In summary,  of   considered timescales only thermal and dynamical are short enough to be related to  the observed short-term stochastic variability. It may be that the variability on the scale of days is driven by local changes, and on the longer scale (perhaps hundreds of days) by a different mechanism\citep{kokubo2015}. 

The variability on several years timescale could also be explained by the X-ray reprocessing model \cite{kubota2018}, assuming that the AGN UV-optical variability is a results of reprocessing of X-ray or far-UV emission \citep{krolik1991}. However, see discussion in \citet{kokubo2015} - there RM studies seems to support that picture \citep{mchardy2018}, but there is conflicting evidence \citep{edelson2014, zhu2018}. 

If  the  stochastic variability  on ${\sim} 0.2$ mag ($10-20\%$) level can be explained by thermal fluctuations, the other time scales may be more related to the dramatic changes in brightness of the continuum as observed in changing-look AGN. \citet{noda2018} favor a change in mass accretion rate, followed by a propagation of the cooling front \cite{lawrence2018, simm2016}.  










\section{Conclusions}\label{sec:conclusions}




\section{Acknowledgements}
%
%
%
%
%
%

% FROM https://panstarrs.stsci.edu 
The Pan-STARRS1 Surveys (PS1) and the PS1 public science archive have been made possible through contributions by the Institute for Astronomy, the University of Hawaii, the Pan-STARRS Project Office, the Max-Planck Society and its participating institutes, the Max Planck Institute for Astronomy, Heidelberg and the Max Planck Institute for Extraterrestrial Physics, Garching, The Johns Hopkins University, Durham University, the University of Edinburgh, the Queen's University Belfast, the Harvard-Smithsonian Center for Astrophysics, the Las Cumbres Observatory Global Telescope Network Incorporated, the National Central University of Taiwan, the Space Telescope Science Institute, the National Aeronautics and Space Administration under Grant No. NNX08AR22G issued through the Planetary Science Division of the NASA Science Mission Directorate, the National Science Foundation Grant No. AST-1238877, the University of Maryland, Eotvos Lorand University (ELTE), the Los Alamos National Laboratory, and the Gordon and Betty Moore Foundation.

% FROM  https://www.sdss.org/collaboration/citing-sdss/
Funding for the Sloan Digital Sky Survey IV has been provided by the Alfred P. Sloan Foundation, the U.S. Department of Energy Office of Science, and the Participating Institutions. SDSS-IV acknowledges
support and resources from the Center for High-Performance Computing at
the University of Utah. The SDSS web site is www.sdss.org.

SDSS-IV is managed by the Astrophysical Research Consortium for the 
Participating Institutions of the SDSS Collaboration including the 
Brazilian Participation Group, the Carnegie Institution for Science, 
Carnegie Mellon University, the Chilean Participation Group, the French Participation Group, Harvard-Smithsonian Center for Astrophysics, 
Instituto de Astrof\'isica de Canarias, The Johns Hopkins University, Kavli Institute for the Physics and Mathematics of the Universe (IPMU) / 
University of Tokyo, the Korean Participation Group, Lawrence Berkeley National Laboratory, 
Leibniz Institut f\"ur Astrophysik Potsdam (AIP),  
Max-Planck-Institut f\"ur Astronomie (MPIA Heidelberg), 
Max-Planck-Institut f\"ur Astrophysik (MPA Garching), 
Max-Planck-Institut f\"ur Extraterrestrische Physik (MPE), 
National Astronomical Observatories of China, New Mexico State University, 
New York University, University of Notre Dame, 
Observat\'ario Nacional / MCTI, The Ohio State University, 
Pennsylvania State University, Shanghai Astronomical Observatory, 
United Kingdom Participation Group,
Universidad Nacional Aut\'onoma de M\'exico, University of Arizona, 
University of Colorado Boulder, University of Oxford, University of Portsmouth, 
University of Utah, University of Virginia, University of Washington, University of Wisconsin, 
Vanderbilt University, and Yale University.

% I didn't use CRTS data in the end  ...
%
% from http://nesssi.cacr.caltech.edu/DataRelease/policy.html  
% 
% The CSS survey is funded by the National Aeronautics and Space
% Administration under Grant No. NNG05GF22G issued through the Science
% Mission Directorate Near-Earth Objects Observations Program.  The CRTS
% survey is supported by the U.S.~National Science Foundation under
% grants AST-0909182.


%%%%%%%%%%%%%%%%%%%%%%%%%%%%%%%%%%%%%%%%%%%%%%%%%%

%%%%%%%%%%%%%%%%%%%% REFERENCES %%%%%%%%%%%%%%%%%%

% The best way to enter references is to use BibTeX:

\bibliographystyle{aasjournal} 
\bibliography{references}

%%%%%%%%%%%%%%%%%%%%%%%%%%%%%%%%%%%%%%%%%%%%%%%%%%

\end{document}

% End of file `sample62.tex'.
